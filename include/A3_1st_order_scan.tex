\newpage
\section{Asymmetric NGS model parameter scan results}
\label{sec:appendix_1st_order_scan}

\begin{figure}
    \vspace{-0.5cm}
    \captionsetup[subfigure]{justification=centering}
    \centering
    \begin{subfigure}{0.5\textwidth}
        \centering
        \includegraphics[width=\textwidth]{figure/scan_1st_order/P1_at_r_1.pdf}
        \caption{}
    \end{subfigure}%
    \begin{subfigure}{0.5\textwidth}
        \centering
        \includegraphics[width=\textwidth]{figure/scan_1st_order/T1_at_r_1.pdf}
        \caption{}
    \end{subfigure}
    \begin{subfigure}{0.5\textwidth}
        \centering
        \includegraphics[width=\textwidth]{figure/scan_1st_order/U1_at_r_1.pdf}
        \caption{}
    \end{subfigure}%
    \begin{subfigure}{0.5\textwidth}
        \centering
        \includegraphics[width=\textwidth]{figure/scan_1st_order/V1_at_r_1.pdf}
        \caption{}
    \end{subfigure}
    \begin{subfigure}{0.5\textwidth}
        \centering
        \includegraphics[width=\textwidth]{figure/scan_1st_order/Q1_at_r_1.pdf}
        \caption{}
    \end{subfigure}%
    \begin{subfigure}{0.5\textwidth}
        \centering
        \includegraphics[width=\textwidth]{figure/scan_1st_order/E1_at_r_1.pdf}
        \caption{}
    \end{subfigure}%
    \caption{Optimized sonic values $y_1(r=1)$ of our asymmetric perturbation model for the neutral ablation cloud dynamics. The relation to $E_\text{rel}/q_\text{rel}$ is shown to be linear, while $\gamma$ and $E_\star(E_\text{bc})$ change the slope slightly. The vertical dashed line is at $E_\text{rel}/q_\text{rel} = -1.17$, which is the value where $p_1(r_\text{p})$, i.e. the pellet rocket force, becomes negative. Note that $v_{1,\theta}(r=1)$ shares the same root.}
    \label{fig:perturbation_sonic_values}
\end{figure}