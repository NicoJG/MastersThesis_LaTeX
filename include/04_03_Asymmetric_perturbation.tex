\subsection{Asymmetric NGS model}
\label{ssec:asymmetric_perturbation_model}

Given the procedure described in the previous sections to calculate the full ablation dynamics under the assumption of spherical symmetry, it is now possible to describe the asymmetry in the ablation dynamics by a perturbative model.
The baseline model is the previously described isotropic NGS model, which will be assumed as fully known.
It will be denoted from now on with the subscript 0.
Assuming the real ablation process is described well by the isotropic NGS model quantities $y_0(r)$ with a small correction $\var{y}(r,\theta)$, which depends on the asymmetry along one axis ($z$-axis in \cref{fig:neutral_cloud_heating}), the full physical quantities can be modelled as
\begin{equation}
\begin{aligned}
    \rho &= \rho_0 + \var\rho\, , & 
    p &= p_0 + \var{p}\, , \\
    T &= T_0 + \var{T}\, , & 
    \vec{v} &= \hat{r} (v_0 + \var{v}_r) + \hat{\theta} (0+\var{v}_\theta)\, , \\
    q &= q_0 + \var{q}\, , & 
    E &= E_0 + \var{E}\, .
\end{aligned}
\end{equation}
The spherical coordinate system is chosen as depicted in \cref{fig:neutral_cloud_heating,fig:pellet_surface_force}, while any $\varphi$-dependence is neglected because it would not contribute to the pellet rocket force.

The system is treated as a quasi-steady state ideal gas, as described at the beginning of \cref{sec:neutral_gas_shielding}.
Therefore, \cref{eq:ideal_gas_law,eq:full_mass_conservation,eq:full_momentum_conservation,eq:full_energy_conservation,eq:heat_flux_approximation,eq:full_effective_heat_flux,eq:full_electron_energy_loss} form the basis of our asymmetric NGS model.
This includes the approximation of a radial flux of electrons being equivalent to the actually parallel flux along the magnetic field lines, as represented by \cref{eq:heat_flux_approximation}.
Linearizing the set of equations in the perturbation quantities $\var{y}$, combined with the fact that the quantities $y_0$ satisfy those equations themselves, yields
\begin{gather}
    \shortintertext{\centering the ideal gas law}
    \var{\rho} = m \frac{\var{p}}{T_0} - m \frac{p_0}{T_0^2} \var{T} \, , \label{eq:linearized_ideal_gas_law} \\
    %
    \shortintertext{\centering the mass conservation equation}
    \vec{\nabla} \cdot (\var{\rho} \vec{v}_0 + \rho_0 \var{\vec{v}}) = 0 \, , \label{eq:linearized_mass_conservation} \\
    %
    \shortintertext{\centering the momentum conservation equation}
    \rho_0 (\vec{v}_0 \cdot \vec{\nabla})\var{\vec{v}} + \rho_0 (\var{\vec{v}} \cdot \vec{\nabla})\vec{v}_0 + \var{\rho} (\vec{v}_0 \cdot \vec{\nabla}) \vec{v}_0 = - \vec{\nabla}\var{p} \, , \label{eq:linearized_momentum_conservation} \\
    %
    \shortintertext{\centering the energy conservation equation}
    \vec{\nabla} \cdot \left[ \left( \frac{1}{2}\rho_0 v_0^2 + \frac{\gamma}{\gamma - 1}p_0 \right)\var{\vec{v}} + \left( \frac{1}{2}\var{\rho} v_0^2 + \rho_0 (\vec{v}_0\cdot \var{\vec{v}}) + \frac{\gamma}{\gamma - 1}\var{p} \right)\vec{v}_0 \right] = \mu \pdv{\var{q}}{r} \, , \label{eq:linearized_energy_conservation} \\
    %
    \shortintertext{\centering the electron energy loss equation}
    \pdv{\var{E}}{r} = 2\frac{\var{\rho}}{m}L(E_0) + 2 \frac{\rho_0}{m} \eval{\pdv{L}{E}}_{E_0}  \!\!\!\! \var{E} \label{eq:linearized_effective_heat_flux} \\
    %
    \shortintertext{\centering and the effective heat flux equation}
    \pdv{\var{q}}{r} = \frac{\var{\rho}}{m} q_0 \Lambda(E_0) + \frac{\rho_0}{m} \var{q} \Lambda(E_0) + \frac{\rho_0}{m} q_0 \eval{\pdv{\Lambda}{E}}_{E_0} \!\!\!\! \var{E} \, . \label{eq:linearized_electron_energy_loss}
\end{gather}

Now we want to find a set of equations for the $r$-dependence and a separate set of equations for the $\theta$-dependence.
Fortunately, this is possible without further approximations by expanding the perturbation in terms of general fully orthogonal basis functions $\{X_l(\theta)\}$ in the form
\begin{align}
\begin{aligned}
    \var{\rho} &= \sum_l \rho_l(r) X_l(\theta) \, , &
    \var{p} &= \sum_l p_l(r) X_l(\theta) \, , \\
    \var{T} &= \sum_l T_l(r) X_l(\theta) \, , &
    \var{v}_r &= \sum_l v_{l,r}(r) X_l(\theta) \, , \\
    \var{q} &= \sum_l q_l(r) X_l(\theta) \, , &
    \var{E} &= \sum_l E_l(r) X_l(\theta) \, , &
\end{aligned}
\end{align}
while expanding only $\var{v}_\theta$ in terms of a different general basis\footnote{We initially assume no relation between $X_l$ and $X'_l$.} $\{X'_l(\theta)\}$ as
\begin{equation}
    \var{v}_\theta = \sum_l v_{l,\theta}(r) X'_l(\theta) \, .
\end{equation}


% the set of \cref{eq:linearized_ideal_gas_law,eq:linearized_mass_conservation,eq:linearized_momentum_conservation,eq:linearized_energy_conservation,eq:linearized_effective_heat_flux,eq:linearized_electron_energy_loss}
Taking the derivatives in terms of spherical coordinates, the only $\theta$-derivative appearing in \cref{eq:linearized_mass_conservation,eq:linearized_energy_conservation} is
\begin{equation*}
    \vec{\nabla}\cdot(\var{v}_\theta \hat{\theta}) = \frac{1}{r \sin\theta}\pdv{\theta}(\sin\theta \var{v}_\theta) \, ,
\end{equation*}
while all other terms are linear in the $\theta$-dependence.
Therefore, separating the $r$- from the $\theta$-dependence in those equations is possible by requiring
\begin{equation}
    X_l(\theta) \propto \frac{1}{\sin\theta} \pdv{\theta}(\sin\theta X'_l(\theta)) \, .
    \label{eq:X_l_requiremetn}
\end{equation}
Similarly, the only $\theta$-derivative in \cref{eq:linearized_momentum_conservation} is
\begin{equation*}
    \hat{\theta} \cdot \vec{\nabla}\var{p} = \frac{1}{r} \pdv{\var{p}}{\theta} \, ,
\end{equation*}
while the only other components in the $\hat{\theta}$-direction are linear in $\var{v}_\theta$.
This leads us to require
\begin{equation}
    X'_l(\theta) = \pdv{X_l}{\theta}\, .
    \label{eq:Y_l_requiremetn}
\end{equation}
The last two \cref{eq:linearized_effective_heat_flux,eq:linearized_electron_energy_loss} by design do not contain $\theta$-derivatives or $\var{v}_\theta$.
Combining those two requirements of \cref{eq:X_l_requiremetn,eq:Y_l_requiremetn} and choosing the proportionality constant in \cref{eq:X_l_requiremetn} to be $-1/(l(l+1))$ leads to the defining differential equation for associated Legendre polynomials $X_l(\theta) = P^0_l(\cos\theta)$.
These are precisely the $m=0$ spherical harmonics with no $\varphi$-dependence.
Thus, the $\theta$-dependence of each mode $l$ is known and leads to a set of equations for the $r$-dependent coefficients of each mode $l$.
Note that the $l=0$ mode is independent of $\theta$, which motivates our choice of notation $y_0$ for the isotropic NGS model quantities.

After a rather lengthy derivation, the set of equations determining the $r$-dependence of the asymmetric neutral ablation cloud quantities is
\begin{align}
    &\rho_l = m \left( \frac{p_l}{T_0} - \frac{p_0}{T_0^2} T_l \right) \, , \label{eq:physical_perturbation_ideal_gas_law} \\
    &\pdv{\rho_0}{r}v_{l,r} + \rho_0 \left[ \frac{1}{r^2} \pdv{r}(r^2 v_{l,r}) - \frac{l(l+1)}{r} v_{l,\theta} \right] + v_0 \pdv{\rho_l}{r} + \frac{1}{r^2}\pdv{r}(r^2 v_0) \rho_l = 0 \, , \label{eq:physical_perturbation_mass_conservation} \\
    &\rho_0 v_0 \pdv{v_{l,r}}{r} + \rho_0 \pdv{v_0}{r} v_{l,r} + v_0 \pdv{v_0}{r} \rho_l = - \pdv{p_l}{r} \, , \label{eq:physical_perturbation_r_momentum_conservation} \\
    &\rho_0 v_0 \pdv{v_{l,\theta}}{r} + \rho_0 \frac{v_0}{r} v_{l,\theta} = - \frac{p_l}{r} \, , \label{eq:physical_perturbation_theta_momentum_conservation} \\
    &\left[ v_{l,r} \pdv{r} + \frac{1}{r^2}\pdv{r}(r^2 v_{l,r}) - \frac{l(l+1)}{r} v_{l,\theta} \right] \left( \frac{1}{2} \rho v_0^2 + \frac{\gamma}{\gamma - 1} p_0 \right) \nonumber \\
    &+ \left[ v_0 \pdv{r} + \frac{1}{r^2}\pdv{r}(r^2 v_0) \right] \left( \frac{1}{2} \rho_l v_0^2 + \rho_0 v_0 v_{l,r} + \frac{\gamma}{\gamma-1}p_l \right) = \mu \pdv{q_l}{r} \, , \label{eq:physical_perturbation_energy_conservation}
\end{align}
while the electron dynamics in the neutral gas is described by
\begin{align}
    &\pdv{E_l}{r} = 2 \frac{\rho_l}{m} L(E_0) + 2 \frac{\rho_0}{m} \eval{\pdv{L}{E}}_{E_0} E_l \, , \label{eq:physical_perturbation_electron_energy_loss} \\
    &\pdv{q_l}{r} = \frac{\rho_l}{m} q_0 \Lambda(E_0) + \frac{\rho_0}{m} q_l \Lambda(E_0) + \frac{\rho_0}{m} q_0 \eval{\pdv{\Lambda}{E}}_{E_0} E_l \, . \label{eq:physical_perturbation_effective_heat_flux}
\end{align}

Apart from enabling the separation of variables, expansion in terms of Legendre polynomials is convenient for calculating the pellet rocket force.
As shown in \cref{sec:pellet_surface}, the pellet rocket force only depends on the projection of $\var{\rho}$, $\var{v_r}$, $\var{p}$ and $\int \var{v_\theta} \dd{\theta}$ onto $P^0_1=\cos\theta$ at the pellet surface.
Since the ablation dynamics of all $l$-modes are independent here, only the $l=1$ mode is needed in our model.
Without loss of generality, it suffices to see the perturbation quantities as
\begin{align}
\begin{aligned}
    \var{\rho} &= \rho_1(r) \cos\theta \, , &
    \var{p} &= p_1(r) \cos\theta \, , \\
    \var{T} &= T_1(r) \cos\theta \, , &
    \var{\vec{v}} &= \hat{r} v_{1,r}(r) \cos\theta - \hat{\theta} v_{1,\theta}(r) \sin\theta \, , \\
    \var{q} &= q_1(r) \cos\theta \, , &
    \var{E} &= E_1(r) \cos\theta \, . &
\end{aligned}
\end{align}
Physically, this means that for positive coefficients $y_1(r)$, the full quantity $y(\vec{r})$ has slightly increased values in the positive $\hat{z}$ direction and slightly decreased values in the negative $\hat{z}$ direction, while at $z=0$ the isotropic NGS model is unperturbed.
Since only the angular velocity was expanded in the different basis $X'_l(\theta)$, the perturbation $\var{v}_\theta$ is now proportional to $-\sin\theta$.
In other words, the angular flow is going from positive $z$ to negative $z$ in the case of negative $v_{1,\theta}(r)$, and vice versa.
However, the angular flow is always zero at $\theta=0, \pi$ and the strongest at $\theta=\pi/2$.

The boundary conditions for the asymmetric NGS model quantities are derived from the general NGS boundary conditions in \cref{eq:full_boundary_conditions} to be 
\begin{gather}
\begin{gathered}
    q_1(r_p) = 0, \quad
    T_1(r_p) = 0, \quad
    v_{1,\theta}(r_p) = 0, \quad
    p_1(r \rightarrow \infty) = 0, \\
    q_1(r \rightarrow \infty) = \frac{3}{2} \int_0^\pi q_\text{bc}(\theta) \cos\theta \sin\theta \dd{\theta}, \\
    E_1(r \rightarrow \infty) = \frac{3}{2} \int_0^\pi E_\text{bc}(\theta) \cos\theta \sin\theta \dd{\theta}.
    \label{eq:perturbation_physical_bc}
\end{gathered}
\end{gather}
Since introducing $v_{1,\theta}$ adds the need for one additional boundary condition, we assume $v_{1,\theta}(r_p) = 0$.
The incoming electron heat flux and energy are projected onto the $\cos\theta$ mode, without assuming a particular $\theta$-dependence.

Solving for the perturbative ablation dynamics is again not possible analytically, and the equations need to be prepared for numerical analysis.
A convenient way to quantify the degree of asymmetry in the external heat source is by defining
\begin{equation}
    q_\text{rel} = \frac{q_{1}(\infty)}{q_{0}(\infty)} \quad \text{and} \quad 
    E_\text{rel} = \frac{E_{1}(\infty)}{E_{0}(\infty)} \, .
    \label{eq:perturbation_normalized_bc}
\end{equation}
These two quantities are the only parameters needed in our asymmetric NGS model, on top of the isotropic NGS model parameters.  
The perturbation assumption requires $|q_\text{rel}| \ll 1$ and $|E_\text{rel}| \ll 1$.
Furthermore, the signs determine which side of the ablation cloud receives a higher heat flux $q$ or higher average electron energy $E$.
Later on, it will be seen that in the case of drift induced plasmoid shielding asymmetry, as explained in \cref{sec:plasmoid_shielding,ssec:plasmoid_cloud}, the average energy is typically higher on the side of lower heat flux\footnote{Remember that the number of incoming electrons changes the heat flux but not the average energy.}, i.e. $E_\text{rel}/q_\text{rel} < 0$.

The previously presented normalization of the isotropic NGS model to the sonic radius allowed the complexity of the problem to be reduced by use of additional physics knowledge in the form of the speed of sound definition in \cref{eq:speed_of_sound}.
Such additional knowledge is not available for our asymmetric NGS model.
Nevertheless, it turns out to be convenient to also normalize the asymmetric NGS model quantities $y_1$ to the sonic radius, as
\begin{gather}
\begin{gathered}
    \widetilde{\rho}_1=\frac{\rho_1}{\rho_\star q_\text{rel}}, \quad 
    \widetilde{p}_1 = \frac{p_1}{p_\star q_\text{rel}}, \quad 
    \widetilde{T}_1 = \frac{T_1}{T_\star q_\text{rel}}, \quad 
    \widetilde{v}_{1,r}=\frac{v_{1,r}}{v_\star q_\text{rel}}, \quad 
    \widetilde{v}_{1,\theta}=\frac{v_{1,\theta}}{v_\star q_\text{rel}}, \\
    \widetilde{q}_1=\frac{q_1}{q_\star q_\text{rel}}, \quad 
    \widetilde{E}_1=\frac{E_1}{E_\star  q_\text{rel}}, \quad 
\end{gathered}
\label{eq:1st_order_normalization}
\end{gather}
by including the degree of heat flux asymmetry $q_\text{rel}$ in the definition.
Together with normalizing the isotropic NGS model quantities $y_0$ as previously defined in \cref{eq:ngs_normalization}, this simplifies the perturbative heat source boundary conditions to
\begin{equation}
    \widetilde{q}_{1}(\infty) = \widetilde{q}_{0}(\infty) \quad \text{and} \quad 
    \widetilde{E}_{1}(\infty) = \widetilde{E}_{0}(\infty) \cdot \frac{E_\text{rel}}{q_\text{rel}}.
\end{equation}
Again, the tilde notation is dropped from now on and all quantities can be considered normalized, if not stated otherwise.
Note that, in contrast to the normalized isotropic NGS model, the normalized perturbation quantities at the sonic radius are unknown.

The chosen normalization leaves the system of \cref{eq:physical_perturbation_ideal_gas_law,eq:physical_perturbation_mass_conservation,eq:physical_perturbation_r_momentum_conservation,eq:physical_perturbation_theta_momentum_conservation,eq:physical_perturbation_energy_conservation,eq:physical_perturbation_effective_heat_flux,eq:physical_perturbation_electron_energy_loss} nearly unchanged.
In particular, the linearity in the perturbation quantities leads to a cancellation of all $q_\text{rel}$ factors.
The only changes in terms of new factors on the right side of the equations are 
\begin{equation*} 
% https://tex.stackexchange.com/questions/103508/adjusting-space-between-array-rows-and-columns
{\arraycolsep=1.4pt\def\arraystretch{1.3}
\begin{array}{cll}
    \frac{1}{m} && \text{in \cref{eq:physical_perturbation_ideal_gas_law},} \\
    \frac{1}{\gamma} &\quad\quad& \text{in \cref{eq:physical_perturbation_r_momentum_conservation,eq:physical_perturbation_theta_momentum_conservation},} \\
    \frac{\gamma}{\gamma-1} \frac{2}{\lambda_\star \mu} && \text{in \cref{eq:physical_perturbation_energy_conservation} and} \\
    m \lambda_\star &&\text{in \cref{eq:physical_perturbation_electron_energy_loss,eq:physical_perturbation_effective_heat_flux}.}
\end{array}
}
\end{equation*}
Therefore, apart from the normalized isotropic NGS model parameters $\gamma$, $E_\star$ and $\lambda_\star$, the only new parameter needed to determine the normalized asymmetric NGS model is $E_\text{rel}/q_\text{rel}$.

Since the derived system of equations is linear in both the perturbation quantities $y_1$ and their derivatives $\partial y_1/\partial r$, it is convenient to write the differential equations in terms of $6 \times 6$, $y_0$-dependent, matrices $A(r)$ and $B(r)$, such that
\begin{equation}
    A \pdv{\vec{y}_1}{r} = B \vec{y}_1 \quad \text{with} \quad \vec{y}_1 = (p_1, T_1, v_{1,r}, v_{1,\theta}, q_1, E_1)^T \, ,
\end{equation}
where $\rho_1$ was replaced through the linearized and normalized ideal gas law (\cref{eq:physical_perturbation_ideal_gas_law} without $m$).
Symbolic computation allows us to find an analytic expression for $C = A^{-1}B$ so that
\begin{equation}
    \pdv{\vec{y}_1}{r} = C \vec{y}_1 \, .
    \label{eq:perturbation_normalized_system}
\end{equation}
Since this expression is large and those details are not necessary here, it is only given in \cref{sec:appendix_sympy_expressions}.
However, an important feature of $C$ is that it contains the apparent singularity $1/(T_0 - v_0^2)$ in front of the first three rows.
Requiring again that $\partial \vec{y}_1/\partial r$ is finite at the sonic radius leads to a matrix $\eval{(T_0 - v_0^2)C}_{r=1}$ of which the first three rows have rank 1.
Thus, this requirement reduces the number of unknowns at the sonic radius by one through the obtained equation
\begin{equation}
v_{1,\theta}(r=1) = \left[ \left(1- \frac{\chi_\star}{2}\right) v_{1,r} + \left(1 + \frac{\chi_\star}{4}\right)T_1 - q_1 - \frac{\Psi_\star}{2 \lambda_\star L(E=1)} E_1 \right]_{r=1} \, ,
\label{eq:perturbation_sonic_relation}
\end{equation}
where $\chi_\star$ and $\Psi_\star$ are shorthands defined by \cref{eq:chi_star}.
Equivalently to the normalized NGS model, the derivatives at $r=1$ can then be evaluated using L'Hôpital's rule.
The corresponding large expression for $C_\star = \eval{C}_{r=1}$ is given in \cref{sec:appendix_sympy_expressions}.
Having solved this apparent singularity, everything is provided to start finding numerical solutions.

%%%%%%%%%%%%%%%%%%%%%%%%%%%%%%%%%%%%%%%%%%%%%%%%%%%%%%%%%%%%%%%%%%%%%%%%%%%%%%%%%%%%%%%%%%%%%%%%%%%%%%
\subsection{Numerical solution of the asymmetric NGS model}

The procedure to find the numerical solution of the normalized perturbation quantities is chosen to be similar to the procedure used for the isotropic dynamics.
Even though the chosen normalization in \cref{eq:1st_order_normalization} does not introduce unknown parameters into the system of equations, it also does not bring knowledge about starting values for numerical integration.
It is difficult to start from the normalized pellet radius $r_\text{p}$ because, there, the derivatives diverge.
Starting from $r\rightarrow\infty$ is difficult to handle numerically.
Having found a way to eliminate one unknown at $r=1$ through \cref{eq:perturbation_sonic_relation}, this poses as a convenient starting point for solving the perturbation dynamics.
The differential \cref{eq:perturbation_normalized_system} is solved for both increasing and decreasing $r$, and the guesses for $\vec{y}_1(r=1)$ are adjusted so that the boundary conditions in \cref{eq:perturbation_physical_bc,eq:perturbation_normalized_bc} are fulfilled.
Note, that reducing the unknowns at $r=1$ also removed the need for one of the boundary conditions, and it turns out that $p_1(\infty)=0$ is always fulfilled.
The numerical methods used here are the same as those described in \cref{ssec:isotropic_numerics}.
%Those methods take variable steps in $r$, which means that the numerical solution to the normalized NGS model needs to be interpolated to those $r$ positions to save computation time.
%Additionally, finite numerical precision allows the boundary conditions to be fulfilled only close the $r_\text{p}$ found in the isotropic solution.
After adjustment of numerical tolerances and testing different optimization schemes, a robust algorithm was developed to reliably solve the asymmetric perturbation dynamics with high precision.

\begin{figure}
    \centering
    \includegraphics{figure/ode1_02_Estar_3.000e+04_gamma_7_5_E_rel_over_q_rel_-0.50.pdf}
    \caption{Radial dependence of a numerical example solution to the normalized perturbative ablation dynamics. The corresponding isotropic solution is shown in \cref{fig:example_0th_order_solution} with the parameters $\gamma = 7/5$ and $E_\star(E_\text{bc}) = \qty{30}{\kilo\eV}$. The heat source asymmetry is characterized here by $E_\text{rel}/q_\text{rel} = -0.5$.}
    \label{fig:example_1st_order_solution}
\end{figure}

An example solution is shown in \cref{fig:example_1st_order_solution}.
This is the corresponding perturbation to the isotropic solution shown in \cref{fig:example_0th_order_solution} ($\gamma = 7/5$, $E_\star = \qty{30}{\keV}$), with a heating asymmetry ratio $E_\text{rel}/q_\text{rel} = -0.5$.
While this plot shows the radial dependence of the asymmetry perturbation, it might be easier to interpret the asymmetry through visualization of the full radial and angular dependence.
Therefore, \cref{fig:full_solution_visuals} shows half of the 2D spatial variation of both the isotropic dynamics (on the left) and the asymmetric perturbation (on the right).
The pellet is visualized by the grey circle in the middle, and the dashed line around it indicates the sonic radius.
In reality, the neutral ablation cloud boundary is much further away than visualized, but it can be considered a zoomed-in view.
Scalar quantities are presented through a variation in colour, where darker values mean higher absolute values. 
For the perturbation quantities, red means an increase and blue means a decrease compared to the isotropic NGS model.
Note that this visualization shows all quantities as their normalized version, while the physical perturbation quantities are much smaller than the isotropic quantities.
While both the isotropic pressure $p_0$ and the isotropic density $\rho_0$ are largest close to the pellet surface, the pressure asymmetry $p_1$ shows opposite behaviour of the density asymmetry $\rho_1$.
The quantities $T$, $q$, $E$ show clearly that the major part of heating the neutral gas happens close to the pellet and the asymmetry in temperature $T_1$ follows the asymmetry in heat flux $q_1$.
The flow velocity is presented as a vector field, showing a radial outflow in the isotropic dynamics and a flow from the upper side to the lower side is evident in the perturbation.

\begin{figure}
    \centering
    \includegraphics{figure/full_solution_visuals.pdf}
    \caption{Full spatial dependence of the chosen numerical example solution for the perturbative ablation dynamics. The left sides show the NGS model solution, as in \cref{fig:example_0th_order_solution}. The right sides show the $\cos\theta$ perturbation solutions, with the radial dependence as shown in \cref{fig:example_1st_order_solution}. For illustrative purposes, $v_{1,\theta}$ is scaled up by a factor of 4. The dashed circle marks the sonic radius.}
    \label{fig:full_solution_visuals}
\end{figure}

Overall, it must be noted that \cref{fig:full_solution_visuals} is only one specific solution.
While the dynamics changes only slightly with $\gamma$ and $E_\text{bc}$, the asymmetry parameter $E_\text{rel}/q_\text{rel}$ can change the dynamics on a qualitative level.
The solution shown in \cref{fig:full_solution_visuals} agrees well with our expectation of how an asymmetric heat source changes the ablation dynamics.
However, the two further solutions shown in \cref{fig:ode1_interesting} give surprising insights into ablation physics.

\begin{figure}
    \captionsetup[subfigure]{justification=centering}
    \centering
    \begin{subfigure}[b]{0.49\textwidth}
        \centering
        \includegraphics[width=\textwidth]{figure/ode1_05_Estar_3.000e+04_gamma_7_5_E_rel_over_q_rel_+1.00.pdf}
        \caption{}
        \label{fig:ode1_interesting_+1.0}
    \end{subfigure}%
    \hfill
    \begin{subfigure}[b]{0.49\textwidth}
        \centering
        \includegraphics[width=\textwidth]{figure/ode1_00_Estar_3.000e+04_gamma_7_5_E_rel_over_q_rel_-1.50.pdf}
        \caption{}
        \label{fig:ode1_interesting_-1.5}
    \end{subfigure}
    \caption{Further numerical solutions of the perturbative ablation dynamics. The corresponding isotropic quantities are shown in \cref{fig:example_0th_order_solution}. The left figure (with $E_\text{rel}/q_\text{rel} = +1$) shows a temperature increase on the low heat flux side. The right figure (with $E_\text{rel}/q_\text{rel} = -1.5$) shows a pressure increase on the low heat flux side, leading to a pellet rocket force towards the high heat flux side, the grey region indicates the pellet.}
    \label{fig:ode1_interesting}
\end{figure}

With an increased average electron energy on the same side as the increased heat flux, i.e. $E_\text{rel}/q_\text{rel} = +1$, as shown in \cref{fig:ode1_interesting_+1.0}, the ablation cloud temperature and flow velocity can be larger on the opposite side.
At first, this might seem unphysical.
However, it can be understood, as the increased density leads to an increase in required energy to raise the temperature the same amount. 
It also leads to a higher pressure asymmetry at the pellet surface compared to the $E_\text{rel}/q_\text{rel} = -0.5$ dynamics.

Another qualitatively different solution, which is more relevant for modelling the pellet rocket effect, is shown in \cref{fig:ode1_interesting_-1.5} with $E_\text{rel}/q_\text{rel} = -1.5$.
This means the asymmetry in average incident electron energy is opposite to the asymmetry in incident heat flux.
This was, to a lesser extent, also the case in the first solution, shown in \cref{fig:example_1st_order_solution} with $E_\text{rel}/q_\text{rel} = -0.5$.
However, here, the pressure asymmetry at the pellet radius becomes negative.
This would result in a negative pellet rocket force, i.e. towards the high external heat flux side.
Whether this reversed rocket force is physically accurate cannot be evaluated in this thesis.
It might be an artifact from the approximations of mono-energetic electrons losing heat radially.

\begin{figure}
    \centering
    \includegraphics{figure/P1_at_r_p.pdf}
    \caption{Linear dependence of the pressure asymmetry at the pellet surface on the parameter $E_\text{rel}/q_\text{rel}$. The vertical dashed line at $-1.17$ shows the change in polarity. The colour of the lines denote $\gamma$, while the brightness denotes $E_\text{bc}$, with the exact values shown by the black lines on the colour bars.}
    \label{fig:P1_at_r_p}
\end{figure}

The three example solutions illustrate that the pellet rocket force, which stems mainly from $p_1(r_\text{p})$, is highly dependent on $E_\text{rel}/q_\text{rel}$.
To quantify this phenomenon, a parameter scan is performed over $\gamma$, $E_\text{bc}$ and $E_\text{rel}/q_\text{rel}$.
The corresponding optimized values at the sonic radius are shown in \cref{sec:appendix_1st_order_scan} for completeness.
The resulting $p_1(r_\text{p})$ are shown in \cref{fig:P1_at_r_p}.
Marked with a vertical dashed line is the value $E_\text{rel}/q_\text{rel} \approx -1.17$, where the pressure asymmetry, and consequently the pellet rocket force, changes polarity.
Prominently, the dependence on $E_\text{rel}/q_\text{rel}$ is found to be purely linear, with the slope and position only depending weakly on $\gamma$ and $E_\text{bc}$.
Linear regression has shown relative fit errors of less than $1\%$ in the tested orders of magnitude of $|E_\text{rel}/q_\text{rel}|$ from $\num{1e-2}$ to $\num{1e6}$.
This linear dependence is convenient, since it allows for a simple semi-analytical formula connecting the normalized pressure asymmetry to the degree of asymmetry of the external heating source
\begin{equation}
    p_1(r_\text{p}) = a \cdot \left(\frac{E_\text{rel}}{q_\text{rel}} - b \right) \, ,
    \label{eq:p1_at_r_p}
\end{equation}
where $a$ is the slope and $b$ is the root (where $p_1(r_\text{p})$ changes polarity).
%Note that $p_1$ is normalized as chosen in \cref{eq:1st_order_normalization}.
The fit parameters are found to be $a \approx 2.0$ to $2.9$ and $b \approx -1.21$ to $-1.17$.
The explicit dependence on $\gamma$ and $E_\text{bc}$ values is shown in \cref{fig:P1_fit_parameters}.
Corresponding scaling laws are provided in \cref{sec:appendix_scaling_laws}.

\begin{figure}
    \centering
    \includegraphics{figure/P1_fit_parameters.pdf}
    \caption{Linear regression parameters for $p_1(r_\text{p})$ dependence on $E_\text{rel}/q_\text{rel}$. Shown is the dependence on $\gamma$ and $E_\text{bc}$. $a$ is the slope and $b$ is the value at which $p_1(r_\text{p})$ changes sign.}
    \label{fig:P1_fit_parameters}
\end{figure}

There is one special case that has been neglected in our analysis.
If there is no asymmetry in the external heat flux, i.e. $q_\text{rel} = 0$, but the average energy of incident electrons varies over the $\hat{z}$-axis, i.e. $E_\text{rel} \neq 0$, the chosen normalization in \cref{eq:1st_order_normalization} is not valid.
In the same way that the factor $q_\text{rel}$ cancelled out in the normalized system of equations, the expressions remain unchanged when normalizing to $E_\text{rel}$ instead.
Only the normalized boundary conditions change to
\begin{gather}
    \frac{q_1(r \rightarrow \infty)}{q_\star E_\text{rel}} = 0 \quad \text{and} \quad 
    \frac{E_1(r \rightarrow \infty)}{E_\star E_\text{rel}} = \widetilde{E}_0(r\rightarrow\infty) \, .
\end{gather}
Solving the dynamics numerically in the same way as before yields similar results as for $q_\text{rel} \neq 0$. 
It is found that
\begin{equation}
    \frac{p_1(r_p)|_{q_\text{rel}=0}}{p_\star E_\text{rel}} = a \, ,
\end{equation}
with the same values of $a$ as shown in \cref{fig:P1_fit_parameters}.

Finally, it was mentioned in \cref{sec:pellet_surface} that the pellet rocket force depends mainly on $p_1(r_\text{p})$.
This statement will now be supported by evidence, given our choice of approximations and boundary conditions.
Inserting the normalized perturbation quantities into the formula for the pellet rocket force in \cref{eq:rocket_force_full} and using the definition of the sound speed $(\rho_\star v_\star^2 = \gamma p_\star)$ gives
\begin{equation}
    F = \frac{4 \pi r_\text{p}^2}{3} p_\star q_\text{rel} \underbrace{\left( \gamma \widetilde{v}_0^2 \widetilde{\rho}_1 + 2 \gamma \widetilde{\rho}_0 \widetilde{v}_0 (\widetilde{v}_{1,r} - \widetilde{v}_{1,\theta}) + \widetilde{p}_1 \right)_{r=r_\text{p}}}_{\widetilde{F}} \, .
    \label{eq:normalized_force}
\end{equation}
The chosen boundary condition $v_{1,\theta}(r_\text{p}) = 0$ together with the fact that $r^2 \rho_0 v_0^2 = \textit{const.}$ results in the corresponding term in the force being zero.
Asymptotic analysis on the other terms is non-trivial, since $v_0(r_\text{p}) \rightarrow 0$ and $v_{1,r}(r_\text{p}) \rightarrow 0$ but $\rho_0(r_\text{p})\rightarrow\infty$ and $\rho_1(r_\text{p})\rightarrow\infty$.
Therefore, this analysis has to be done numerically and \cref{fig:force_relative_error} shows the relative contribution of the non-pressure terms in $\widetilde{F}$.
This shows clearly that the pressure asymmetry is the only quantity that is important for the pellet rocket force, whereas adding other terms would change the result only in the third or fourth significant figure.
Note that the noise in \cref{fig:force_relative_error} is a result of accumulated numerical errors when calculating the numerical solutions and a finite precision when optimizing the boundary conditions.
However, the achieved numerical precision is considered sufficient, given the approximations of the NGS model.

\begin{figure}
    \centering
    \includegraphics{figure/force_relative_error.pdf}
    \caption{Relative contribution of non-pressure asymmetry ($\widetilde{p}_1(r_\text{p})$) quantities to the dimensionless terms $\widetilde{F}$ in the pellet rocket force.}
    \label{fig:force_relative_error}
\end{figure}

Overall, it has been shown here that it is possible to find numerical solutions to the perturbative dynamics in the neutral ablation cloud.
A semi-analytical model for the pellet rocket force is derived as
\begin{equation}
\boxed{%
    F = \frac{4 \pi r_\text{p}^2}{3} p_\star \left( a E_\text{rel} - a b q_\text{rel} \right) \, ,
}
\label{eq:final_pellet_rocket_force}
\end{equation}
by combining \cref{eq:p1_at_r_p,eq:normalized_force}, with $p_\star$ given through \cref{eq:prefactor_p_star} and $a$ and $b$ given in \cref{fig:P1_fit_parameters}.
Therefore, only the shielding of the neutral ablation cloud from the background plasma electrons is still needed to model the pellet rocket effect.
How the plasmoid shielding leads to estimates for the parameters $E_\text{bc}$, $q_\text{bc}$, $E_\text{rel}$ and $q_\text{rel}$ is the topic of the next section.



