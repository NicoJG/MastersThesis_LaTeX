\subsection{Asymmetric perturbation model}
\label{ssec:asymmetric_perturbation_model}

Provided a procedure of calculating the full ablation dynamics under the assumption of spherical symmetry, it is now possible to describe the asymmetry in the ablation dynamics as a perturbative model.
The baseline model is the previously described NGS model and will be denoted from now on with the index 0.
Assuming the real ablation process is described well by the NGS model quantities $y_0(r)$ with a small correction $\var{y}(r,\theta)$, which depends on the asymmetry along one axis ($z$-axis in \cref{fig:neutral_cloud_heating}), the full physical quantities can be modelled as
\begin{equation}
\begin{aligned}
    \rho &= \rho_0 + \var\rho\, , & 
    p &= p_0 + \var{p}\, , \\
    T &= T_0 + \var{T}\, , & 
    \vec{v} &= \hat{r} (v_0 + \var{v}_r) + \hat{\theta} (0+\var{v}_\theta)\, , \\
    q &= q_0 + \var{q}\, , & 
    E &= E_0 + \var{E}\, .
\end{aligned}
\end{equation}
The spherical coordinate system is chosen as depicted in \cref{fig:neutral_cloud_heating,fig:pellet_surface_force}, while neglecting the $\varphi$-dependence.

Treating the system as a quasi-steady-state ideal gas is retained in our model in the same way the NGS model is treated.
Therefore, the \cref{eq:ideal_gas_law,eq:full_mass_conservation,eq:full_momentum_conservation,eq:full_energy_conservation,eq:heat_flux_approximation,eq:full_effective_heat_flux,eq:full_electron_energy_loss} form the basis of our asymmetric perturbative model.
This includes the approximation of a purely radial flux of electrons, as represented by \cref{eq:heat_flux_approximation}.
Linearizing the set of equations in the perturbation quantities $\var{y}$, combined with the fact that the NGS quantities $y_0$ satisfy those equations themselves, yields
\begin{flalign}
    &\var{\rho} = m \frac{\var{p}}{T_0} - m \frac{p_0}{T_0^2} \var{T} 
    &\text{(ideal gas law),}& \label{eq:linearized_ideal_gas_law} \\
    &\vec{\nabla} \cdot (\var{\rho} \vec{v}_0 + \rho_0 \var{\vec{v}}) = 0 
    &\text{(mass conservation),}&\label{eq:linearized_mass_conservation} \\
    &\rho_0 (\vec{v}_0 \cdot \vec{\nabla})\var{\vec{v}} + \rho_0 (\var{\vec{v}} \cdot \vec{\nabla})\vec{v}_0 + \var{\rho} (\vec{v}_0 \cdot \vec{\nabla})\var{\vec{v}} = - \vec{\nabla}\var{p} 
    \span\span\nonumber\\&&\text{(momentum conservation),}&\label{eq:linearized_momentum_conservation} \\
    &\vec{\nabla} \cdot \left[ \left( \frac{1}{2}\rho_0 v_0^2 + \frac{\gamma}{\gamma - 1}p_0 \right)\var{\vec{v}} + \left( \frac{1}{2}\var{\rho} v_0^2 + \rho_0 (\vec{v}_0\cdot \var{\vec{v}}) + \frac{\gamma}{\gamma - 1}\var{p} \right)\vec{v}_0 \right] = \mu \pdv{\var{q}}{r} 
    \span\span\nonumber\\&&\text{(energy conservation),}& \label{eq:linearized_energy_conservation} \\
    &\pdv{\var{E}}{r} = 3\frac{\var{\rho}}{m}L(E_0) + \frac{1}{2} \frac{\rho_0}{m} \eval{\pdv{L}{E}}_{E_0}  \!\!\!\! \var{E} 
    &\text{(electron energy loss),}&\label{eq:linearized_effective_heat_flux} \\
    &\pdv{\var{q}}{r} = \frac{\var{\rho}}{m} q_0 \Lambda(E_0) + \frac{\rho_0}{m} \var{q} \Lambda(E_0) + \frac{\rho_0}{m} q_0 \eval{\pdv{\Lambda}{E}}_{E_0} \!\!\!\! \var{E} \quad
    \span\text{(effective heat flux).}& \label{eq:linearized_electron_energy_loss}
\end{flalign}

Now we want to find a set of equations for the $r$-dependence and a separate set of equations for the $\theta$-dependence.
Fortunately, this is possible without further approximations by expanding the perturbation in terms of general fully orthogonal basis functions $\{X_l(\theta)\}$ in the form
\begin{align}
\begin{aligned}
    \var{\rho} &= \sum_l \rho_l(r) X_l(\theta) \, , &
    \var{p} &= \sum_l p_l(r) X_l(\theta) \, , \\
    \var{T} &= \sum_l T_l(r) X_l(\theta) \, , &
    \var{v}_r &= \sum_l v_{l,r}(r) X_l(\theta) \, , \\
    \var{q} &= \sum_l q_l(r) X_l(\theta) \, , &
    \var{E} &= \sum_l E_l(r) X_l(\theta) \, , &
\end{aligned}
\end{align}
while expanding only $\var{v}_\theta$ in terms of a different general basis $\{Y_l(\theta)\}$ as
\begin{equation}
    \var{v}_\theta = \sum_l v_{l,\theta}(r) Y_l(\theta) \, .
\end{equation}

% the set of \cref{eq:linearized_ideal_gas_law,eq:linearized_mass_conservation,eq:linearized_momentum_conservation,eq:linearized_energy_conservation,eq:linearized_effective_heat_flux,eq:linearized_electron_energy_loss}
Taking the derivatives in terms of spherical coordinates, the only $\theta$-derivative appearing in \cref{eq:linearized_mass_conservation,eq:linearized_energy_conservation} is
\begin{equation*}
    \vec{\nabla}\cdot(\var{v}_\theta \hat{\theta}) = \frac{1}{r \sin\theta}\pdv{\theta}(\sin\theta \var{v}_\theta) \, ,
\end{equation*}
while all other terms are linear in the $\theta$-dependence.
Therefore, separating the $r$- from the $\theta$-dependence in those equations is possible by requiring
\begin{equation}
    X_l(\theta) \propto \frac{1}{\sin\theta} \pdv{\theta}(\sin\theta Y_l(\theta)) \, .
    \label{eq:X_l_requiremetn}
\end{equation}
Similarly, the only $\theta$-derivative in \cref{eq:linearized_momentum_conservation} is
\begin{equation*}
    \hat{\theta} \cdot \vec{\nabla}\var{p} = \frac{1}{r} \pdv{\var{p}}{\theta} \, ,
\end{equation*}
while the only other components in the $\hat{\theta}$-direction are linear in $\var{v}_\theta$.
This leads us to require
\begin{equation}
    Y_l(\theta) = \pdv{X_l}{\theta}\, .
    \label{eq:Y_l_requiremetn}
\end{equation}
The last two \cref{eq:linearized_effective_heat_flux,eq:linearized_electron_energy_loss} by design do not contain $\theta$-derivatives or $\var{v}_\theta$.
Combining those two requirements and choosing the proportionality constant in \cref{eq:X_l_requiremetn} to be $-1/(l(l+1))$ leads to the defining differential equation for associated Legendre polynomials $X_l(\theta) = P^0_l(\cos\theta)$.
Thus, the $\theta$-dependence of each mode $l$ is known and leads to a set of equations for the $r$-dependent coefficients of each mode $l$.
Note that the $l=0$ mode is independent of $\theta$, which motivates our choice of notation $y_0$ for the NGS model quantities.

After a rather lengthy derivation, the set of equations describing the $r$-dependence of the asymmetric neutral ablation cloud quantities is
\begin{align}
    &\rho_l = m \left( \frac{p_l}{T_0} - \frac{p_0}{T_0^2} T_l \right) \, , \\
    &\pdv{\rho_0}{r}v_{l,r} + \rho_0 \left[ \frac{1}{r^2} \pdv{r}(r^2 v_{l,r}) - \frac{l(l+1)}{r} v_{l,\theta} \right] + v_0 \pdv{\rho_l}{r} + \frac{1}{r^2}\pdv{r}(r^2 v_0) \rho_l = 0 \, , \\
    &\rho_0 v_0 \pdv{v_{l,r}}{r} + \rho_0 \pdv{v_0}{r} v_{l,r} + v_0 \pdv{v_0}{r} \rho_l = - \pdv{p_l}{r} \, , \\
    &\rho_0 v_0 \pdv{v_{l,\theta}}{r} + \rho_0 \frac{v_0}{r} v_{l,\theta} = - \frac{p_l}{r} \, , \\
    &\left[ v_{l,r} \pdv{r} + \frac{1}{r^2}\pdv{r}(r^2 v_{l,r}) - \frac{l(l+1)}{r} v_{l,\theta} \right] \left( \frac{1}{2} \rho v_0^2 + \frac{\gamma}{\gamma - 1} p_0 \right) \nonumber \\
    &+ \left[ v_0 \pdv{r} + \frac{1}{r^2}\pdv{r}(r^2 v_0) \right] \left( \frac{1}{2} \rho_l v_0^2 + \rho_0 v_0 v_{l,r} + \frac{\gamma}{\gamma-1}p_l \right) = \mu \pdv{q_l}{r} \, ,
\end{align}
while the electron dynamics in the neutral gas are described by
\begin{align}
    &\pdv{E_l}{r} = 2 \frac{\rho_l}{m} L(E_0) + 2 \frac{\rho_0}{m} \eval{\pdv{L}{E}}_{E_0} E_l \, , \\
    &\pdv{q_l}{r} = \frac{\rho_l}{m} q_0 \Lambda(E_0) + \rho_0 q_l \Lambda(E_0) + \rho_0 q_0 \eval{\pdv{\Lambda}{E}}_{E_0} E_l \, .
\end{align}

Apart from enabling the separation of variables, expansion in terms of Legendre polynomials is convenient for calculating the pellet rocket force.
As shown in \cref{sec:pellet_surface}, the pellet rocket force only depends on the projection of $\var{\rho}$, $\var{v_r}$, $\var{p}$ and $\int \var{v_\theta} \dd{\theta}$ onto $P^0_1=\cos\theta$ at the pellet surface.
Since the ablation dynamics of all $l$-modes are independent here, only the $l=1$ mode is needed in our model.
Without loss of generality, it suffices to see the perturbation quantities as
\begin{align}
\begin{aligned}
    \var{\rho} &= \rho_1(r) \cos\theta \, , &
    \var{p} &= p_1(r) \cos\theta \, , \\
    \var{T} &= T_1(r) \cos\theta \, , &
    \var{\vec{v}} &= \hat{r} v_{1,r}(r) \cos\theta + \hat{\theta} v_{1,\theta}(r) \sin\theta \, , \\
    \var{q} &= q_1(r) \cos\theta \, , &
    \var{E} &= E_1(r) \cos\theta \, . &
\end{aligned}
\end{align}

(Now is maybe the point to start explaining the normalization. Or maybe add a note to say what positive and negative $y_1$ mean. Also the boundary conditions need to be mentioned)





    \subsubsection{Analytical description}
    \subsubsection{Numerical solution}