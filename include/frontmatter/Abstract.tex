% CREATED BY MAGNUS GUSTAVER, 2020
\begin{raggedright}
\thetitle\\
%\thesubtitle\\
\theauthor\\
\thedepartment\\
Chalmers University of Technology\\
\if\InstitutionLocation G
University of Gothenburg\\
\fi
\end{raggedright}

\thispagestyle{plain}			% Supress header 
\section*{Abstract}

Nuclear fusion power generation is widely regarded as a promising technology for a clean and sustainable future.
To achieve positive energy gain from fusion, the fuel (deuterium and tritium) has to be sustained at temperatures around ten times hotter than the core of the sun.
Most of the power plant designs are currently based on magnetic confinement fusion (MCF), where this fusion plasma is isolated and compressed by strong magnetic fields.
Refuelling, control of the plasma density profile and mitigation of off-normal events (disruptions) is done by injecting fast tiny pellets of dense frozen material, which quickly disintegrate from the extreme heat in the fusion plasma.
While the efficiency of this material deposition in the plasma depends on the pellet trajectory, the dynamics involved are poorly understood.
This thesis develops a semi-analytical model for the so-called \emph{pellet rocket effect}, which accelerates and deflects pellets in a fusion plasma.
Asymmetries in the heat flux onto the pellet surface enhance the ablation on one side of the pellet.
Consequently, the pellet is pushed in the opposite direction to the ejected material, similarly to a rocket.
This effect was shown in experiments to significantly modify the pellet trajectory.
Projections for reactor scale devices indicate that it may even stop the pellet before it reaches the plasma core, which would severely limit the effectiveness of pellet injection methods.
Our model predicts magnitudes of the pellet rocket acceleration ($\qtyrange{e5}{e6}{\m/\s^2}$) similar to the results of more sophisticated simulations and experimental observations.
%Even though our model could not be sufficiently validated, a similar approach is expected 
Thus, the essential physics of the pellet rocket effect are successfully captured in our model.
With further validation and potential improvements, we expect that this approach can be used
to inform the planning and operation of pellet injection in future power plants.


% KEYWORDS (MAXIMUM 10 WORDS)
\vfill
\textbf{Keywords:} fusion plasma, magnetic confinement, pellet injection, ablation, pellet rocket acceleration, asymmetric heating, NGS model
% set the PDF metadata keywords

\newpage				% Create empty back of side
\thispagestyle{empty}
\mbox{}