\chapter{Modelling the pellet rocket effect}

The underlying principle of the pellet rocket effect is that asymmetric heating of the pellet and the ablation cloud leads to a higher ablation and pressure on one side of the pellet. 
This results in a rocket-like force on the pellet, which accelerates it towards the less heated side.
In general, this phenomenon involves complex, three-dimensional, non-linear fluid dynamics.
Fortunately, in the case of pellet injection in magnetic confinement fusion, the pellet heating can be seen as predominantly spherically symmetric.
Taking the anisotropic pellet ablation dynamics as a linear perturbation on top of the isotropic dynamics, enables us to develop a semi-analytical model for the pellet rocket effect.

Conceptually, this model is divided in three different parts.
The first part, as described in \cref{sec:pellet_surface}, considers the momentum transfer on the pellet surface to calculate the net force on the pellet.
Expanding the angular dependence of the gas properties at the pellet surface in spherical harmonics, yields a formula for the pellet rocket force, involving only the $\cos\theta$-dependence of the ablation rate and pressure at the pellet surface.

The second part of the model, as described in \cref{sec:neutral_gas_shielding}, quantifies the shielding of the pellet from an external asymmetric heating source through the neutral gas ablation cloud.
Widely used for predicting isotropic pellet ablation dynamics is a model known as the Neutral Gas Shielding (NGS) model, developed by \textcite{parks_effect_1978}.
There, quasi-steady-state fluid equations for an ideal gas are solved under the approximation of a mono-energetic beam of incoming electrons, which traverse the ablation cloud radially.
This model is in good agreement with experimental data and while several more sophisticated models were developed, most models yield the same results \autocite{pegourie_review_2007}.
Therefore, the NGS model is the basis of our model.
The asymmetric ablation dynamics are then added as a perturbation.
Since the full radial solution of the NGS model is needed to calculate the asymmetric dynamics, the NGS model has to be fully reproduced, as described in \cref{ssec:isotropic_model}.
The asymmetric perturbation dynamics are then modelled, as described in \cref{ssec:asymmetric_perturbation_model} given the external heating source.

While this model can be used by itself, if the heat source into the neutral ablation cloud is known, our goal is to predict the pellet rocket acceleration along the major radius of a tokamak from the background plasma parameters.
Therefore, the third part of our model, as described in \cref{sec:plasmoid_shielding}, calculates the heat source asymmetry induced by the outward drift of the plasmoid of ablated material.
How this drift results in a finite and varying shielding length of the electrons through the plasmoid is described in \cref{ssec:plasmoid_shielding_length}.
In \cref{ssec:plasmoid_heat_flux_attenuation}, the shielding length is used to estimate the heat source at the boundary of the neutral gas cloud.

Finally, in \cref{sec:model_summary}, all three parts are combined to describe the full model.

%%%%%%%%%%%%%%%%%%%%%%%%%%%%%%%%%%%%%%%%%%%%%%%%%%%%%%%%%%%%%%%%%%%%%%%%%%%%%%%
\section{Force on the pellet surface}
\label{sec:pellet_surface}

Consider a spherical pellet of radius $r_\text{p}$ surrounded by a neutral gas.
Physically, the force on the pellet is the combination of ablated particles leaving the pellet surface and the gas pressure pushing on the pellet surface.
In the following, a formula for this force is derived under the assumption that the anisotropic dynamics are small compared to the spherically symmetric dynamics.

Mathematically, the momentum transfer at a point $\vec{r}$ in the gas is expressed through the momentum flux tensor
\begin{equation*}
    \tensor{\Pi} = \rho \left(\vec{v} \otimes \vec{v} \right) + p \tensor{\mathbbone} \, ,
\end{equation*}
with the mass density $\rho(\vec{r})$, the fluid velocity $\vec{v}(\vec{r})$ and the pressure $p(\vec{r})$. 
The symbol $\otimes$ denotes the dyadic product\footnote{equivalent to a tensor product} and $\tensor{\mathbbone}$ denotes the 3-dimensional identity tensor.  
Assuming local momentum conservation at the pellet surface $S$, the net force on the pellet is
\begin{equation*}
    \vec{F} = - \iint_{S} \tensor{\Pi} \cdot \dd{\vec{S}} \, ,
\end{equation*}
where the minus sign indicates that this is the force exerted on the pellet, while the surface element $\dd{\vec{S}}$ points outwards.
\begin{figure}
    \centering
    \includegraphics{figure/pellet_surface_force.pdf}
    \caption{Caption}
    \label{fig:pellet_surface_force}
\end{figure}
Choosing spherical coordinates $r,\theta,\varphi$ so that the pellet is centred at the origin and a positive force $F$ points in the negative $\hat{z}$-direction, as shown in \cref{fig:pellet_surface_force}, the force becomes
\begin{equation*}
    F = - \hat{z} \cdot \vec{F} = r_\text{p}^2 \int_{\theta=0}^{\pi} \int_{\varphi=0}^{2\pi} \left( \rho \left(\hat{z}\cdot  \vec{v} \otimes \vec{v}  \cdot \hat{r}\right) + p \left(\hat{z}\cdot\hat{r}\right) \right) \,\dd{\Omega} \, ,
\end{equation*}
with the differential solid angle $\dd{\Omega}=\sin\theta \dd{\theta}\dd{\varphi}$. 
Expressing $\vec{v} = v_r \hat{r} + v_\theta \hat{\theta} + v_\varphi \hat{\varphi}$ and using geometric relations between the unit vectors leads to
\begin{equation*}
    F = r_\text{p}^2 \int_{\theta=0}^{\pi} \int_{\varphi=0}^{2\pi} \left( \rho v_r (v_r \cos\theta + v_\theta \sin\theta) + p \cos\theta \right) \,\dd{\Omega} \, .
\end{equation*}
At this point, the first and only approximation for deriving the force formula has to be made.
The anisotropic dynamics are taken as a small perturbation on the spherically symmetric dynamics in the form
\begin{align*}
    v_r(\vec{r}) &= v_0(r) + \var{v_r}(r,\theta,\varphi), &
    v_\theta(\vec{r}) &= 0 + \var{v_\theta}(r,\theta,\varphi), \\
    \rho(\vec{r}) &= \rho_0(r) + \var{\rho}(r,\theta,\varphi), &
    p(\vec{r}) &= p_0(r) + \var{p}(r,\theta,\varphi) \, .
\end{align*}
Linearization of the force in this perturbation and using $\int_0^\pi \cos\theta \dd{\theta} = 0$ leads to
\begin{equation}
    F = r_\text{p}^2 \int_{\theta=0}^{\pi} \int_{\varphi=0}^{2\pi} \left(\left( \var{\rho} v_0^2 + 2 \rho_0 v_0 \var{v_r} + \var{p} \right) \cos\theta + \rho_0 v_0 \var{v_\theta} \sin\theta \right) \,\dd{\Omega} \, .
    \label{eq:force_before_expansion}
\end{equation}
The last term can be rewritten as a $\cos\theta$ dependent term through integration by parts 
\begin{equation*}
    \int_0^\pi \var{v_\theta} \sin^2\theta \dd{\theta} = \underbrace{\left[ \left(\int \var{v_\theta} \dd{\theta}\right) \sin^2\theta \right]_0^\pi}_{=0} - \int_0^\pi \left(\int \var{v_\theta} \dd{\theta}\right) 2\cos\theta\sin\theta \dd{\theta}
\end{equation*}
assuming that $\left(\int \var{v_\theta} \dd{\theta}\right)$ is finite.
The surface integral can be then solved by expanding $\var{\rho}$, $\var{v_r}$, $\var{p}$ and $\left(\int \var{v_\theta} \dd{\theta}\right)$  in terms of spherical harmonics 
\begin{equation*}
    Y_l^m(\theta,\varphi) = \sqrt{\frac{(l-m)!}{(l+m)!}} \mathcal{P}_l^m(\cos\theta) e^{im\varphi} \, ,
\end{equation*}
with the associated Legendre polynomials $\mathcal{P}_l^m$.
Spherical harmonics are orthogonal in the sense
\begin{equation*}
    \int_{\theta=0}^{\pi} \int_{\varphi=0}^{2\pi} Y_l^m (Y_{l'}^{m'})^* \dd{\Omega} = 
    \begin{cases}
        \frac{4 \pi}{2l+1} & \text{for }l=l' \, , \, m=m' \\
        0 & \text{else}
    \end{cases} \, ,
\end{equation*}
which is helpful because ${Y_1^0=\cos\theta}$ appears in the surface integral \cref{eq:force_before_expansion}.
Therefore, for any general $\var{\rho}$, $\var{v_r}$, $\var{p}$ and $\int \var{v_\theta} \dd{\theta}$, only their projection onto $\cos\theta$, i.e. the coefficient of the $l=1$, $m=0$ mode, contributes to the force.
Inserting the expansions
\begin{align*}
    \var{\rho}(r,\theta,\varphi) &= \rho_1(r)\cos\theta + \dots \, , & 
    \var{p}(r,\theta,\varphi) &= p_1(r)\cos\theta + \dots \, , \\
    \var{v_r}(r,\theta,\varphi) &= v_{1,r}(r)\cos\theta + \dots \, , &
    \var{v_\theta}(r,\theta,\varphi) &= v_{1,\theta}(r)\sin\theta + \dots
\end{align*}
into \cref{eq:force_before_expansion} yields the formula for the pellet rocket force
\begin{equation}
    \boxed{F = \frac{4 \pi r_\text{p}^2}{3} \left( \rho_1 v_0^2 + 2 \rho_0 v_0 (v_{1,r} + v_{1,\theta}) + p_1 \right)_{r=r_\text{p}}} \, .
    \label{eq:rocket_force_full}
\end{equation}
This formula can also be understood physically.
The ablation rate per unit area $g$, i.e. the mass flux through the pellet surface, is
\begin{equation*}
    g(\theta) = \rho \vec{v} \cdot \hat{r} \approx \left( \rho_0 v_0 + (\rho_1 v_0 + \rho_0 v_{1,r}) \cos\theta + \dots \right)_{r=r_\text{p}} = g_0 + g_1 \cos\theta + \dots \, .
\end{equation*}
Therefore, the first two terms in \cref{eq:rocket_force_full} describe the force arising from an asymmetric ablation, the term $\rho_0 v_0 v_{1,\theta}$ describes a force from mass flowing around the pellet surface and the last term describes the gas pressure asymmetry. 
All of this is integrated over the pellet surface area $4 \pi r_\text{p}^2$.

Under the self-regulating shielding assumptions, used in the next parts of this model, the pellet rocket force is predominantly caused by the pressure asymmetry in the neutral gas ablation cloud.
Therefore, the pellet rocket force is
\begin{equation}
    F = \frac{4 \pi r_\text{p}^2}{3} p_1(r_\text{p}) \, ,
\end{equation}
which is the same formula as used for the empirical model developed by \textcite{szepesi_radial_2007}.
The main challenge of this thesis is to develop a model for the pressure asymmetry at the pellet surface $p_1(r_\text{p})$ given an external heating source.
This is the subject of the following sections.

%%%%%%%%%%%%%%%%%%%%%%%%%%%%%%%%%%%%%%%%%%%%%%%%%%%%%%%%%%%%%%%%%%%%%%%%%%%%%%%%%%%
\section{Neutral gas shielding}
\label{sec:neutral_gas_shielding}

As mentioned before, the basis of our model is the NGS model, developed by \textcite{parks_effect_1978}.
Most of this section is dedicated to describing and reproducing this semi-analytical model.
The underlying physics processes related to ablation of hydrogen pellets are described already in \cref{sec:ablation_dynamics} and will not be repeated here.
Instead, the following text describes first general ideas and approximations of the NGS model. 
Then the mathematical details of how solutions are found are presented in \cref{ssec:isotropic_model}.
Finally, \cref{ssec:asymmetric_perturbation_model} describes the perturbative extension to the NGS model, which we have developed.

\begin{figure}
    \centering
    \includegraphics{figure/neutral_cloud_heating.pdf}
    \caption{Caption}
    \label{fig:neutral_cloud_heating}
\end{figure}

Since the neutral ablation cloud can be considered a transonic ideal gas, the equation of state is
\begin{align}
    &\rho = m \frac{p}{T}  &\text{(ideal gas law)} \, , \label{eq:ideal_gas_law}
%\end{align}
\intertext{%
with mass density $\rho$, pressure $p$, temperature $T$ (in units of energy through Boltzmann's constant $k_\text{B}$) and mass $m$ of one molecule (or atom) in the gas.
The full gas dynamics are obtained by considering the steady state conservation laws
}
%\begin{align}
&\vec{\nabla} \cdot (\rho \vec{v}) = 0 &\text{(mass conservation)} \, , \label{eq:full_mass_conservation} \\
&\rho (\vec{v} \cdot \vec{\nabla}) \vec{v} = - \vec{\nabla} p &\text{(momentum conservation)} \, , \label{eq:full_momentum_conservation} \\
&\vec{\nabla} \cdot \left[\left( \frac{\rho v^2}{2} + \frac{\gamma p}{\gamma - 1} \right) \vec{v}\right] = \dv{Q}{t} &\text{(energy conservation)} \, ,\label{eq:full_energy_conservation}
\end{align}
also known as Euler equations, with flow velocity $\vec{v}$ and internal heat $Q$.
$\gamma = C_p/C_V$ is the adiabatic index of the gas or heat capacity ratio.
Since in an ideal gas the adiabatic index is related to the number of degrees of freedom $f$ of the gas particles through $\gamma = 1+2/f$, it can be assumed to be $\gamma = 5/3$ for a monatomic gas, $\gamma = 7/5$ for a diatomic gas and $\gamma = 9/7$ for a linear triatomic gas.
In the case of frozen deuterium pellets, the temperature close to the pellet is so low, that dissociation can be neglected, and the whole ablation cloud is modelled as a \ce{D2} gas.

The external heat source $\dd{Q}/\dd{t}$ at each point in the gas is subject to the major approximations done in the NGS model.
Heat conduction in the neutral cloud is neglected, since this occurs only on a much slower timescale than the heating from highly energetic electrons loosing energy in the cloud through elastic backscattering and inelastic processes \autocite{parks_effect_1978}.
While, in reality, the incoming electrons are constrained by the magnetic field to traverse the cloud on straight parallel lines, as illustrated in \cref{fig:neutral_cloud_heating}, the NGS model assumes spherical symmetry.
Therefore, the heat source
\begin{equation}
    \dv{Q}{t} \approx - \mu \vec{\nabla}\cdot \vec{q} = \mu \left( \pdv{q}{r} + \frac{2}{r}q \right) \approx \mu \pdv{q}{r}
    \label{eq:heat_flux_approximation}
\end{equation}
is modelled as if the electrons lose their energy on a radial path towards the pellet. 
Only a fraction $\mu$ of this energy loss goes into heating the gas, while the rest goes into Bremsstrahlung radiation and backscattered electrons.
\textcite{parks_effect_1978} state that the fraction $\mu$ is fairly well modelled to be between $60 \%$ and $70\%$ at all points in the cloud.
The exact reasoning for neglecting the term $\frac{2}{r}q$ is not clearly stated by \textcite{parks_effect_1978}.
However, this presumably is related to an otherwise overestimated heat flux near the pellet surface and considering the success of the NGS model, this assumption is retained in our model.
Additionally, the incident electrons are approximated to all have the same energy once they reach the point $\vec{r}$ in the neutral ablation cloud.
These major approximations of a mono-energetic beam of incoming electrons loosing energy radially have been found to be sufficiently accurate for predicting ablation rates in magnetic confinement fusion.
How much those approximations degrade the accuracy for predicting the pellet rocket effect has to be evaluated by comparison to experiments.

Modelling the exact dynamics for how the electrons lose energy on their path through the gas is a non-trivial task.
Therefore, the NGS model uses empirical scaling laws for the scattering cross-sections of electrons in a hydrogen gas.
The electron dynamics are thus governed by the differential equations
\begin{align}
    &\dv{E}{r} = 2 \frac{\rho}{m} L(E) &\text{(electron energy loss)} \, , \label{eq:full_electron_energy_loss} \\
    &\dv{q}{r} = \frac{q}{\lambda_\text{mfp}(E)} = \frac{\rho}{m} q \Lambda(E) &\text{(effective heat flux)}  \, , \label{eq:full_effective_heat_flux}
\end{align}
where the mean free path $\lambda_\text{mfp}$ is modelled through the effective energy flux cross-section $\Lambda(E) = \hat{\sigma}_T(E) + 2 L(E)/E$ with the empirical energy loss function
\begin{gather}
L(E>\qty{20}{\eV}) = \frac{\qty{8.62e-15}{\eV.\cm^2}}{\left(\frac{E/\unit{\eV}}{100}\right)^{0.823} \!\!\!\! +\left(\frac{E/\unit{\eV}}{60}\right)^{-0.125}  \!\!\!\! +\left(\frac{E/\unit{\eV}}{48}\right)^{-1.94}}
\end{gather}
derived by \textcite{miles_electron-impact_1972} and the corresponding effective backscattering cross-section
\begin{equation}
\hat{\sigma}_T (E) = \begin{dcases}
    \left(\frac{\num{8.8e-13}}{(E/\unit{\eV})^{1.71}}-\frac{\num{1.62e-12}}{(E/\unit{\eV})^{1.932}} \right)\, \unit{\cm^2} & \text{for } E>\qty{100}{eV} \\
    \left(\frac{\num{1.1e-14}}{(E/\unit{\eV})} \right) \unit{\cm^2} & \text{for } E<\qty{100}{\eV}
\end{dcases}
\end{equation}
derived by \textcite{parks_model_1977-1} based on experimentally measured values by \textcite{maecker_ionen-_1955}.

The interdependence between the electron heat flux and the ablation cloud density at each point means that the full system of \cref{eq:ideal_gas_law,eq:full_mass_conservation,eq:full_momentum_conservation,eq:full_energy_conservation,eq:heat_flux_approximation,eq:full_effective_heat_flux,eq:full_electron_energy_loss} needs to be solved self-consistently.
For that, enough boundary conditions have to be motivated that describe the physical system.
The low sublimation energy of hydrogen is the reason that the ablation cloud establishes itself to nearly fully shield the pellet.
Therefore, the heat flux as well as the gas temperature at the pellet surface can be neglected and considered to be exactly zero.
The outer boundary of the neutral ablation cloud to the plasmoid ablation cloud establishes itself as a shock front, where the flow velocity rapidly decreases to adjust to the pressure in the plasmoid.
Considering that the back pressure through the ablation cloud and the heat conduction play no significant role in the ablation dynamics, the boundary of the neutral ablation cloud can be seen as infinitely far away from the pellet with negligible pressure.
Given the heat flux and average energy of the electrons reaching the ablation cloud, the boundary conditions can be summarized as
\begin{equation}
\begin{gathered}
    q(r_\text{p})=0, \quad T(r_\text{p})=0, \\
    p(r \rightarrow \infty) = 0, \quad
    q(r\rightarrow\infty) = q_\text{bc}, \quad E(r\rightarrow\infty) = E_\text{bc} \, .
    \label{eq:full_boundary_conditions}
\end{gathered}
\end{equation}
Where $q_\text{bc}(\theta,\varphi)$ and $E_\text{bc}(\theta,\varphi)$ depends on the amount of plasmoid shielding from the background plasma electrons at temperature $T_\text{bg}$ and density $n_\text{bg}$, as modelled in \cref{sec:plasmoid_shielding}.
Note that no assumptions have to be made on the velocity at which ablated molecules leave the pellet surface, but it is found to be negligible in the numerical solutions.
However, for our anisotropic perturbative model, we have to assume that the molecules leave the pellet surface only with a radial velocity and the angular velocity is assumed in our model to be zero.
        
\subsection{Isotropic model}
\label{ssec:isotropic_model}










            \subsubsection{Analytical description}
            \subsubsection{Numerical solution}
        \subsection{Asymmetric perturbation model}
        \label{ssec:asymmetric_perturbation_model}
            \subsubsection{Analytical description}
            \subsubsection{Numerical solution}
    \section{Plasmoid shielding}
    \label{sec:plasmoid_shielding}
        \subsection{Drift induced shielding length asymmetry}
        \label{ssec:plasmoid_shielding_length}
        \subsection{Heat flux attenuation}
        \label{ssec:plasmoid_heat_flux_attenuation}
    \section{Model summary (putting it all together...)}
    \label{sec:model_summary}