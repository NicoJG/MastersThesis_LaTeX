\chapter{Modelling the pellet rocket effect}

The underlying principle of the pellet rocket effect is that asymmetric heating of the pellet and the ablation cloud leads to a higher ablation and pressure on one side of the pellet. 
This results in a rocket-like force on the pellet, which accelerates it towards the less heated side.
In general, this phenomenon involves complex, three-dimensional, non-linear fluid dynamics.
Fortunately, in the case of pellet injection in magnetic confinement fusion, the pellet heating can be seen as predominantly spherically symmetric.
Taking the anisotropic pellet ablation dynamics as a linear perturbation on top of the isotropic dynamics, enables us to develop a semi-analytical model for the pellet rocket effect.

Conceptually, this model is divided in three different parts.
The first part, as described in \cref{sec:pellet_surface}, considers the momentum transfer on the pellet surface to calculate the net force on the pellet.
Expanding the angular dependence of the gas properties at the pellet surface in spherical harmonics, yields a formula for the pellet rocket force, involving only the $\cos\theta$-dependence of the ablation rate and pressure at the pellet surface.

The second part of the model, as described in \cref{sec:neutral_gas_shielding}, quantifies the shielding of the pellet from an external asymmetric heating source through the neutral gas ablation cloud.
Widely used for predicting isotropic pellet ablation dynamics is a model known as the Neutral Gas Shielding (NGS) model, developed by \textcite{parks_effect_1978}.
There, quasi-steady-state fluid equations for an ideal gas are solved under the approximation of a mono-energetic beam of incoming electrons, which traverse the ablation cloud radially.
This model is in good agreement with experimental data and while several more sophisticated models were developed, most models yield the same results \autocite{pegourie_review_2007}.
Therefore, the NGS model is the basis of our model.
The asymmetric ablation dynamics are then added as a perturbation.
Since the full radial solution of the NGS model is needed to calculate the asymmetric dynamics, the NGS model has to be fully reproduced, as described in \cref{ssec:isotropic_model}.
The asymmetric perturbation dynamics are then modelled, as described in \cref{ssec:asymmetric_perturbation_model} given the external heating source.

While this model can be used by itself, if the heat source into the neutral ablation cloud is known, our goal is to predict the pellet rocket acceleration along the major radius of a tokamak from the background plasma parameters.
Therefore, the third part of our model, as described in \cref{sec:plasmoid_shielding}, calculates the heat source asymmetry induced by the outward drift of the plasmoid of ablated material.
How this drift results in a finite and varying shielding length of the electrons through the plasmoid is described in \cref{ssec:plasmoid_shielding_length}.
In \cref{ssec:plasmoid_heat_flux_attenuation}, the shielding length is used to estimate the heat source at the boundary of the neutral gas cloud.

Finally, in \cref{sec:model_summary}, all three parts are combined to describe the full model.

%%%%%%%%%%%%%%%%%%%%%%%%%%%%%%%%%%%%%%%%%%%%%%%%%%%%%%%%%%%%%%%%%%%%%%%%%%%%%%%
\section{Force on the pellet surface}
\label{sec:pellet_surface}

Consider a spherical pellet of radius $r_\text{p}$ surrounded by a neutral gas.
Physically, the force on the pellet is the combination of ablated particles leaving the pellet surface and the gas pressure pushing on the pellet surface.
In the following, a formula for this force is derived under the assumption that the anisotropic dynamics are small compared to the spherically symmetric dynamics.

Mathematically, the momentum transfer at a point $\vec{r}$ in the gas is expressed through the momentum flux tensor
\begin{equation*}
    \tensor{\Pi} = \rho \left(\vec{v} \otimes \vec{v} \right) + p \tensor{\mathbbone} \, ,
\end{equation*}
with the mass density $\rho(\vec{r})$, the fluid velocity $\vec{v}(\vec{r})$ and the pressure $p(\vec{r})$. 
The symbol $\otimes$ denotes the dyadic product\footnote{equivalent to a tensor product} and $\tensor{\mathbbone}$ denotes the 3-dimensional identity tensor.  
Assuming local momentum conservation at the pellet surface $S$, the net force on the pellet is
\begin{equation*}
    \vec{F} = - \iint_{S} \tensor{\Pi} \cdot \dd{\vec{S}} \, ,
\end{equation*}
where the minus sign indicates that this is the force exerted on the pellet, while the surface element $\dd{\vec{S}}$ points outwards.
\begin{figure}
    \centering
    \includegraphics{figure/pellet_surface_force.pdf}
    \caption{Illustration of how the pellet rocket force arises from both an asymmetry in pressure on the pellet surface (visualized by the red background) and an asymmetric ablation (visualized by the vectors). Additionally, the coordinate system used throughout this thesis is denoted by the unit vectors $\hat{z}$, $\hat{r}$, $\hat{\theta}$. The pellet is modelled as a solid sphere of radius $r_\text{p}$.}
    \label{fig:pellet_surface_force}
\end{figure}
Choosing spherical coordinates $r,\theta,\varphi$ so that the pellet is centred at the origin and a positive force $F$ points in the negative $\hat{z}$-direction, as shown in \cref{fig:pellet_surface_force}, the force becomes
\begin{equation*}
    F = - \hat{z} \cdot \vec{F} = r_\text{p}^2 \int_{\theta=0}^{\pi} \int_{\varphi=0}^{2\pi} \left( \rho \left(\hat{z}\cdot  \vec{v} \otimes \vec{v}  \cdot \hat{r}\right) + p \left(\hat{z}\cdot\hat{r}\right) \right) \,\dd{\Omega} \, ,
\end{equation*}
with the differential solid angle $\dd{\Omega}=\sin\theta \dd{\theta}\dd{\varphi}$. 
Expressing $\vec{v} = v_r \hat{r} + v_\theta \hat{\theta} + v_\varphi \hat{\varphi}$ and using geometric relations between the unit vectors leads to
\begin{equation*}
    F = r_\text{p}^2 \int_{\theta=0}^{\pi} \int_{\varphi=0}^{2\pi} \left( \rho v_r (v_r \cos\theta + v_\theta \sin\theta) + p \cos\theta \right) \,\dd{\Omega} \, .
\end{equation*}
At this point, the first and only approximation for deriving the force formula has to be made.
The anisotropic dynamics are taken as a small perturbation on the spherically symmetric dynamics in the form
\begin{align*}
    v_r(\vec{r}) &= v_0(r) + \var{v_r}(r,\theta,\varphi), &
    v_\theta(\vec{r}) &= 0 + \var{v_\theta}(r,\theta,\varphi), \\
    \rho(\vec{r}) &= \rho_0(r) + \var{\rho}(r,\theta,\varphi), &
    p(\vec{r}) &= p_0(r) + \var{p}(r,\theta,\varphi) \, .
\end{align*}
Linearization of the force in this perturbation and using $\int_0^\pi \cos\theta \dd{\theta} = 0$ leads to
\begin{equation}
    F = r_\text{p}^2 \int_{\theta=0}^{\pi} \int_{\varphi=0}^{2\pi} \left(\left( \var{\rho} v_0^2 + 2 \rho_0 v_0 \var{v_r} + \var{p} \right) \cos\theta + \rho_0 v_0 \var{v_\theta} \sin\theta \right) \,\dd{\Omega} \, .
    \label{eq:force_before_expansion}
\end{equation}
The last term can be rewritten as a $\cos\theta$ dependent term through integration by parts 
\begin{equation*}
    \int_0^\pi \var{v_\theta} \sin^2\theta \dd{\theta} = \underbrace{\left[ \left(\int \var{v_\theta} \dd{\theta}\right) \sin^2\theta \right]_0^\pi}_{=0} - \int_0^\pi \left(\int \var{v_\theta} \dd{\theta}\right) 2\cos\theta\sin\theta \dd{\theta}
\end{equation*}
assuming that $\left(\int \var{v_\theta} \dd{\theta}\right)$ is finite.
The surface integral can be then solved by expanding $\var{\rho}$, $\var{v_r}$, $\var{p}$ and $\left(\int \var{v_\theta} \dd{\theta}\right)$  in terms of spherical harmonics 
\begin{equation*}
    Y_l^m(\theta,\varphi) = \sqrt{\frac{(l-m)!}{(l+m)!}} \mathcal{P}_l^m(\cos\theta) e^{im\varphi} \, ,
\end{equation*}
with the associated Legendre polynomials $\mathcal{P}_l^m$.
Spherical harmonics are orthogonal in the sense
\begin{equation*}
    \int_{\theta=0}^{\pi} \int_{\varphi=0}^{2\pi} Y_l^m (Y_{l'}^{m'})^* \dd{\Omega} = 
    \begin{cases}
        \frac{4 \pi}{2l+1} & \text{for }l=l' \, , \, m=m' \\
        0 & \text{else}
    \end{cases} \, ,
\end{equation*}
which is helpful because ${Y_1^0=\cos\theta}$ appears in the surface integral \cref{eq:force_before_expansion}.
Therefore, for any general $\var{\rho}$, $\var{v_r}$, $\var{p}$ and $\int \var{v_\theta} \dd{\theta}$, only their projection onto $\cos\theta$, i.e. the coefficient of the $l=1$, $m=0$ mode, contributes to the force.
Inserting the expansions
\begin{align*}
    \var{\rho}(r,\theta,\varphi) &= \rho_1(r)\cos\theta + \dots \, , & 
    \var{p}(r,\theta,\varphi) &= p_1(r)\cos\theta + \dots \, , \\
    \var{v_r}(r,\theta,\varphi) &= v_{1,r}(r)\cos\theta + \dots \, , &
    \var{v_\theta}(r,\theta,\varphi) &= v_{1,\theta}(r)\sin\theta + \dots
\end{align*}
into \cref{eq:force_before_expansion} yields the formula for the pellet rocket force
\begin{equation}
    \boxed{F = \frac{4 \pi r_\text{p}^2}{3} \left( \rho_1 v_0^2 + 2 \rho_0 v_0 (v_{1,r} + v_{1,\theta}) + p_1 \right)_{r=r_\text{p}}} \, .
    \label{eq:rocket_force_full}
\end{equation}
This formula can also be understood physically.
The ablation rate per unit area $g$, i.e. the mass flux through the pellet surface, is
\begin{equation*}
    g(\theta) = \rho \vec{v} \cdot \hat{r} \approx \left( \rho_0 v_0 + (\rho_1 v_0 + \rho_0 v_{1,r}) \cos\theta + \dots \right)_{r=r_\text{p}} = g_0 + g_1 \cos\theta + \dots \, .
\end{equation*}
Therefore, the first two terms in \cref{eq:rocket_force_full} describe the force arising from an asymmetric ablation, the term $\rho_0 v_0 v_{1,\theta}$ describes a force from mass flowing around the pellet surface and the last term describes the gas pressure asymmetry. 
All of this is integrated over the pellet surface area $4 \pi r_\text{p}^2$.

Under the self-regulating shielding assumptions, used in the next parts of this model, the pellet rocket force is predominantly caused by the pressure asymmetry in the neutral gas ablation cloud.
Therefore, the pellet rocket force is
\begin{equation}
    F = \frac{4 \pi r_\text{p}^2}{3} p_1(r_\text{p}) \, ,
\end{equation}
which is the same formula as used for the empirical model developed by \textcite{szepesi_radial_2007}.
The main challenge of this thesis is to develop a model for the pressure asymmetry at the pellet surface $p_1(r_\text{p})$ given an external heating source.
This is the subject of the following sections.

%%%%%%%%%%%%%%%%%%%%%%%%%%%%%%%%%%%%%%%%%%%%%%%%%%%%%%%%%%%%%%%%%%%%%%%%%%%%%%%%%%%
\section{Neutral gas shielding}
\label{sec:neutral_gas_shielding}

As mentioned before, the basis of our model is the NGS model, developed by \textcite{parks_effect_1978}.
Most of this section is dedicated to describing and reproducing this semi-analytical model.
The underlying physics processes related to ablation of hydrogen pellets are described already in \cref{sec:ablation_dynamics} and will not be repeated here.
Instead, the following text describes first general ideas and approximations of the NGS model. 
Then the mathematical details of how solutions are found are presented in \cref{ssec:isotropic_model}.
Finally, \cref{ssec:asymmetric_perturbation_model} describes the perturbative extension to the NGS model, which we have developed.

Since the neutral ablation cloud can be considered a transonic ideal gas, the equation of state is
\begin{align}
    &\frac{\rho}{m} = \frac{p}{T}  &\text{(ideal gas law)} \, , \label{eq:ideal_gas_law}
%\end{align}
\intertext{%
with mass density $\rho$, pressure $p$, temperature $T$ (in units of energy through Boltzmann's constant $k_\text{B}$) and mass $m$ of one molecule (or atom) in the gas.
The full gas dynamics are obtained by considering the steady state conservation laws
}
%\begin{align}
&\vec{\nabla} \cdot (\rho \vec{v}) = 0 &\text{(mass conservation)} \, , \label{eq:full_mass_conservation} \\
&\rho (\vec{v} \cdot \vec{\nabla}) \vec{v} = - \vec{\nabla} p &\text{(momentum conservation)\footnotemark} \, , \label{eq:full_momentum_conservation} \\
&\vec{\nabla} \cdot \left[\left( \frac{\rho v^2}{2} + \frac{\gamma p}{\gamma - 1} \right) \vec{v}\right] = Q &\text{(energy conservation)} \, ,\label{eq:full_energy_conservation}
\end{align}
\footnotetext{The directional derivative of $\vec{b}$ along $\vec{a}$ is denoted as $(\vec{a}\cdot\vec{\nabla})\vec{b}$.}%
also known as Euler equations, with flow velocity $\vec{v}$ and external heat source $Q$.
$\gamma = C_p/C_V$ is the adiabatic index of the gas or heat capacity ratio.
Since in an ideal gas the adiabatic index is related to the number of degrees of freedom $f$ of the gas particles through $\gamma = 1+2/f$, it can be assumed to be $\gamma = 5/3$ for a monatomic gas, $\gamma = 7/5$ for a diatomic gas and $\gamma = 9/7$ for a linear triatomic gas.
In the case of frozen deuterium pellets, the temperature close to the pellet is so low, that dissociation can be neglected, and the whole ablation cloud is modelled as a \ce{D2} gas.

\begin{figure}
    \centering
    \includegraphics{figure/neutral_cloud_heating.pdf}
    \caption{Illustration of the asymmetric heating of the neutral gas ablation cloud by incoming electrons travelling along the magnetic field lines. The boundary between the neutral ablation cloud and the ionized ablation cloud is considered as $r\rightarrow\infty$. The electrons are modelled to lose their energy radially inwards, while in reality they traverse the neutral ablation cloud on parallel straight lines from two sides.}
    \label{fig:neutral_cloud_heating}
\end{figure}

The external heat source $Q$ at each point in the gas is subject to the major approximations done in the NGS model.
Heat conduction in the neutral cloud is neglected, since this occurs only on a much slower timescale than the heating from highly energetic electrons loosing energy in the cloud through elastic backscattering and inelastic processes \autocite{parks_effect_1978}.
While, in reality, the incoming electrons are constrained by the magnetic field to traverse the cloud on straight parallel lines, as illustrated in \cref{fig:neutral_cloud_heating}, the NGS model assumes spherical symmetry.
Therefore, the heat source
\begin{equation}
    Q \approx - \mu \vec{\nabla}\cdot \vec{q} = \mu \left( \pdv{q}{r} + \frac{2}{r}q \right) \approx \mu \pdv{q}{r}
    \label{eq:heat_flux_approximation}
\end{equation}
is modelled as if the electrons lose their energy on a radial path towards the pellet. 
\Cref{fig:neutral_cloud_heating} also shows the coordinate system used throughout our model.
Only a fraction $\mu$ of this energy loss goes into heating the gas, while the rest goes into Bremsstrahlung radiation and backscattered electrons.
\textcite{parks_effect_1978} state that the fraction $\mu$ is fairly well modelled to be between $60 \%$ and $70\%$ at all points in the cloud.
The exact reasoning for neglecting the term $\frac{2}{r}q$ is not clearly stated by \textcite{parks_effect_1978}.
However, this presumably is related to an otherwise overestimated heat flux near the pellet surface and considering the success of the NGS model, this assumption is retained in our model.
Additionally, the incident electrons are approximated to all have the same energy once they reach the point $\vec{r}$ in the neutral ablation cloud.
These major approximations of a mono-energetic beam of incoming electrons loosing energy radially have been found to be sufficiently accurate for predicting ablation rates in magnetic confinement fusion.
How much those approximations degrade the accuracy for predicting the pellet rocket effect has to be evaluated by comparison to experiments.

Modelling the exact dynamics for how the electrons lose energy on their path through the gas is a non-trivial task.
Therefore, the NGS model uses empirical scaling laws for the scattering cross-sections of electrons in a hydrogen gas.
The electron dynamics are thus governed by the differential equations
\begin{align}
    &\dv{E}{r} = 2 \frac{\rho}{m} L(E) &\text{(electron energy loss)} \, , \label{eq:full_electron_energy_loss} \\
    &\dv{q}{r} = \frac{q}{\lambda_\text{mfp}(E)} = \frac{\rho}{m} q \Lambda(E) &\text{(effective heat flux)}  \, , \label{eq:full_effective_heat_flux}
\end{align}
where the mean free path $\lambda_\text{mfp}$ is modelled through the effective energy flux cross-section $\Lambda(E) = \hat{\sigma}_T(E) + 2 L(E)/E$ with the empirical energy loss function
\begin{gather}
L(E>\qty{20}{\eV}) = \frac{\qty{8.62e-15}{\eV.\cm^2}}{\left(\frac{E/\unit{\eV}}{100}\right)^{0.823} \!\!\!\! +\left(\frac{E/\unit{\eV}}{60}\right)^{-0.125}  \!\!\!\! +\left(\frac{E/\unit{\eV}}{48}\right)^{-1.94}}
\label{eq:energy_loss_function}
\end{gather}
derived by \textcite{miles_electron-impact_1972} and the corresponding effective backscattering cross-section
\begin{equation}
\hat{\sigma}_T (E) = \begin{dcases}
    \left(\frac{\num{8.8e-13}}{(E/\unit{\eV})^{1.71}}-\frac{\num{1.62e-12}}{(E/\unit{\eV})^{1.932}} \right)\, \unit{\cm^2} & \text{for } E>\qty{100}{eV} \\
    \left(\frac{\num{1.1e-14}}{(E/\unit{\eV})} \right) \unit{\cm^2} & \text{for } E<\qty{100}{\eV}
\end{dcases}
\end{equation}
derived by \textcite{parks_model_1977-1} based on experimentally measured values by \textcite{maecker_ionen-_1955}.
These functions are visualized in \cref{fig:energy_attenuation}.

\begin{figure}
    \centering
    \includegraphics{figure/empirical_cross_sections.pdf}
    \caption{Empirical functions $\Lambda(E)$, $L(E)$ and $\hat{\sigma}_T(E)$ used in the NGS model by \textcite{parks_effect_1978} for the energy loss of electrons in a hydrogen gas.}
    \label{fig:energy_attenuation}
\end{figure}

The interdependence between the electron heat flux and the ablation cloud density at each point means that the full system of \cref{eq:ideal_gas_law,eq:full_mass_conservation,eq:full_momentum_conservation,eq:full_energy_conservation,eq:heat_flux_approximation,eq:full_effective_heat_flux,eq:full_electron_energy_loss} needs to be solved self-consistently.
For that, enough boundary conditions have to be motivated that describe the physical system.
The low sublimation energy of hydrogen is the reason that the ablation cloud establishes itself to nearly fully shield the pellet.
Therefore, the heat flux as well as the gas temperature at the pellet surface can be neglected and considered to be exactly zero.
The outer boundary of the neutral ablation cloud to the plasmoid ablation cloud establishes itself as a shock front, where the flow velocity rapidly decreases to adjust to the pressure in the plasmoid.
Considering that the back pressure through the ablation cloud and the heat conduction play no significant role in the ablation dynamics, the boundary of the neutral ablation cloud can be seen as infinitely far away from the pellet with negligible pressure.
Given the heat flux and average energy of the electrons reaching the ablation cloud, the boundary conditions can be summarized as
\begin{equation}
\begin{gathered}
    q(r_\text{p})=0, \quad T(r_\text{p})=0, \\
    p_\infty := p(r \rightarrow \infty) = 0, \quad
    q_\infty := q(r\rightarrow\infty) = q_\text{bc}, \quad E_\infty :=  E(r\rightarrow\infty) = E_\text{bc} \, .
    \label{eq:full_boundary_conditions}
\end{gathered}
\end{equation}
Where $q_\text{bc}(\theta,\varphi)$ and $E_\text{bc}(\theta,\varphi)$ depends on the amount of plasmoid shielding from the background plasma electrons at temperature $T_\text{bg}$ and density $n_\text{bg}$, as modelled in \cref{sec:plasmoid_shielding}.
Note that no assumptions have to be made on the velocity at which ablated molecules leave the pellet surface, but it is found to be negligible in the numerical solutions.
However, for our anisotropic perturbative model, we have to assume that the molecules leave the pellet surface only with a radial velocity and the angular velocity is assumed in our model to be zero.

% Finally, before going into the specifics of the isotropic model and the anisotropic perturbation, it is helpful to get a refresher in multivariable calculus, where the general derivatives in spherical coordinates, neglecting any $\varphi$-dependence for brevity, become
% \begin{align}
%     \mathrm{grad}f = \vec{\nabla}f =& \, \pdv{f}{r} \hat{r} + \frac{1}{r}\pdv{f}{\theta} \hat{\theta} \, , \\ 
%     \mathrm{div}\vec{A} = \vec{\nabla} \cdot \vec{A} =& \, \pdv{A_r}{r} + \frac{2}{r}A_r + \frac{1}{r}\pdv{A_\theta}{\theta} + \frac{\cos\theta}{r \sin\theta} A_\theta \, , \\
%     \begin{split}
%     (\vec{A}\cdot\vec{\nabla})\vec{B} =& 
%      \left(A_r \pdv{B_r}{r} + \frac{A_\theta}{r}\pdv{B_r}{\theta} + \frac{A_\theta B_\theta}{r} \right) \hat{r} \\
%      &+ \left( A_r \pdv{B_\theta}{r} + \frac{A_\theta}{r}\pdv{B_\theta}{\theta} + \frac{A_\theta B_r}{r} \right) \hat{\theta} \, .
%     \end{split}
% \end{align}
        
\subsection{Isotropic model}
\label{ssec:isotropic_model}

Consider now the case of full spherical symmetry, where each quantity only depends on the radial coordinate $r$ and the velocity is purely radial ($\vec{v}=v \hat{r}$).
Using this symmetry to evaluate and integrate the mass conservation \cref{eq:full_mass_conservation} leads to
\begin{equation}
    4 \pi r^2 \frac{\rho}{m} v = \textit{const} = G \, .
    \label{eq:ngs_mass_conservation}
\end{equation}
This radially constant quantity represents the total outflow of particles through any spherical shell and is thus denoted as particle ablation rate $G$.
Note that \textcite{parks_effect_1978} defined $G$ to be the mass ablation rate $G_\text{Parks} = m \cdot G$.
Similarly, using \cref{eq:ngs_mass_conservation} and \cref{eq:ideal_gas_law}, the momentum conservation \cref{eq:full_momentum_conservation} and the energy conservation \cref{eq:full_energy_conservation} become
\begin{gather}
    \rho v \pdv{v}{r} = - \pdv{p}{r} \quad \quad \text{and} \label{eq:ngs_momentum_conservation} \\
    \frac{G}{4 \pi r^2} \pdv{r}(\frac{m}{2}v^2 + \frac{\gamma}{\gamma - 1} T) = \mu \pdv{q}{r} \label{eq:ngs_energy_conservation}
\end{gather}
respectively. 
Together with the unchanged \cref{eq:full_electron_energy_loss,eq:full_effective_heat_flux} describing the incident electron dynamics and the boundary conditions in \cref{eq:full_boundary_conditions} the full radial solutions can be determined.

Since finding a fully analytical expression for the solutions is too demanding and potentially impossible, the equations have to be prepared for numerical integration.
It turns out to be useful to normalize all quantities to their value at the sonic radius $r_\star$, i.e. the radius at which the flow velocity $v$ transitions between subsonic to supersonic speeds, in the form
\begin{gather}
\begin{gathered}
    \widetilde{\rho}=\frac{\rho}{\rho_\star}, \quad \widetilde{p} = \frac{p}{p_\star}, \quad \widetilde{T} = \frac{T}{T_\star}, \quad \widetilde{v}=\frac{v}{v_\star}, \quad \widetilde{r}=\frac{r}{r_\star} \\
    \widetilde{q}=\frac{q}{q_\star}, \quad \widetilde{E}=\frac{E}{E_\star}, \quad \widetilde{\Lambda} = \frac{\Lambda}{\Lambda_\star}, \quad \widetilde{L} = \frac{L}{E_\star \Lambda_\star} , \quad \text{with} \quad \Lambda_\star=\Lambda(E_\star).
\end{gathered}
\label{eq:ngs_normalization}
\end{gather}
In this thesis, all quantities denoted with a $\star$ represent their physical values at the sonic radius.
Here, variables denoted with a \textasciitilde{}  represent their normalized, dimensionless version.
However, the \textasciitilde{} is omitted in the following to prevent visual clutter and all quantities can be considered normalized, if not stated otherwise.

Substituting the physical quantities by the normalized quantities in the ideal gas law \cref{eq:ideal_gas_law} and the mass conservation \cref{eq:ngs_mass_conservation} gives
\begin{gather}
    \rho = \frac{p}{T}\, \quad \text{and} \\
    r^2\frac{p}{T}v = 1 \, .
\end{gather}
The normalized set of differential equations is then obtained after some lengthy but straightforward algebra by additionally using the definition of the speed of sound in an ideal gas
\begin{equation}
    v_\star = \sqrt{\frac{\gamma T_\star}{m}} \, . \label{eq:speed_of_sound}
\end{equation}
%and evaluating \cref{eq:ideal_gas_law} and \cref{eq:ngs_mass_conservation} at the sonic radius.
Meaning, the \cref{eq:ngs_momentum_conservation,eq:ngs_energy_conservation,eq:full_electron_energy_loss,eq:full_effective_heat_flux} can be rewritten in the normalized form
\begin{align}
    &\pdv{v^2}{r} = \frac{4v^2T}{(T-v^2)r}\left(\frac{q\Lambda r}{Tv} -1\right) \, , \label{eq:ngs_normalized_dv2}\\
    &\pdv{T}{r} = \frac{2\Lambda q}{v}-\frac{1}{2}(\gamma -1)\pdv{v^2}{r} \, , \label{eq:ngs_normalized_dT}\\
    &\pdv{E}{r} = 2\lambda _\star\frac{L}{r^2v}\, ,\label{eq:ngs_normalized_dE}\\
    &\pdv{q}{r} = \lambda _\star\frac{q\Lambda }{vr^2} \,.\label{eq:ngs_normalized_dq}
\end{align}
No additional approximations were made in this derivation and all normalization constants were combined into a single dimensionless quantity
\begin{equation}
    \lambda_\star = r_\star \Lambda_\star \frac{p_\star}{T_\star} \, .
    \label{eq:lambda_star_definition}
\end{equation}
Since, by design, all normalized quantities (except $L$) are known to be 1 at the sonic radius $r=1$, the full radial dependence of $v(r)$, $T(r)$, $E(r)$ and $q(r)$ is set by providing $\gamma$, $\lambda_\star$ and $E_\star$ (hidden in $\Lambda$ and $L$).
While $\gamma$ is a material dependent parameter, $\lambda_\star$ and $E_\star$ will be shown to be functions only of $E_\text{bc}$ for the chosen boundary conditions in \cref{eq:full_boundary_conditions}.

Considering $T(1)=v(1)=1$ in \cref{eq:ngs_normalized_dv2} could lead to believe a singularity $1/(T-v^2)$ appeared.
However, requiring that $\partial{v^2}/\partial{r}$ is finite at the sonic radius allowed to derive the additional constraint on the normalization constants
\begin{equation}
    \frac{4 \pi r_\star^2}{G} \frac{(\gamma - 1)}{\gamma} \frac{\lambda_\star \mu q_\star}{T_\star} = 2 \, ,\label{eq:ngs_singularity_constraint}
\end{equation}
which was used in deriving \cref{eq:ngs_normalized_dv2} and \cref{eq:ngs_normalized_dT}.
Using L'Hôpital's rule enables evaluation of the apparent singularity at the sonic radius, yielding
\begin{equation}
    \chi_\star := \eval{\pdv{v^2}{r}}_{r = 1} = \frac{2}{\gamma+1} \left((3 - \gamma) + \sqrt{(3-\gamma)^2 - 2(\lambda_\star + \Psi_\star - 1)(\gamma + 1)} \right) \, ,
\end{equation}
with an additional shorthand defined as $\Psi_\star := 2 \lambda_\star \left( L \pdv{\Lambda}{E} \right)_{E=1}$.
Now everything is known to start finding numerical solutions for both the normalized and consequently the physical quantities.

\subsubsection{Numerical solution}

The goal of the procedure described below is to find numerical solutions for $v(r)$, $T(r)$, $E(r)$ and $q(r)$, which satisfy both the differential \cref{eq:ngs_normalized_dv2,eq:ngs_normalized_dT,eq:ngs_normalized_dE,eq:ngs_normalized_dq} and the boundary conditions in \cref{eq:full_boundary_conditions}. 
For this purpose, for some chosen $\gamma$ and $E_\star$, we optimize $\lambda_\star$ so that $T(r_p) = q(r_p) = 0$ at some radius $r_p<1$, which is then interpreted as the normalized pellet radius. 
Together with the values $q_\infty$ and $q_\infty$ all normalization constants are determined, and the full physical solution can be calculated. 
As a first guess, it suffices to set $E_\star \approx E_\text{bc}$, since most of the energy is absorbed inside the sonic radius.
However, a scan over $E_\star$ allows setting it more precisely.

Many different methods exist to solve a set of first order differential equations numerically.
Knowing that 
\begin{equation*}
    v(r=1)=1,\quad T(r=1)=1,\quad E(r=1)=1,\quad q(r=1)=1,
\end{equation*}
any method would work which solves initial value problems.
The method of numerical integration found to work well here is an \enquote{implicit multi-step variable-order (1 to 5) method based on a backward differentiation formula for the derivative approximation}\footnote{see \url{https://docs.scipy.org/doc/scipy/reference/generated/scipy.integrate.solve_ivp.html}} based on the implementation by \textcite{shampine_matlab_1997}.
The solutions are calculated from $r=1$ both downwards and upwards.
Optimization of $\lambda_\star$ is done through minimizing the error of the boundary conditions $T(r_p) = q(r_p) = 0$.
Here, \enquote{a modification of the Powell hybrid method as implemented in MINPACK}\footnote{see \url{https://docs.scipy.org/doc/scipy/reference/generated/scipy.optimize.root.html}} is used \autocite{more_user_1980}.

\begin{figure}
    \centering
    \includegraphics{figure/ode0_solution_Estar_3.000e+04_gamma_7_5.pdf}
    \caption{Caption (I have to write the caption and make the figure look nicer!)}
    \label{fig:example_0th_order_solution}
\end{figure}

An example solution for $\gamma = 7/5$ and $E_\star = \qty{30}{\keV}$ is shown in \cref{fig:example_0th_order_solution}.
This same solution is shown by \textcite{parks_effect_1978} in their fig. 3 and 4, which serves as a validation of the correctness of our numerical solutions.
Visible is that $E(r\rightarrow\infty) \approx 1$, which justifies $E_\star \approx E_\text{bc}$.

Another validation is performed by scanning over the parameters $\gamma$ and $E_\text{bc}$ and plotting the calculated results for $\lambda_\star$, $r_\text{p}$, $E_\infty$ and $q_\infty$, as shown in \cref{fig:ode0_scan}.
Again, the same graphic is found in the paper by \textcite{parks_effect_1978} as their fig. 1 and 2.
Additionally, scaling laws for $\gamma = 5/3$ (dotted lines) are given in the paper, which agree well with our results.
It is important to note that in \cref{fig:ode0_scan} the energy dependence is shown on a logarithmic x-axis, while the variance on the y-axis is small and linear.
Thus, the dependence on $E_\text{bc}$ is very weak, and the quantities can be considered nearly constant.
The scaling laws representing our solution (dashed lines) are given in \cref{chap:appendix_scaling_laws}.
(Do I want to show how well the boundary conditions are met? And that the limitation $E < \qty{20}{\eV}$ might matter a bit? Also, I might want to show $p_0(r_p)$)

\begin{figure}
    \centering
    \includegraphics[width=0.8\textwidth]{figure/recreated_Parks_E_inf.pdf}
    \caption{Caption (I have to write the caption and make the figure look nicer!)}
    \label{fig:ode0_scan}
\end{figure}

The full solution of the normalized spherically symmetric quantities is enough for modelling the normalized asymmetric perturbation, as described in the next section.
However, to calculate any physical quantities, the normalization constants in \cref{eq:ngs_normalization} need to be known.
Given the physical boundary conditions, i.e. model input parameters, $r_\text{p}$, $E_\text{bc}$ and $q_\text{bc}$, the results as presented in \cref{fig:ode0_scan}, directly enable calculating
\begin{align}
    r_\star = \frac{r_p}{\widetilde{r}_p}, \quad 
    E_\star = \frac{E_\text{bc}}{\widetilde{E}_\infty}, \quad \text{and} \quad
    q_\star = \frac{q_\text{bc}}{\widetilde{q}_\infty}\,.
\end{align}
Apart from $\rho_\star = m p_\star/T_\star$, the still unknown normalization constants are $p_\star$, $T_\star$ and $v_\star$.
These quantities, together with the particle ablation rate $G$, are determined by solving the system of \cref{eq:ngs_mass_conservation,eq:speed_of_sound,eq:ngs_singularity_constraint,eq:lambda_star_definition}, yielding
\begin{align}
    \Aboxed{p_\star &= f_p(E_\text{bc}, \gamma) \cdot \left[ \frac{m (\mu q_\text{bc})^2}{\Lambda_\star r_\text{p}} \right]^\frac{1}{3}} \, , \label{eq:prefactor_p_star} \\
    T_\star &= f_T(E_\text{bc}, \gamma) \cdot \left[ m (\Lambda_\star \mu q_\text{bc} r_\text{p})^2 \right]^\frac{1}{3} \, , \label{eq:prefactor_T_star} \\
    v_\star &= f_v(E_\text{bc}, \gamma) \cdot \left[ \frac{\Lambda_\star \mu q_\text{bc} r_\text{p}}{m} \right]^\frac{1}{3} \, , \label{eq:prefactor_v_star} \\
    G &= f_G(E_\text{bc}, \gamma) \cdot \left[ \frac{\mu q_\text{bc} r_\text{p}^4}{\Lambda_\star^2 m} \right]^\frac{1}{3} \, . \label{eq:prefactor_G}
\end{align}
The prefactors $f_p$, $f_T$, $f_v$ and $f_G$ combine all dimensionless factors and are shown in \cref{fig:prefactors_0th_order} to only weakly depend on $E_\text{bc}$.
Scaling laws are again provided in \cref{chap:appendix_scaling_laws}.
Since the pellet rocket effect mainly depends on the pressure asymmetry at the pellet surface, the most important formula here is \cref{eq:prefactor_p_star}, where $f_p \approx 0.15$ and as stated earlier $\mu \approx 0.65$.

\begin{figure}
    \centering
    \includegraphics[width=0.8\textwidth]{figure/prefactors_0th_order.pdf}
    \caption{Caption (I have to write the caption and make the figure look nicer!)}
    \label{fig:prefactors_0th_order}
\end{figure}

(Now all equations are set up. Now is a good time to summarize. I also need to describe how the sonic quantities can be derived. And then I need to describe how the numerical solutions are found)


        \subsection{Asymmetric perturbation model}
        \label{ssec:asymmetric_perturbation_model}
            \subsubsection{Analytical description}
            \subsubsection{Numerical solution}
    \section{Plasmoid shielding}
    \label{sec:plasmoid_shielding}
        \subsection{Drift induced shielding length asymmetry}
        \label{ssec:plasmoid_shielding_length}
        \subsection{Heat flux attenuation}
        \label{ssec:plasmoid_heat_flux_attenuation}
    \section{Model summary (putting it all together...)}
    \label{sec:model_summary}