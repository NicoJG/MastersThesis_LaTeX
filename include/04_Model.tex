\chapter{Modelling the pellet rocket effect}

The underlying principle of the pellet rocket effect is that asymmetric heating of the pellet and the ablation cloud leads to a higher ablation and pressure on one side of the pellet. 
This results in a rocket-like force on the pellet, which accelerates it towards the less heated side.
In general, this phenomenon involves complex, three-dimensional, non-linear fluid dynamics.
Fortunately, in the case of pellet injection in magnetic confinement fusion, the pellet heating can be seen as predominantly spherically symmetric.
Taking the anisotropic pellet ablation dynamics as a linear perturbation on top of the isotropic dynamics, enables us to develop a semi-analytical model for the pellet rocket effect.

Conceptually, this model is divided in three different parts.
The first part, as described in \cref{sec:pellet_surface}, considers the momentum transfer on the pellet surface to calculate the net force on the pellet.
Expanding the angular dependence of the gas properties at the pellet surface in spherical harmonics, yields a formula for the pellet rocket force, involving only the $\cos\theta$-dependence of the ablation rate and pressure at the pellet surface.

The second part of the model, as described in \cref{sec:neutral_gas_shielding}, quantifies the shielding of the pellet from an external asymmetric heating source through the neutral gas ablation cloud.
Widely used for predicting isotropic pellet ablation dynamics is a model known as the Neutral Gas Shielding (NGS) model, developed by \textcite{parks_effect_1978}.
There, quasi-steady-state fluid equations for an ideal gas are solved under the approximation of a mono-energetic beam of incoming electrons, which traverse the ablation cloud radially.
This model is in good agreement with experimental data and while several more sophisticated models were developed, most models yield the same results \autocite{pegourie_review_2007}.
Therefore, the NGS model is the basis of our model.
The asymmetric ablation dynamics are then added as a perturbation.
Since the full radial solution of the NGS model is needed to calculate the asymmetric dynamics, the NGS model has to be fully reproduced, as described in \cref{ssec:isotropic_model}.
The asymmetric perturbation dynamics are then modelled, as described in \cref{ssec:asymmetric_perturbation_model} given the external heating source.

While this model can be used by itself, if the heat source into the neutral ablation cloud is known, our goal is to predict the pellet rocket acceleration along the major radius of a tokamak from the background plasma parameters.
Therefore, the third part of our model, as described in \cref{sec:plasmoid_shielding}, calculates the heat source asymmetry induced by the outward drift of the plasmoid of ablated material.
How this drift results in a finite and varying shielding length of the electrons through the plasmoid is described in \cref{ssec:plasmoid_shielding_length}.
In \cref{ssec:plasmoid_heat_flux_attenuation}, the shielding length is used to estimate the heat source at the boundary of the neutral gas cloud.

Finally, in \cref{sec:model_summary}, all three parts are combined to describe the full model.

%%%%%%%%%%%%%%%%%%%%%%%%%%%%%%%%%%%%%%%%%%%%%%%%%%%%%%%%%%%%%%%%%%%%%%%%%%%%%%%%%%%%%%%%%
\section{Force on the pellet surface}
\label{sec:pellet_surface}

Consider a spherical pellet of radius $r_\text{p}$ surrounded by a neutral gas.
Physically, the force on the pellet arises from the combination of ablated particles leaving the pellet surface and the gas pressure pushing on the pellet surface.
In the following, a formula for this force is derived under the assumption that the anisotropic dynamics are small compared to the spherically symmetric dynamics.

Mathematically, the momentum transfer at a point $\vec{r}$ in the gas is expressed through the momentum flux tensor
\begin{equation*}
    \tensor{\Pi} = \rho \vec{v} \vec{v} + p \tensor{\mathbbone} \, ,
\end{equation*}
with the mass density $\rho(\vec{r})$, the fluid velocity $\vec{v}(\vec{r})$ and the pressure $p(\vec{r})$. 
The notation $\vec{v}\vec{v}$ represents the dyadic product\footnote{Equivalent to a tensor product.}, and $\tensor{\mathbbone}$ denotes the 3-dimensional identity tensor.  
Assuming local momentum conservation at the pellet surface $S$, the net force on the pellet is
\begin{equation*}
    \vec{F} = - \iint_{S} \tensor{\Pi} \cdot \dd{\vec{S}} \, ,
\end{equation*}
where the minus sign indicates that this is the force exerted on the pellet, while the surface element $\dd{\vec{S}}$ points outwards.
\begin{figure}
    \centering
    \includegraphics{figure/pellet_surface_force.pdf}
    \caption{Illustration of how the pellet rocket force arises from both an asymmetry in pressure on the pellet surface (visualized by the red background) and an asymmetric ablation (visualized by the vectors). Additionally, the coordinate system used throughout this thesis is indicated. The unit vector $\hat{z}$ denotes the axis of asymmetry. $\hat{r}$ and $\hat{\theta}$ denote the spherical coordinates, in which $\hat{\varphi}$ would point into the paper. The pellet is modelled as a solid sphere of radius $r_\text{p}$.}
    \label{fig:pellet_surface_force}
\end{figure}
We choose spherical coordinates $\{r,\theta,\varphi\}$ so that the pellet is centred at the origin and a positive force $F$ points in the negative $\hat{z}$-direction, as shown in \cref{fig:pellet_surface_force}. 
The force then becomes
\begin{equation*}
    F = - \hat{z} \cdot \vec{F} = r_\text{p}^2 \iint_{S} \left[ \rho \left(\hat{z}\cdot  \vec{v} \vec{v}  \cdot \hat{r}\right) + p \left(\hat{z}\cdot\hat{r}\right) \right] \,\dd{\Omega} \, ,
\end{equation*}
with the differential solid angle $\dd{\Omega}=\sin\theta \dd{\theta}\dd{\varphi}$. 
Expressing $\vec{v} = v_r \hat{r} + v_\theta \hat{\theta} + v_\varphi \hat{\varphi}$ and using geometric relations between the unit vectors leads to
\begin{equation*}
    F = r_\text{p}^2 \iint_{S} \left[ \rho v_r (v_r \cos\theta + v_\theta \sin\theta) + p \cos\theta \right] \,\dd{\Omega} \, .
\end{equation*}
At this point, the first and only approximation for deriving the force formula has to be made.
The anisotropic dynamics are taken as a small perturbation on the spherically symmetric dynamics in the form
\begin{align*}
    v_r(\vec{r}) &= v_0(r) + \var{v_r}(r,\theta,\varphi), &
    v_\theta(\vec{r}) &= 0 + \var{v_\theta}(r,\theta,\varphi), \\
    \rho(\vec{r}) &= \rho_0(r) + \var{\rho}(r,\theta,\varphi), &
    p(\vec{r}) &= p_0(r) + \var{p}(r,\theta,\varphi) \, .
\end{align*}
Linearizing the force in this perturbation and using $\int_0^\pi \cos\theta \dd{\theta} = 0$ leads to
\begin{equation}
    F = r_\text{p}^2 \int_{\theta=0}^{\pi} \int_{\varphi=0}^{2\pi} \left(\left( \var{\rho} v_0^2 + 2 \rho_0 v_0 \var{v_r} + \var{p} \right) \cos\theta + \rho_0 v_0 \var{v_\theta} \sin\theta \right) \, \sin\theta \dd{\theta}\dd{\varphi} \, .
    \label{eq:force_before_expansion}
\end{equation}
The last term can be rewritten as a term proportional to $\cos\theta$ through integration by parts 
\begin{equation*}
    \int_0^\pi \var{v_\theta} \sin^2\theta \dd{\theta} = \underbrace{\left[ \left(\int \var{v_\theta} \dd{\theta}\right) \sin^2\theta \right]_0^\pi}_{=0} - \int_0^\pi \left(\int \var{v_\theta} \dd{\theta}\right) 2\cos\theta\sin\theta \dd{\theta} \, ,
\end{equation*}
assuming that $\left(\int \var{v_\theta} \dd{\theta}\right)$ is finite.
The surface integral can then be solved by expanding $\var{\rho}$, $\var{v_r}$, $\var{p}$ and $\left(\int \var{v_\theta} \dd{\theta}\right)$  in terms of spherical harmonics 
\begin{equation*}
    Y_l^m(\theta,\varphi) = \sqrt{\frac{(l-m)!}{(l+m)!}} \mathcal{P}_l^m(\cos\theta) e^{im\varphi} \, ,
\end{equation*}
with the associated Legendre polynomials $\mathcal{P}_l^m$.
Spherical harmonics are orthogonal in the sense
\begin{equation*}
    \int_{\theta=0}^{\pi} \int_{\varphi=0}^{2\pi} Y_l^m (Y_{l'}^{m'})^* \sin\theta \dd{\theta}\dd{\varphi} = 
    \begin{cases}
        \frac{4 \pi}{2l+1} & \text{for }l=l' \, , \, m=m' \\
        0 & \text{otherwise}
    \end{cases} \, ,
\end{equation*}
which is helpful because ${Y_1^0=\cos\theta}$ appears in front of every term in the surface integral \cref{eq:force_before_expansion}.
Therefore, for any general variations $\var{\rho}$, $\var{v_r}$, $\var{p}$ and $\int \var{v_\theta} \dd{\theta}$, only their projection onto $\cos\theta$, i.e. the coefficient of the $l=1$, $m=0$ mode, contributes to the force.
Note that for $\var{v_\theta}$ the projection onto $-\sin\theta$ is relevant, since only the integral $\int \var{v_\theta} \dd{\theta}$ is expanded in terms of spherical harmonics.
Inserting the relevant expansions
\begin{align*}
    \var{\rho}(r,\theta,\varphi) &= \rho_1(r)\cos\theta + \dots \, , & 
    \var{p}(r,\theta,\varphi) &= p_1(r)\cos\theta + \dots \, , \\
    \var{v_r}(r,\theta,\varphi) &= v_{1,r}(r)\cos\theta + \dots \, , &
    \var{v_\theta}(r,\theta,\varphi) &= - v_{1,\theta}(r)\sin\theta + \dots
\end{align*}
into \cref{eq:force_before_expansion} yields the formula for the pellet rocket force
\begin{equation}
    \boxed{F = \frac{4 \pi r_\text{p}^2}{3} \left( \rho_1 v_0^2 + 2 \rho_0 v_0 (v_{1,r} - v_{1,\theta}) + p_1 \right)_{r=r_\text{p}}} \, .
    \label{eq:rocket_force_full}
\end{equation}
This formula can also be understood physically.
The ablation rate per unit area $g$, i.e. the mass flux through the pellet surface, is
\begin{equation*}
    g(\theta) = \rho \vec{v} \cdot \hat{r} \approx \left( \rho_0 v_0 + (\rho_1 v_0 + \rho_0 v_{1,r}) \cos\theta + \dots \right)_{r=r_\text{p}} = g_0 + g_1 \cos\theta + \dots \, .
\end{equation*}
Therefore, the first two terms in \cref{eq:rocket_force_full} describe the force arising from asymmetric ablation.
The term $\rho_0 v_0 v_{1,\theta}$ describes a force from mass flowing around the pellet surface, and the last term $p_1$ describes the gas pressure asymmetry. 
All of this is integrated over the pellet surface area $4 \pi r_\text{p}^2$.

Under the self-regulating shielding assumptions, used in the next parts of this model, the pellet rocket force is predominantly caused by the pressure asymmetry in the neutral gas ablation cloud.
Therefore, the pellet rocket force is
\begin{equation}
    F = \frac{4 \pi r_\text{p}^2}{3} p_1(r_\text{p}) \, ,
\end{equation}
which is essentially the same formula as used for the empirical model developed by \textcite{szepesi_radial_2007}.
The main challenge of this thesis is to develop a model for the pressure asymmetry at the pellet surface $p_1(r_\text{p})$ given an external heating source.
This is the subject of the following sections, by developing the asymmetric NGS model.

\section{Neutral gas shielding (NGS)}
\label{sec:neutral_gas_shielding}

As noted in the introduction to \cref{chap:model}, the basis of our model is the isotropic NGS model, developed by \textcite{parks_effect_1978}.
Most of this section is dedicated to describing and reproducing this semi-analytical model.
The underlying physics processes related to ablation of hydrogen pellets were described already in \cref{sec:ablation_dynamics} and will not be repeated here.
The following text describes first the general ideas and approximations of the NGS model, which are used for both the isotropic and the asymmetric model. 
Then the mathematical details of how solutions are found in the isotropic case are presented in \cref{ssec:isotropic_model}.
Finally, \cref{ssec:asymmetric_perturbation_model} describes the perturbative extension in the form of our asymmetric NGS model.

Since the neutral ablation cloud can be considered a transonic ideal gas, the equation of state is
\begin{align}
    &\frac{\rho}{m} = \frac{p}{T}  &\text{(ideal gas law)} \, , \label{eq:ideal_gas_law}
%\end{align}
\intertext{%
with mass density $\rho$, pressure $p$, temperature\footnote{Temperatures are assumed to be in units of energy by including Boltzmann's constant.} $T$ and mass $m$ of one molecule (or atom) in the gas.
The full gas dynamics are obtained by considering the steady state conservation laws
}
%\begin{align}
&\vec{\nabla} \cdot (\rho \vec{v}) = 0 &\text{(mass conservation)} \, , \label{eq:full_mass_conservation} \\
&\rho (\vec{v} \cdot \vec{\nabla}) \vec{v} = - \vec{\nabla} p &\text{(momentum conservation)} \, , \label{eq:full_momentum_conservation} \\
&\vec{\nabla} \cdot \left[\left( \frac{\rho v^2}{2} + \frac{\gamma p}{\gamma - 1} \right) \vec{v}\right] = Q &\text{(energy conservation)} \, ,\label{eq:full_energy_conservation}
\end{align}
also known as Euler equations, with flow velocity $\vec{v}$, external heat source $Q$ and the adiabatic index\footnote{Also called heat capacity ratio.} of the gas $\gamma$.
In an ideal gas, the adiabatic index is related to the number of degrees of freedom $f$ of the gas particles through $\gamma = 1+2/f$, so it can be assumed to be $\gamma = 5/3$ for a monatomic gas, $\gamma = 7/5$ for a diatomic gas and $\gamma = 9/7$ for a linear triatomic gas.
In the case of frozen deuterium pellets, the temperature close to the pellet is so low that dissociation can be neglected, and the whole neutral ablation cloud is approximated as a \ce{D2} gas.

\begin{figure}
    \centering
    \includegraphics{figure/neutral_cloud_heating.pdf}
    \caption{Illustration of the asymmetric heating of the neutral gas ablation cloud by incoming electrons travelling along the magnetic field lines. Most of the heating occurs inside the sonic radius $r_\star$, close to the pellet radius $r_\text{p}$. The boundary between the neutral ablation cloud and the ionized ablation cloud is comparatively far away at $r_\text{i}$. The electrons are modelled to lose their energy radially inwards, while in reality they traverse the neutral ablation cloud on parallel straight lines from two sides.}
    \label{fig:neutral_cloud_heating}
\end{figure}

The external heat source $Q$ at each point in the gas is subject to the major approximations of the NGS model.
Heat conduction in the neutral cloud is neglected, since this occurs only on a much slower timescale than the heating from highly energetic electrons losing energy in the cloud through elastic scattering and inelastic processes \autocite{parks_effect_1978}.
While, in reality, the incoming electrons are constrained by the magnetic field to traverse the cloud on straight parallel lines, the NGS model approximates this path with an equivalent radial path, as illustrated in \cref{fig:neutral_cloud_heating}.
Therefore, the heat source
\begin{equation}
    Q = - \mu \vec{\nabla}\cdot \vec{q} = \mu \pdv{q}{x} \approx \mu \pdv{q}{r}
    \label{eq:heat_flux_approximation}
\end{equation}
is modelled as if the electrons lose their energy radially inwards instead of along the field lines in the direction $x$. 
This approximation is motivated by the fact that the energy flux reaching a spherical shell depends on the integrated density along the electron trajectory, which is geometrically similar for both the radial and the parallel paths \autocite{parks_model_1977-1}.
Considering the success of the isotropic NGS model, this approximation is retained in our model.
Only a fraction $\mu$ of the electron energy loss goes into heating the gas, while the rest goes mainly into Bremsstrahlung radiation and backscattered electrons.
\textcite{parks_effect_1978} state that the fraction $\mu$ is fairly well modelled to be between $60 \%$ and $70\%$ at all points in the cloud in the case of a hydrogenic gas.
Additionally, the incident electrons are approximated to all have the same energy once they reach the point $\vec{r}$ in the neutral ablation cloud.
These strong approximations of a mono-energetic beam of incoming electrons losing energy radially have been found to be sufficiently accurate for predicting pellet ablation rates in magnetic confinement fusion.
How those approximations degrade the accuracy of predictions of the pellet rocket effect has to be evaluated by comparison to experiments or numerical simulations.

Modelling the exact dynamics of how the electrons lose energy on their path through the gas is a non-trivial task.
Therefore, the NGS model uses empirical scaling laws for the scattering cross-sections of electrons in a hydrogen gas.
The electron dynamics are thus governed by the differential equations for the energy $E$ and heat flux $q$
\begin{align}
    &\dv{E}{r} = 2 \frac{\rho}{m} L(E) &\text{(electron energy loss)} \, , \label{eq:full_electron_energy_loss} \\
    &\dv{q}{r} = \frac{q}{\lambda_\text{mfp}(E)} = \frac{\rho}{m} q \Lambda(E) &\text{(effective heat flux)}  \, , \label{eq:full_effective_heat_flux}
\end{align}
where the mean free path $\lambda_\text{mfp}$ is modelled through the effective energy flux cross-section $\Lambda(E) = \hat{\sigma}_T(E) + 2 L(E)/E$, with the empirical energy loss function
\begin{gather}
L(E>\qty{20}{\eV}) = \frac{\qty{8.62e-15}{\eV.\cm^2}}{\left(\frac{E/\unit{\eV}}{100}\right)^{0.823} \!\!\!\! +\left(\frac{E/\unit{\eV}}{60}\right)^{-0.125}  \!\!\!\! +\left(\frac{E/\unit{\eV}}{48}\right)^{-1.94}} \, ,
\label{eq:energy_loss_function}
\end{gather}
derived by \textcite{miles_electron-impact_1972}. The corresponding effective backscattering cross-section
\begin{equation}
\hat{\sigma}_T (E) = \begin{dcases}
    \left(\frac{\num{8.8e-13}}{(E/\unit{\eV})^{1.71}}-\frac{\num{1.62e-12}}{(E/\unit{\eV})^{1.932}} \right)\, \unit{\cm^2} & \text{for } E>\qty{100}{eV} \\
    \left(\frac{\num{1.1e-14}}{(E/\unit{\eV})} \right) \unit{\cm^2} & \text{for } E<\qty{100}{\eV}
\end{dcases}
\end{equation}
was derived by \textcite{parks_model_1977-1} based on experimentally measured values by \textcite{maecker_ionen-_1955}.
These functions are visualized in \cref{fig:energy_attenuation}.

\begin{figure}
    \centering
    \includegraphics{figure/empirical_cross_sections.pdf}
    \caption{Empirical functions $\Lambda(E)$, $L(E)$ and $\hat{\sigma}_T(E)$ used in the NGS model by \textcite{parks_effect_1978} for the energy loss of electrons in a hydrogen gas.}
    \label{fig:energy_attenuation}
\end{figure}

The interdependence between the electron heat flux and the ablation cloud density at each point means that the full system of \cref{eq:ideal_gas_law,eq:full_mass_conservation,eq:full_momentum_conservation,eq:full_energy_conservation,eq:heat_flux_approximation,eq:full_effective_heat_flux,eq:full_electron_energy_loss} needs to be solved self-consistently.
For that, the correct number of boundary conditions have to be motivated that describe the physical system.
The low sublimation energy of hydrogen is the reason that the ablation cloud establishes itself to nearly fully shield the pellet.
Therefore, the heat flux as well as the gas temperature at the pellet surface can be neglected and taken to be exactly zero.
%The outer boundary of the neutral ablation cloud, where it meets the plasmoid ablation cloud, establishes itself as a shock front, where the flow velocity rapidly decreases to adjust to the pressure in the plasmoid.
Considering that the back pressure from the fusion plasma and the heat conduction play no significant role in the ablation dynamics, the boundary of the neutral ablation cloud can be seen as infinitely far away ($r_i \rightarrow \infty$) from the pellet with negligible pressure.
Given the heat flux and average energy of the electrons reaching the neutral ablation cloud, the boundary conditions can be summarized as
\begin{equation}
\begin{gathered}
    q(r_\text{p})=0, \quad T(r_\text{p})=0, \\
    p(r \rightarrow \infty) = 0, \quad
    q(r\rightarrow\infty) = q_\text{bc}, \quad E(r\rightarrow\infty) = E_\text{bc} \, .
    \label{eq:full_boundary_conditions}
\end{gathered}
\end{equation}
The heating parameters $q_\text{bc}(\theta,\varphi)$ and $E_\text{bc}(\theta,\varphi)$ depend on the plasmoid shielding of the background plasma electrons at temperature $T_\text{bg}$ and density $n_\text{bg}$, as modelled further down in \cref{sec:plasmoid_shielding}.
Note that no assumptions have to be made on the velocity at which ablated molecules leave the pellet surface, but it is found to be negligible in the numerical solutions under these boundary conditions.
However, for our asymmetric NGS model, we have to additionally assume the angular flow velocity at the pellet surface to be zero.

%%%%%%%%%%%%%%%%%%%%%%%%%%%%%%%%%%%%%%%%%%%%%%%%%%%%%%%%%%%%%%%%%%%%%%%%%%%
\subsection{Isotropic NGS model}
\label{ssec:isotropic_model}

Consider now the case of full spherical symmetry, where each quantity only depends on the radial coordinate $r$ and the velocity is purely radial ($\vec{v}=v \hat{r}$).
Using this symmetry to evaluate and integrate the mass conservation \cref{eq:full_mass_conservation} leads to
\begin{equation}
    4 \pi r^2 \frac{\rho}{m} v = \textit{const} = G \, .
    \label{eq:ngs_mass_conservation}
\end{equation}
This radially constant quantity represents the total outflow of particles through any spherical shell and is equal to the particle ablation rate\footnote{Note that \textcite{parks_effect_1978} defined $G$ to be the mass ablation rate $G_\text{Parks} = m \cdot G$.} $G$.
Similarly, using \cref{eq:ngs_mass_conservation} and the ideal gas law \cref{eq:ideal_gas_law}, the momentum conservation \cref{eq:full_momentum_conservation} and the energy conservation \cref{eq:full_energy_conservation} become
\begin{gather}
    \rho v \pdv{v}{r} = - \pdv{p}{r} \quad \quad \text{and} \label{eq:ngs_momentum_conservation} \\
    \frac{G}{4 \pi r^2} \pdv{r}(\frac{m}{2}v^2 + \frac{\gamma}{\gamma - 1} T) = \mu \pdv{q}{r} \label{eq:ngs_energy_conservation}
\end{gather}
respectively. 
Together with \cref{eq:full_electron_energy_loss,eq:full_effective_heat_flux} describing the incident electron dynamics and the boundary conditions in \cref{eq:full_boundary_conditions}, the full radial solutions can be determined.

A fully analytical expression for the solution is not tractable, so the equations were prepared for numerical integration.
It turns out to be useful to normalize all quantities to their values at the sonic radius $r_\star$, i.e. the radius at which the flow velocity $v$ transitions between subsonic to supersonic speeds, in the form
\begin{gather}
\begin{gathered}
    \widetilde{\rho}=\frac{\rho}{\rho_\star}, \quad \widetilde{p} = \frac{p}{p_\star}, \quad \widetilde{T} = \frac{T}{T_\star}, \quad \widetilde{v}=\frac{v}{v_\star}, \quad \widetilde{r}=\frac{r}{r_\star} \\
    \widetilde{q}=\frac{q}{q_\star}, \quad \widetilde{E}=\frac{E}{E_\star}, \quad \widetilde{\Lambda} = \frac{\Lambda}{\Lambda_\star}, \quad \widetilde{L} = \frac{L}{E_\star \Lambda_\star} , \quad \text{with} \quad \Lambda_\star=\Lambda(E_\star).
\end{gathered}
\label{eq:ngs_normalization}
\end{gather}
In this thesis, all quantities denoted with a $\star$ represent their physical values at the sonic radius.
Here, variables with a tilde  represent their normalized, dimensionless version.
However, the tildes will be omitted in the following to prevent visual clutter and all quantities can be considered normalized, if not stated otherwise.

Substituting the physical quantities by the normalized quantities in the ideal gas law \cref{eq:ideal_gas_law} and the mass conservation \cref{eq:ngs_mass_conservation} gives
\begin{align}
    \rho &= \frac{p}{T}\, \quad \text{and} \\
    r^2\frac{p}{T}v &= 1 \, .
\end{align}
The normalized set of differential equations is then obtained after some lengthy but straightforward algebra by additionally using the definition of the speed of sound in an ideal gas
\begin{equation}
    v_\star = \sqrt{\frac{\gamma T_\star}{m}} \, . \label{eq:speed_of_sound}
\end{equation}
%and evaluating \cref{eq:ideal_gas_law} and \cref{eq:ngs_mass_conservation} at the sonic radius.
The system of \cref{eq:ngs_momentum_conservation,eq:ngs_energy_conservation,eq:full_electron_energy_loss,eq:full_effective_heat_flux} can thus be rewritten in the normalized form
\begin{align}
    &\pdv{v^2}{r} = \frac{4v^2T}{(T-v^2)r}\left(\frac{q\Lambda r}{Tv} -1\right) \, , \label{eq:ngs_normalized_dv2}\\
    &\pdv{T}{r} = \frac{2\Lambda q}{v}-\frac{1}{2}(\gamma -1)\pdv{v^2}{r} \, , \label{eq:ngs_normalized_dT}\\
    &\pdv{E}{r} = 2\lambda _\star\frac{L}{r^2v}\, ,\label{eq:ngs_normalized_dE}\\
    &\pdv{q}{r} = \lambda _\star\frac{q\Lambda }{vr^2} \,.\label{eq:ngs_normalized_dq}
\end{align}
No approximations were made in this derivation and all normalization constants were combined into a single dimensionless quantity
\begin{equation}
    \lambda_\star = r_\star \Lambda_\star \frac{p_\star}{T_\star} \, .
    \label{eq:lambda_star_definition}
\end{equation}
Since, by design, all normalized quantities (except $L$) are 1 at the sonic radius $r=1$, the full radial dependence of $v(r)$, $T(r)$, $E(r)$ and $q(r)$ is determined by providing $\gamma$, $\lambda_\star$ and $E_\star$ (implicit in $\Lambda$ and $L$).
While $\gamma$ is a material dependent parameter, $\lambda_\star$ and $E_\star$ will be shown to be functions only of $E_\text{bc}$ for the chosen boundary conditions in \cref{eq:full_boundary_conditions}.

Considering $T(1)=v(1)=1$ introduces an apparent singularity in \cref{eq:ngs_normalized_dv2} in the form of $1/(T-v^2)$.
However, requiring that $\partial{v^2}/\partial{r}$ is finite at the sonic radius produces an additional constraint on the normalization constants, found during the derivation of \cref{eq:ngs_normalized_dv2,eq:ngs_normalized_dT} to be
\begin{equation}
    \frac{4 \pi r_\star^2}{G} \frac{(\gamma - 1)}{\gamma} \frac{\lambda_\star \mu q_\star}{T_\star} = 2 \, .\label{eq:ngs_singularity_constraint}
\end{equation}
Using L'Hôpital's rule enables evaluation of the apparent singularity at the sonic radius, yielding
\begin{equation}
    \chi_\star := \eval{\pdv{v^2}{r}}_{r = 1} = \frac{2}{\gamma+1} \left((3 - \gamma) + \sqrt{(3-\gamma)^2 - 2(\lambda_\star + \Psi_\star - 1)(\gamma + 1)} \right) \, ,
    \label{eq:chi_star}
\end{equation}
with an additional shorthand defined as $\Psi_\star := 2 \lambda_\star \left( L \pdv{\Lambda}{E} \right)_{E=1}$.
Now everything is known to start finding numerical solutions for the normalized quantities, which directly determine the physical quantities.

%%%%%%%%%%%%%%%%%%%%%%%%%%%%%%%%%%%%%%%%%%%%%%%%%%%%%%%%%%%%%%%%%%%%%%%%%%%%%%%%%%%%%%%%%%%
\subsection{Numerical solution of the isotropic NGS model}
\label{ssec:isotropic_numerics}

The goal of the procedure described below is to find numerical solutions for $v(r)$, $T(r)$, $E(r)$ and $q(r)$, which satisfy both the differential \cref{eq:ngs_normalized_dv2,eq:ngs_normalized_dT,eq:ngs_normalized_dE,eq:ngs_normalized_dq} and the boundary conditions in \cref{eq:full_boundary_conditions}. 
For this purpose, for some chosen $\gamma$ and $E_\star$, we adjust $\lambda_\star$ so that $T(r_p) = 0 = q(r_p)$ at some normalized radius $r_p<1$, which is then interpreted as the normalized pellet radius. 
Together with the values $q(\infty)$ and $E(\infty)$ all normalization constants are determined, and the full physical solution can be calculated. 
Remember that $E_\star$ denotes the incident electron energy at the sonic radius, which is not known a priori.
As a first guess, it suffices to set $E_\star \approx E_\text{bc}$, since most of the energy is absorbed only inside the sonic radius.
However, a scan over $E_\star$ allows setting it more precisely afterwards.

Many different methods exist to solve a set of first order differential equations numerically.
Knowing that 
\begin{equation*}
    v(r=1)=1,\quad T(r=1)=1,\quad E(r=1)=1,\quad q(r=1)=1,
\end{equation*}
any method would work which solves initial value problems.
Since the derivatives are determined if all quantities at one radial position are known, the full spatial dependence can be built up iteratively.
The method of numerical integration found to work well here is an \enquote{implicit multi-step variable-order (1 to 5) method based on a backward differentiation formula for the derivative approximation}\footnote{See \url{https://docs.scipy.org/doc/scipy/reference/generated/scipy.integrate.solve_ivp.html} .} based on the implementation by \textcite{shampine_matlab_1997}.
The solutions are calculated from $r=1$ in the directions of both increasing and decreasing $r$.
Optimization of $\lambda_\star$ is done through minimizing the error of the boundary conditions $T(r_p) = q(r_p) = 0$.
Here, \enquote{a modification of the Powell hybrid method as implemented in MINPACK}\footnote{See \url{https://docs.scipy.org/doc/scipy/reference/generated/scipy.optimize.root.html} .} is used \autocite{more_user_1980}.

\begin{figure}
    \centering
    \includegraphics{figure/ode0_solution_Estar_3.000e+04_gamma_7_5.pdf}
    \caption{Numerical example solution of the normalized isotropic ablation dynamics. Chosen are the parameters $\gamma = 7/5$ and $E_\star = \qty{30}{\kilo\eV} \approx E_\text{bc}$ to have a direct comparison to the solution shown by \textcite{parks_effect_1978}.}
    \label{fig:example_0th_order_solution}
\end{figure}

An example solution for $\gamma = 7/5$ and $E_\star = \qty{30}{\keV}$ is shown in \cref{fig:example_0th_order_solution}, where the subscript 0 denotes the isotropic NGS model quantities as opposed to the asymmetric NGS model quantities with a subscript 1, as introduced in the next section.
This same solution is shown by \textcite{parks_effect_1978} in their figs. 3 and 4, which serves as a validation of our implemented numerical procedure.
Visible is that $E(\infty) \approx 1$, which justifies $E_\star \approx E_\text{bc}$.

Another validation is performed by scanning over the parameters $\gamma$ and $E_\text{bc}$ and plotting the calculated results for $\lambda_\star$, $r_\text{p}$, $E(\infty)$ and $q(\infty)$, as shown in \cref{fig:ode0_scan}.
Again, we see a match to figs. 1 and 2 of the paper by \textcite{parks_effect_1978}.
Additionally, scaling laws for $\gamma = 5/3$ (dotted lines) are given in the paper, which agree well with our results.
It is important to note that in \cref{fig:ode0_scan} the energy dependence is shown on a logarithmic x-axis, while the variance on the y-axis is small and linear.
Thus, the dependence on $E_\text{bc}$ is very weak, and the quantities can be considered nearly constant.
The scaling laws representing our solution (dashed lines) are given in \cref{sec:appendix_scaling_laws}.
%(Do I want to show how well the boundary conditions are met? And that the limitation $E < \qty{20}{\eV}$ might matter a bit? Also, I might want to show $p_0(r_p)$)

\begin{figure}
    \centering
    \includegraphics{figure/recreated_Parks_E_inf.pdf}
    \caption{Weak dependence on $\gamma$ and $E_\text{bc}$ of the normalized isotropic NGS model output. The shown quantities $\lambda_\star$, $r_\text{p}$, $E_0(\infty)$ and $q_0(\infty)$ are needed to calculate the physical ablation dynamics.}
    \label{fig:ode0_scan}
\end{figure}

The solution of the normalized spherically symmetric quantities is all that is needed to model the normalized asymmetric perturbation, as described in the next section about the asymmetric NGS model.
However, to calculate any physical quantities, the sonic normalization constants, as defined in \cref{eq:ngs_normalization}, must be known.
Given the physical boundary conditions, i.e. model input parameters, $r_\text{p}$, $E_\text{bc}$ and $q_\text{bc}$, the results as presented in \cref{fig:ode0_scan} allow the direct calculation of
\begin{align}
    r_\star = \frac{r_p}{\widetilde{r}_p}, \quad 
    E_\star = \frac{E_\text{bc}}{\widetilde{E}(\infty)}, \quad \text{and} \quad
    q_\star = \frac{q_\text{bc}}{\widetilde{q}(\infty)}\,.
\end{align}
Apart from the sonic density $\rho_\star = m p_\star/T_\star$ (ideal gas law), the remaining unknown normalization constants are $p_\star$, $T_\star$ and $v_\star$.
These quantities, together with the particle ablation rate $G$, are determined by solving the system of \cref{eq:ngs_mass_conservation,eq:speed_of_sound,eq:ngs_singularity_constraint,eq:lambda_star_definition}, yielding
\begin{align}
    \Aboxed{p_\star &= \underbrace{ \frac{\lambda_\star}{\gamma} \left( \frac{\widetilde{r}_\text{p} (\gamma-1)^2}{4 \widetilde{q}^2(\infty)} \right)^\frac{1}{3} }_{f_p(E_\text{bc}, \gamma)} \cdot \left[ \frac{m (\mu q_\text{bc})^2}{\Lambda_\star r_\text{p}} \right]^\frac{1}{3}} \, , \label{eq:prefactor_p_star} \\
    T_\star &= \underbrace{ \frac{1}{\gamma} \left( \frac{\gamma-1}{2  \widetilde{r}_\text{p} \widetilde{q}(\infty)} \right)^\frac{2}{3} }_{f_T(E_\text{bc}, \gamma)} \cdot \left[ \sqrt{m} \Lambda_\star \mu q_\text{bc} r_\text{p} \right]^\frac{2}{3} \, , \label{eq:prefactor_T_star} \\
    v_\star &= \underbrace{ \left( \frac{\gamma - 1}{2 \widetilde{r}_\text{p} \widetilde{q}(\infty) } \right)^\frac{1}{3} }_{f_v(E_\text{bc}, \gamma)} \cdot \left[ \frac{\Lambda_\star \mu q_\text{bc} r_\text{p}}{m} \right]^\frac{1}{3} \, , \label{eq:prefactor_v_star} \\
    G &= \underbrace{ 4 \pi \lambda_\star \left( \frac{\gamma - 1}{2 \widetilde{r}_\text{p}^4 \widetilde{q}(\infty)} \right)^\frac{1}{3} }_{f_G(E_\text{bc}, \gamma)} \cdot \left[ \frac{\mu q_\text{bc} r_\text{p}^4}{\Lambda_\star^2 m} \right]^\frac{1}{3} \, . \label{eq:prefactor_G}
\end{align}
The physical quantities in the square brackets are the main contributions.
The prefactors $f_p$, $f_T$, $f_v$ and $f_G$ combine all dimensionless factors and are shown to only weakly depend on $E_\text{bc}$ in \cref{fig:prefactors_0th_order}.
Scaling laws for those prefactors are provided in \cref{sec:appendix_scaling_laws}.
Since the pellet rocket effect mainly depends on the pressure asymmetry at the pellet surface, the most important formula here is \cref{eq:prefactor_p_star}, where $f_p \approx 0.15$ and as stated earlier $\mu \approx 0.65$.
The importance of $p_\star$ will become clearer towards the end of the next section, in which our approach of modelling asymmetry in the ablation dynamics is detailed.
%(Should mention the mistakes that I have found in Parks paper in his formulas?)

\begin{figure}
    \centering
    \includegraphics{figure/prefactors_0th_order.pdf}
    \caption{Weak dependence on $\gamma$ and $E_\text{bc}$ of the dimensionless prefactors for $p_\star$, $T_\star$, $v_\star$ and $G$ of the isotropic NGS model in \cref{eq:prefactor_p_star,eq:prefactor_T_star,eq:prefactor_v_star,eq:prefactor_G}.}
    \label{fig:prefactors_0th_order}
\end{figure}



\subsection{Asymmetric perturbation model}
\label{ssec:asymmetric_perturbation_model}

Provided a procedure of calculating the full ablation dynamics under the assumption of spherical symmetry, it is now possible to describe the asymmetry in the ablation dynamics as a perturbative model.
The baseline model is the previously described NGS model, will be denoted from now on with the index 0 and can be assumed as fully known.
Assuming the real ablation process is described well by the NGS model quantities $y_0(r)$ with a small correction $\var{y}(r,\theta)$, which depends on the asymmetry along one axis ($z$-axis in \cref{fig:neutral_cloud_heating}), the full physical quantities can be modelled as
\begin{equation}
\begin{aligned}
    \rho &= \rho_0 + \var\rho\, , & 
    p &= p_0 + \var{p}\, , \\
    T &= T_0 + \var{T}\, , & 
    \vec{v} &= \hat{r} (v_0 + \var{v}_r) + \hat{\theta} (0+\var{v}_\theta)\, , \\
    q &= q_0 + \var{q}\, , & 
    E &= E_0 + \var{E}\, .
\end{aligned}
\end{equation}
The spherical coordinate system is chosen as depicted in \cref{fig:neutral_cloud_heating,fig:pellet_surface_force}, while neglecting the $\varphi$-dependence.

Treating the system as a quasi-steady-state ideal gas is retained in our model in the same way the NGS model is treated.
Therefore, the \cref{eq:ideal_gas_law,eq:full_mass_conservation,eq:full_momentum_conservation,eq:full_energy_conservation,eq:heat_flux_approximation,eq:full_effective_heat_flux,eq:full_electron_energy_loss} form the basis of our asymmetric perturbative model.
This includes the approximation of a purely radial flux of electrons, as represented by \cref{eq:heat_flux_approximation}.
Linearizing the set of equations in the perturbation quantities $\var{y}$, combined with the fact that the NGS quantities $y_0$ satisfy those equations themselves, yields
\begin{flalign}
    &\var{\rho} = m \frac{\var{p}}{T_0} - m \frac{p_0}{T_0^2} \var{T} 
    &\text{(ideal gas law),}& \label{eq:linearized_ideal_gas_law} \\
    &\vec{\nabla} \cdot (\var{\rho} \vec{v}_0 + \rho_0 \var{\vec{v}}) = 0 
    &\text{(mass conservation),}&\label{eq:linearized_mass_conservation} \\
    &\rho_0 (\vec{v}_0 \cdot \vec{\nabla})\var{\vec{v}} + \rho_0 (\var{\vec{v}} \cdot \vec{\nabla})\vec{v}_0 + \var{\rho} (\vec{v}_0 \cdot \vec{\nabla}) \vec{v}_0 = - \vec{\nabla}\var{p} 
    \span\span\nonumber\\&&\text{(momentum conservation),}&\label{eq:linearized_momentum_conservation} \\
    &\vec{\nabla} \cdot \left[ \left( \frac{1}{2}\rho_0 v_0^2 + \frac{\gamma}{\gamma - 1}p_0 \right)\var{\vec{v}} + \left( \frac{1}{2}\var{\rho} v_0^2 + \rho_0 (\vec{v}_0\cdot \var{\vec{v}}) + \frac{\gamma}{\gamma - 1}\var{p} \right)\vec{v}_0 \right] = \mu \pdv{\var{q}}{r} 
    \span\span\nonumber\\&&\text{(energy conservation),}& \label{eq:linearized_energy_conservation} \\
    &\pdv{\var{E}}{r} = 2\frac{\var{\rho}}{m}L(E_0) + 2 \frac{\rho_0}{m} \eval{\pdv{L}{E}}_{E_0}  \!\!\!\! \var{E} 
    &\text{(electron energy loss),}&\label{eq:linearized_effective_heat_flux} \\
    &\pdv{\var{q}}{r} = \frac{\var{\rho}}{m} q_0 \Lambda(E_0) + \frac{\rho_0}{m} \var{q} \Lambda(E_0) + \frac{\rho_0}{m} q_0 \eval{\pdv{\Lambda}{E}}_{E_0} \!\!\!\! \var{E} \quad
    \span\text{(effective heat flux).}& \label{eq:linearized_electron_energy_loss}
\end{flalign}

Now we want to find a set of equations for the $r$-dependence and a separate set of equations for the $\theta$-dependence.
Fortunately, this is possible without further approximations by expanding the perturbation in terms of general fully orthogonal basis functions $\{X_l(\theta)\}$ in the form
\begin{align}
\begin{aligned}
    \var{\rho} &= \sum_l \rho_l(r) X_l(\theta) \, , &
    \var{p} &= \sum_l p_l(r) X_l(\theta) \, , \\
    \var{T} &= \sum_l T_l(r) X_l(\theta) \, , &
    \var{v}_r &= \sum_l v_{l,r}(r) X_l(\theta) \, , \\
    \var{q} &= \sum_l q_l(r) X_l(\theta) \, , &
    \var{E} &= \sum_l E_l(r) X_l(\theta) \, , &
\end{aligned}
\end{align}
while expanding only $\var{v}_\theta$ in terms of a different general basis $\{Y_l(\theta)\}$ as
\begin{equation}
    \var{v}_\theta = \sum_l v_{l,\theta}(r) Y_l(\theta) \, .
\end{equation}

% the set of \cref{eq:linearized_ideal_gas_law,eq:linearized_mass_conservation,eq:linearized_momentum_conservation,eq:linearized_energy_conservation,eq:linearized_effective_heat_flux,eq:linearized_electron_energy_loss}
Taking the derivatives in terms of spherical coordinates, the only $\theta$-derivative appearing in \cref{eq:linearized_mass_conservation,eq:linearized_energy_conservation} is
\begin{equation*}
    \vec{\nabla}\cdot(\var{v}_\theta \hat{\theta}) = \frac{1}{r \sin\theta}\pdv{\theta}(\sin\theta \var{v}_\theta) \, ,
\end{equation*}
while all other terms are linear in the $\theta$-dependence.
Therefore, separating the $r$- from the $\theta$-dependence in those equations is possible by requiring
\begin{equation}
    X_l(\theta) \propto \frac{1}{\sin\theta} \pdv{\theta}(\sin\theta Y_l(\theta)) \, .
    \label{eq:X_l_requiremetn}
\end{equation}
Similarly, the only $\theta$-derivative in \cref{eq:linearized_momentum_conservation} is
\begin{equation*}
    \hat{\theta} \cdot \vec{\nabla}\var{p} = \frac{1}{r} \pdv{\var{p}}{\theta} \, ,
\end{equation*}
while the only other components in the $\hat{\theta}$-direction are linear in $\var{v}_\theta$.
This leads us to require
\begin{equation}
    Y_l(\theta) = \pdv{X_l}{\theta}\, .
    \label{eq:Y_l_requiremetn}
\end{equation}
The last two \cref{eq:linearized_effective_heat_flux,eq:linearized_electron_energy_loss} by design do not contain $\theta$-derivatives or $\var{v}_\theta$.
Combining those two requirements and choosing the proportionality constant in \cref{eq:X_l_requiremetn} to be $-1/(l(l+1))$ leads to the defining differential equation for associated Legendre polynomials $X_l(\theta) = P^0_l(\cos\theta)$.
These are precisely the $m=0$ spherical harmonics with no $\varphi$-dependence.
Thus, the $\theta$-dependence of each mode $l$ is known and leads to a set of equations for the $r$-dependent coefficients of each mode $l$.
Note that the $l=0$ mode is independent of $\theta$, which motivates our choice of notation $y_0$ for the NGS model quantities.

After a rather lengthy derivation, the set of equations describing the $r$-dependence of the asymmetric neutral ablation cloud quantities is
\begin{align}
    &\rho_l = m \left( \frac{p_l}{T_0} - \frac{p_0}{T_0^2} T_l \right) \, , \label{eq:physical_perturbation_ideal_gas_law} \\
    &\pdv{\rho_0}{r}v_{l,r} + \rho_0 \left[ \frac{1}{r^2} \pdv{r}(r^2 v_{l,r}) - \frac{l(l+1)}{r} v_{l,\theta} \right] + v_0 \pdv{\rho_l}{r} + \frac{1}{r^2}\pdv{r}(r^2 v_0) \rho_l = 0 \, , \label{eq:physical_perturbation_mass_conservation} \\
    &\rho_0 v_0 \pdv{v_{l,r}}{r} + \rho_0 \pdv{v_0}{r} v_{l,r} + v_0 \pdv{v_0}{r} \rho_l = - \pdv{p_l}{r} \, , \label{eq:physical_perturbation_r_momentum_conservation} \\
    &\rho_0 v_0 \pdv{v_{l,\theta}}{r} + \rho_0 \frac{v_0}{r} v_{l,\theta} = - \frac{p_l}{r} \, , \label{eq:physical_perturbation_theta_momentum_conservation} \\
    &\left[ v_{l,r} \pdv{r} + \frac{1}{r^2}\pdv{r}(r^2 v_{l,r}) - \frac{l(l+1)}{r} v_{l,\theta} \right] \left( \frac{1}{2} \rho v_0^2 + \frac{\gamma}{\gamma - 1} p_0 \right) \nonumber \\
    &+ \left[ v_0 \pdv{r} + \frac{1}{r^2}\pdv{r}(r^2 v_0) \right] \left( \frac{1}{2} \rho_l v_0^2 + \rho_0 v_0 v_{l,r} + \frac{\gamma}{\gamma-1}p_l \right) = \mu \pdv{q_l}{r} \, , \label{eq:physical_perturbation_energy_conservation}
\end{align}
while the electron dynamics in the neutral gas are described by
\begin{align}
    &\pdv{E_l}{r} = 2 \frac{\rho_l}{m} L(E_0) + 2 \frac{\rho_0}{m} \eval{\pdv{L}{E}}_{E_0} E_l \, , \label{eq:physical_perturbation_electron_energy_loss} \\
    &\pdv{q_l}{r} = \frac{\rho_l}{m} q_0 \Lambda(E_0) + \rho_0 q_l \Lambda(E_0) + \rho_0 q_0 \eval{\pdv{\Lambda}{E}}_{E_0} E_l \, . \label{eq:physical_perturbation_effective_heat_flux}
\end{align}

Apart from enabling the separation of variables, expansion in terms of Legendre polynomials is convenient for calculating the pellet rocket force.
As shown in \cref{sec:pellet_surface}, the pellet rocket force only depends on the projection of $\var{\rho}$, $\var{v_r}$, $\var{p}$ and $\int \var{v_\theta} \dd{\theta}$ onto $P^0_1=\cos\theta$ at the pellet surface.
Since the ablation dynamics of all $l$-modes are independent here, only the $l=1$ mode is needed in our model.
Without loss of generality, it suffices to see the perturbation quantities as
\begin{align}
\begin{aligned}
    \var{\rho} &= \rho_1(r) \cos\theta \, , &
    \var{p} &= p_1(r) \cos\theta \, , \\
    \var{T} &= T_1(r) \cos\theta \, , &
    \var{\vec{v}} &= \hat{r} v_{1,r}(r) \cos\theta - \hat{\theta} v_{1,\theta}(r) \sin\theta \, , \\
    \var{q} &= q_1(r) \cos\theta \, , &
    \var{E} &= E_1(r) \cos\theta \, . &
\end{aligned}
\end{align}
This means that for positive coefficients $y_1(r)$, the full quantity $y(\vec{r})$ has slightly increased values in the positive $\hat{z}$ direction and slightly decreased values in the negative $\hat{z}$ direction, while at $z=0$ the NGS model is unperturbed.
However, the sign of $v_{1,\theta}$ determines the direction of the angular flow velocity, which is always largest at $z=0$ and zero along the $\hat{z}$-axis.

The boundary conditions for the perturbation quantities are derived from the boundary conditions in \cref{eq:full_boundary_conditions} to be 
\begin{gather}
\begin{gathered}
    q_1(r_p) = 0, \quad
    T_1(r_p) = 0, \quad
    v_{1,\theta}(r_p) = 0, \quad
    p_1(r \rightarrow \infty) = 0, \\
    q_1(r \rightarrow \infty) = \frac{3}{2} \int_0^\pi q_\text{bc}(\theta) \cos\theta \sin\theta \dd{\theta}, \\
    E_1(r \rightarrow \infty) = \frac{3}{2} \int_0^\pi E_\text{bc}(\theta) \cos\theta \sin\theta \dd{\theta}.
\end{gathered}
\end{gather}
Since introducing $v_{1,\theta}$ adds the need for one additional boundary condition, we assume $v_{1,\theta}(r_p) = 0$.
Without assuming a particular $\theta$-dependence of the incoming electron heat flux and energy, it is projected onto the $\cos\theta$ mode.

Solving for the perturbative ablation dynamics is again not possible fully analytically, and the equations need to be prepared for numerical analysis.
The perturbation quantities $y_1$ are predicted to be on the same order of magnitude as the NGS model quantities $y_0$ times the degree of asymmetry in the external heat source.
Therefore, it turns out to be convenient to define the relative contributions of the asymmetry in the total external heat
\begin{equation}
    q_\text{rel} = \frac{q_{1}(\infty)}{q_{0}(\infty)} \quad \text{and} \quad 
    E_\text{rel} = \frac{E_{1}(\infty)}{E_{0}(\infty)} \, .
\end{equation}
The perturbation assumption requires $|q_\text{rel}| \ll 1$ and $|E_\text{rel}| \ll 1$.
Furthermore, the signs determine which side of the ablation cloud receives a higher heat flux or higher energetic electrons.

The normalization of the NGS model to the sonic radius allowed to reduce the complexity of the problem by use of additional physics knowledge in the form of \cref{eq:speed_of_sound}.
Such additional knowledge is not available for the asymmetric perturbation.
Nevertheless, it turns out to be convenient to normalize the perturbation quantities $y_1$, similarly to the NGS model, as
\begin{gather}
\begin{gathered}
    \widetilde{\rho}_1=\frac{\rho_1}{\rho_\star q_\text{rel}}, \quad 
    \widetilde{p}_1 = \frac{p_1}{p_\star q_\text{rel}}, \quad 
    \widetilde{T}_1 = \frac{T_1}{T_\star q_\text{rel}}, \quad 
    \widetilde{v}_{1,r}=\frac{v_{1,r}}{v_\star q_\text{rel}}, \quad 
    \widetilde{v}_{1,\theta}=\frac{v_{1,\theta}}{v_\star q_\text{rel}}, \\
    \widetilde{q}_1=\frac{q_1}{q_\star q_\text{rel}}, \quad 
    \widetilde{E}_1=\frac{E_1}{E_\star  q_\text{rel}}. \quad 
\end{gathered}
\end{gather}
Together with normalizing $y_0$ as previously defined in \cref{eq:ngs_normalization}, this simplifies the heat source boundary conditions to
\begin{equation}
    \widetilde{q}_{1}(\infty) = \widetilde{q}_{0}(\infty) \quad \text{and} \quad 
    \widetilde{E}_{1}(\infty) = \widetilde{E}_{0}(\infty) \frac{E_\text{rel}}{q_\text{rel}}.
\end{equation}
Again, the \textasciitilde{} notation is dropped from now on and all quantities can be considered normalized, if not stated otherwise.

The chosen normalization leaves the system of \cref{eq:physical_perturbation_ideal_gas_law,eq:physical_perturbation_mass_conservation,eq:physical_perturbation_r_momentum_conservation,eq:physical_perturbation_theta_momentum_conservation,eq:physical_perturbation_energy_conservation,eq:physical_perturbation_effective_heat_flux,eq:physical_perturbation_electron_energy_loss} nearly unchanged.
In particular, the linearity in the perturbation quantities leads to a cancellation of all $q_\text{rel}$ factors.
The only changes in terms of new factors on the right side of the equations are 
\begin{equation*}
\begin{array}{cll}
    \frac{1}{\gamma} &\quad\quad& \text{in \cref{eq:physical_perturbation_r_momentum_conservation,eq:physical_perturbation_theta_momentum_conservation},} \\
    \frac{\gamma}{\gamma-1} \frac{2}{\lambda_\star \mu} && \text{in \cref{eq:physical_perturbation_energy_conservation} and} \\
    m \lambda_\star &&\text{in \cref{eq:physical_perturbation_electron_energy_loss,eq:physical_perturbation_effective_heat_flux}.}
\end{array}
\end{equation*}
Therefore, apart from the normalized NGS model parameters $\gamma$, $E_\star$ and $\lambda_\star$, the only new parameter needed to determine the normalized perturbation is $E_\text{rel}/q_\text{rel}$.

Since the derived system of equations is linear in both the perturbation quantities $y_1$ and their derivatives $\partial y_1/\partial r$, it is convenient to write it, in terms of $6 \times 6$, $y_0$-dependent matrices $A(r)$ and $B(r)$, as
\begin{equation}
    A \pdv{\vec{y}_1}{r} = B \vec{y}_1 \quad \text{with} \quad \vec{y}_1 = (p_1, T_1, v_{1,r}, v_{1,\theta}, q_1, E_1)^T \, .
\end{equation}
Symbolic computation allows finding an analytic expression for $C = A^{-1}B$ so that
\begin{equation}
    \pdv{\vec{y}_1}{r} = C \vec{y}_1 \, .
\end{equation}
Since this expression is large and those details are not necessary here, it is only given in \cref{sec:appendix_sympy_expressions}.
However, an important feature of $C$ is, that it contains the apparent singularity $1/(T_0 - v_0^2)$ in front of the first three rows.
Requiring that $\partial \vec{y}_1/\partial r$ is finite at the sonic radius leads to a matrix $\eval{(T_0 - v_0^2)C}_{r=1}$ of which the first three rows have rank 1.
Thus, this requirement reduces the number of unknowns at the sonic radius by one in the sense of
\begin{equation}
v_{1,\theta} = \left(1- \frac{\chi_\star}{2}\right) v_{1,r} + \left(1 + \frac{\chi_\star}{4}\right)T_1 - q_1 - \frac{\Psi_\star}{2 \lambda_\star L(E=1)} E_1 \, .
\end{equation}
Equivalently to the normalized NGS model, the derivatives at $r=1$ can then be evaluated using L'Hôpital's rule.
The corresponding expression for $C_\star = \eval{C}_{r=1}$ is given in \cref{sec:appendix_sympy_expressions}.
Having solved this apparent singularity, everything is provided to start finding numerical solutions.


(Now I have to describe the numerical procedure.)





    \subsubsection{Analytical description}
    \subsubsection{Numerical solution}

\section{Plasmoid shielding}
\label{sec:plasmoid_shielding}

Even though the major part of the heating occurs in the neutral gas close to the pellet, the electrons already lose some energy while traversing the ionized part of the ablation cloud.
This \emph{plasmoid shielding} of the background plasma electrons depends mainly on the integrated density along the electron path through the plasmoid.
As described in \cref{ssec:plasmoid_cloud}, the ablated material becomes confined by the magnetic field lines once it is ionized, while the pellet and the neutral ablation cloud move unaffected by the magnetic field.
Additionally, the plasmoid drifts towards the low-field side due to an $\vec{E} \times \vec{B}$-drift, as described in \cref{ssec:plasmoid_cloud}.
In \cref{ssec:plasmoid_shielding_length} we quantify how the drift of the ionized ablation material leads to a varying shielding length across the magnetic field lines, which in turn affects the pellet ablation dynamics.
\cref{ssec:plasmoid_heat_flux_attenuation} describes then our model for calculating the effective heat flux and effective electron energy arriving at the neutral cloud and their asymmetries, which are considered as boundary conditions for the neutral ablation cloud dynamics.

%%%%%%%%%%%%%%%%%%%%%%%%%%%%%%%%%%%%%%%%%%%%%%%%%%%%%%%%%%%%%%%%%%%%
\subsection{Drift induced shielding length asymmetry}
\label{ssec:plasmoid_shielding_length}

As described in \cref{sec:ablation_dynamics}, the flow velocity at the boundary of the neutral ablation cloud rapidly drops to subsonic speeds and a shock front forms.
Beyond this shock front, at ionization radius\footnote{Not to be confused with the sonic radius $r_\star$ much closer to the pellet.} $r_\text{i}$, the ablated material can be considered fully ionized and forms the plasmoid ablation cloud.
Radial expansion of the plasmoid due to heating is restricted in the direction perpendicular to the magnetic field lines.
Thus, the plasmoid expands in a tube parallel to the field lines at the speed of sound 
\begin{equation}
    c_\text{s} = \sqrt{\frac{(\gamma_e \langle Z \rangle + \gamma_i) T_\text{pl}}{\langle m_i \rangle}} \, ,
\end{equation}
where plasmoid temperature $T_\text{pl}$ (in units of energy).
In the case of pellets containing only a given hydrogen isotope, the adiabatic indices are $\gamma_e = 1$, $\gamma_i = 3$, the average charge number is $\langle Z \rangle = 1$ and the average ion mass is $\langle m_i \rangle =  m_\mathrm{H}$, $m_\mathrm{D}$ or $m_\mathrm{T}$ \autocite{vallhagen_drift_2023}.

Assuming the pellet is injected radially inwards from the low-field side, the ionized ablated material initially moves, relative to the pellet, at the pellet velocity $v_\text{p}$ outward along the major radius.
Additionally, the $\vec{E} \times \vec{B}$-drift towards the low-field side gradually accelerates material across the field lines and the plasmoid bends outwards compared to the flux surfaces, as illustrated in \cref{fig:plasmoid_shielding}.
As modelled by \textcite{vallhagen_drift_2023}, plasmoid material at the major radius $R_\text{m}$ is under constant acceleration
\begin{equation}
    \dot{v}_\text{pl} = \frac{2 (1+ \langle Z \rangle)}{\langle m_i \rangle R_\text{m}} \left( T_\text{pl} - \frac{2n_\text{bg}}{(1+ \langle Z \rangle)n_\text{pl}} T_\text{bg} \right) 
\end{equation}
shortly after ionization, where $T_\text{bg}$ and $n_\text{bg}$ are the electron temperature and electron density of the background plasma and $n_\text{pl}$ is the electron density in the plasmoid.
% The electron density in the plasmoid
% \begin{equation}
%     n_\text{pl} = \frac{(1+ \langle Z \rangle) \mathcal{G}}{2 \langle m_i \rangle c_\text{s} \pi r_\text{i}^2}
% \end{equation}
% is given by considering that ablated material is outflowing at the mass ablation rate $\mathcal{G}$ along a tube of cross-section $\pi r_\text{i}^2$. 

\begin{figure}
    \centering
    \includegraphics{figure/plasmoid_shielding.pdf}
    \caption{Illustration of the plasmoid shielding asymmetry of the ablation cloud. The drift towards the low-field side induces a shorter shielding length at the high-field side. Note that this is not to scale. The pellet is much smaller than the neutral ablation cloud and the plasmoid shielding length is much longer than the ionization radius, with a less pronounced shielding length difference.}
    \label{fig:plasmoid_shielding}
\end{figure}

Since the particles, once ionized, stop their motion in the positive $\hat{z}$-direction nearly instantaneously, the plasmoid boundary $z(x)$, as depicted in \cref{fig:plasmoid_shielding} with the pellet at the origin, is determined by the trajectory of particles, which are ionized at $(x=0, z=r_\text{i})$.
These particles follow the equation of motion
\begin{equation}
    \vec{r}(t) = (\pm c_\text{s} t) \hat{x} + \left(r_\text{i} - v_\text{p} t - \frac{1}{2} \dot{v}_\text{pl} t^2 \right) \hat{z} \, .
\end{equation}
The shielding length $s(z)$ along a field line at position $z$ is the distance from the plasmoid boundary to the neutral ablation cloud boundary (which lies at ${x = \pm \sqrt{r_\text{i}^2 - z^2}}$).
Therefore, the shielding length can be estimated by eliminating the time dependence in the equation of motion\footnote{We assume a constant acceleration over the whole trajectory.} and taking the absolute value, which gives
\begin{equation}
    s(z) = c_\text{s} \left( - \frac{v_\text{p}}{\dot{v}_\text{pl}} + \sqrt{\left(\frac{v_\text{p}}{\dot{v}_\text{pl}}\right)^2 + \frac{2}{\dot{v}_\text{pl}}(r_\text{i} - z)} \right) - \sqrt{r_\text{i}^2 - z^2} \, .
\end{equation}
The spherically symmetric ablation dynamics are considered to be determined solely through the central shielding length 
\begin{equation}
    s_0 = s(z=0) = c_\text{s} \frac{v_\text{p}}{\dot{v}_\text{pl}} \left( - 1 + \sqrt{1 + \frac{2 \dot{v}_\text{pl}}{v_\text{p}^2} r_\text{i}} \right) - r_\text{i} \, .
    \label{eq:central_shielding_length}
\end{equation}
The asymmetry in the neutral ablation cloud heating is determined by the shielding length variation $\var{s}$ across the field lines hitting the pellet.
Ideally, one would consider the shielding length variation across the whole neutral ablation cloud.
However, this would largely overestimate the degree of asymmetry in our model.
The reason follows from the approximation of a purely radial flux of electrons $\vec{\nabla}\cdot\vec{q} \approx \partial q / \partial r$, as illustrated in \cref{fig:neutral_cloud_heating}.
Since the heating close to the pellet surface dominates, the field lines in the interval $z \in [-r_\text{p}, r_\text{p}]$ are assumed to correspond to the full range $\theta \in [0, \pi]$ of heat flux in our model.
Under the assumption that the shielding length varies nearly linear in this interval, the variation becomes
\begin{align}
    \var{s}(z) &= \eval{\dv{s}{z}}_{z=0} \!\!\!\! \cdot z = \eval{\dv{s}{z}}_{z=0}  \!\!\!\! \cdot  r_\text{p} \cos{\theta} \, , \label{eq:shielding_length_variation} \\
    \text{with} \quad \eval{\dv{s}{z}}_{z=0} &= \frac{- c_\text{s}}{\sqrt{v_\text{p}^2 + 2 \dot{v}_\text{pl} r_\text{i}}} \, . \label{eq:shielding_length_derivative} 
\end{align}
% (Should I compare it to the \textcite{szepesi_comparison_2009} expression for the shielding length, as in Oskars summary document?)


%%%%%%%%%%%%%%%%%%%%%%%%%%%%%%%%%%%%%%%%%%%%%%%%%%%%%%%%%%%%%%%%%%%%
\subsection{Heat flux attenuation}
\label{ssec:plasmoid_heat_flux_attenuation}

The full dynamics of hot electrons losing energy while traversing a colder plasma involves multiple different mechanisms.
A rigorous description of this plasma shielding is ultimately kinetic, and it is outside the scope of this thesis.
Therefore, we approximate the heat flux attenuation by assuming that electrons only reach the neutral ablation cloud if their mean free path $\lambda_\text{mfp}$ is longer than the distance $d$ that they travel through the plasmoid.
Additionally, we assume that the electrons which reach the neutral ablation cloud are unaffected by the plasmoid and retain their thermal kinetic energy from the background plasma.

Although the electrons travel along the magnetic field lines according to the shielding lengths, as derived in \cref{ssec:plasmoid_heat_flux_attenuation}, their gyration around the field line leads to a longer total distance travelled through the plasmoid.
Let $\xi$ be the cosine of the pitch angle of an electron, i.e. the angle between the total velocity vector $\vec{v}_e$ and the velocity $\vec{v}_\parallel$ parallel to the magnetic field line.
Assuming the pitch angle of each electron stays constant while traversing the plasmoid, electrons contribute to heating the neutral ablation cloud if
\begin{equation}
    d = \frac{s}{\xi} < \lambda_\text{mfp} \, .
\end{equation}
Hot electrons slowing down in a cold plasma (the plasmoid) experience dominant collisions with the cold plasma electrons.
The corresponding mean free path is
\begin{equation}
    \lambda_\text{mfp} = \frac{v_e}{\nu_{ee}} = \frac{4 \pi \varepsilon_0^2 m_e^2 v_e^4}{n_\text{pl} e^4 \ln\Lambda} = \left( \frac{v_e}{v_\text{th}} \right)^4 \lambda_T \, , 
    \label{eq:lambda_mfp}
\end{equation} % remove (1+Z)
with the collision frequency $\nu_{ee}$, the hot electron velocity $v_e$, the cold electron density $n_\text{pl}$ in the plasmoid and the electron mass $m_e$ \autocite{helander_collisional_2005}.
For convenience, we define the mean free path $\lambda_T$ at thermal velocity $v_\text{th} = \sqrt{2T_\text{bg}/m_e}$.
The Coulomb-logarithm 
\begin{equation}
    \ln\Lambda = \ln\left( \lambda_\text{D} \cdot b_\text{min}^{-1} \right) = \ln \left( \sqrt{\frac{\varepsilon_0 T_\text{pl}}{n_\text{pl} e^2}} \cdot \left( \frac{\langle Z \rangle e^2}{2 \pi \varepsilon_0 m_e v_\text{th}^2} \right)^{-1} \right)
\end{equation}
is the order of magnitude of the number of particles in the Debye sphere of the plasma, which is typically between 10 and 20 in MCF plasmas \autocite{helander_collisional_2005}.

Since the mean free path depends mainly on the hot electron velocity, the condition for passing through the plasmoid unaffected can be written in terms of a critical velocity $v_\text{c}$ as
\begin{equation}
    v_e > v_\text{c} \quad \text{with} \quad v_\text{c} = \left( \frac{s}{\xi \lambda_T} \right)^\frac{1}{4} v_\text{th} \, .
\end{equation}
We assume that the hot electrons in the background plasma are distributed according to the three-dimensional Maxwellian distribution
\begin{equation}
    f_\text{Maxwell}(\vec{v}) = \left( \frac{1}{\sqrt{\pi} v_\text{th}} \right)^3 \exp\left[-\left(\frac{v}{v_\text{th}}\right)^2\right] \, .
\end{equation}
Then, the heat flux boundary condition for the neutral ablation cloud can be estimated by averaging the energy flux $q(\vec{v})$ over the velocities sufficient to pass through the plasmoid as
\begin{gather}
    q_\text{pl} = \underset{\substack{\abs{\vec{v}_e} > v_\text{c} \\  \xi \in [0,1]}}{\iiint} 
    \underbrace{\underbrace{\xi v_e}_{v_\parallel} \frac{m_e v_e^2}{2} n_\text{bg}}_{q(\vec{v}_e)} f_\text{Maxwell}(\vec{v}_e) \dd[3]{v_e} \, .
    % &= \int_0^{2\pi} \int_0^1 \int_{v_\text{c}}^\infty \left( \xi v_e \frac{m_e v_e^2}{2} n_\text{pl} \right) \left( \frac{1}{\sqrt{\pi} v_\text{th}} \right)^3 \exp\left[-\left(\frac{\abs{\vec{v}}}{v_\text{th}}\right)^2\right] v_e^2 \dd{v_e} \dd{\xi} \dd{\varphi_\text{gyr}}  \, .
\end{gather}
This integral can be solved exactly by substituting ${u := v_e/v_\text{th}} \Rightarrow {u(v_\text{c}) = \left( \xi \alpha \right)^{-\sfrac{1}{4}}}$, with the shorthand ${\alpha := \lambda_T/s}$. The result is
\begin{equation}
\boxed{%
    q_\text{pl}(s) = \underbrace{2 \sqrt{\frac{T_\text{bg}^3}{2 \pi m_e}} n_\text{bg}}_{q_\text{Parks}} 
    \underbrace{\frac{1}{\alpha^2}\left[ e^{\frac{-1}{\sqrt{\alpha}}}\left(-\frac{1}{2}\sqrt{\alpha}+\frac{1}{2}\alpha+\alpha^{3/2}+\alpha^2\right)-\frac{1}{2}\mathrm{Ei}\left(-\frac{1}{\sqrt{\alpha}}\right)\right]}_{f_q(\alpha)} \, , }
    \label{eq:plasmoid_shielding_q}
\end{equation}
with the exponential integral $\mathrm{Ei}(x) = - \int_{-x}^{\infty} \exp(-t)/t \dd{t}$.
Consequently, the heat flux without any plasmoid shielding $q_\text{Parks}$, as assumed by \textcite{parks_effect_1978}, is scaled down in our model by the shielding length-dependent dimensionless function $f_q(\alpha)$.

The effective electron energy at the neutral ablation cloud boundary is assumed to be
\begin{equation}
    E_\text{pl} = \frac{q_\text{pl}}{\Gamma_\text{pl}} \, ,
\end{equation}
where the particle flux $\Gamma_\text{pl}$ of electrons reaching the neutral ablation cloud is estimated similarly to the heat flux as
\begin{equation}
    \Gamma_\text{pl}(\theta) = \underset{\substack{\abs{\vec{v}_e} > v_\text{c} \\  \xi \in [0,1]}}{\iiint} 
    \underbrace{\xi v_e n_\text{bg}}_{\Gamma(\vec{v}_e)} f_\text{Maxwell}(\vec{v}_e) \dd[3]{v_e} \, .
\end{equation}
The result is again a scaling of the effective energy assumed by \textcite{parks_effect_1978} $E_\text{Parks}$ by a dimensionless function $f_E(\alpha)$, as
\begin{equation}
\boxed{%
    E_\text{pl}(s) = \underbrace{2 T_\text{bg}}_{E_\text{Parks}} 
    \underbrace{\left[
    \frac{ e^{\frac{-1}{\sqrt{\alpha}}}\left(-\frac{1}{2}\sqrt{\alpha}+\frac{1}{2}\alpha+\alpha^{3/2}+\alpha^2\right)-\frac{1}{2}\mathrm{Ei}\left(-\frac{1}{\sqrt{\alpha}}\right)}
    {e^{\frac{-1}{\sqrt{\alpha}}}\left(+\frac{1}{2}\sqrt{\alpha}-\frac{1}{2}\alpha+\alpha^{3/2}+\alpha^2\right) + \frac{1}{2}\mathrm{Ei}\left(-\frac{1}{\sqrt{\alpha}}\right)}
    \right]}_{f_E(\alpha)} \, .
    }
    \label{eq:plasmoid_shielding_E}
\end{equation}
The dimensionless functions $f_q(\alpha)$ and $f_E(\alpha)$ are visualized in \cref{fig:plasmoid_shielding_functions}.
While increasing shielding length $s$, i.e. decreasing $\alpha$, attenuates the heat flux, the average energy is enhanced.
This can be explained qualitatively by noticing that, with a longer shielding length through the plasmoid, fewer electrons reach the neutral cloud, but on average, those electrons are more energetic.
Whether the quantitative prediction of this effect is accurate will need further investigation.

\begin{figure}
    \centering
    \includegraphics{figure/plasmoid_shielding_functions.pdf}
    \caption{Scaling functions for the heat flux and energy boundary conditions due to plasmoid shielding, inversely depending on the shielding length $s$.}
    \label{fig:plasmoid_shielding_functions}
\end{figure}

After modelling how the heat flux and effective energy into the neutral ablation cloud depends on the shielding length, it can be combined with the shielding length prediction across different field lines to provide expressions for the boundary conditions to our model.
The central shielding length $s_0$ in \cref{eq:central_shielding_length} directly determines the boundary conditions to the isotropic NGS model as
\begin{equation}
\boxed{
\begin{aligned}
    q_\text{bc} &= q_\text{pl}(s_0) = q_\text{Parks} f_q(\alpha_0) \quad \quad \text{and} \\
    E_\text{bc} &= E_\text{pl}(s_0) = E_\text{Parks} f_E(\alpha_0)
\end{aligned}
} \, ,
\label{eq:plasmoid_shielding_isotropic}
\end{equation}
with $\alpha_0 := \lambda_T/s_0$.

The degree of asymmetry in heating the neutral ablation cloud depends on the shielding length variation given in \cref{eq:shielding_length_variation,eq:shielding_length_derivative}.
However, an additional source of asymmetry are the temperature and electron density gradients of the background plasma.
Consider thus the first order variation
\begin{align}
    \var{q_\text{pl}}(z) &= \eval{\dv{q_\text{pl}}{z}}_{z=0} z \\
    &= \left[ \pdv{q_\text{pl}}{T_\text{bg}} \dv{T_\text{bg}}{z} + \pdv{q_\text{pl}}{n_\text{bg}} \dv{n_\text{bg}}{z} + \pdv{q_\text{pl}}{\alpha} \left( \pdv{\alpha}{T_\text{bg}} \dv{T_\text{bg}}{z} + \pdv{\alpha}{s} \dv{s}{z} \right)  \right]_{z=0} z \\
    &= \left[ \frac{3}{2}\frac{q_\text{pl}}{T_\text{bg}} \dv{T_\text{bg}}{z} + \frac{q_\text{pl}}{n_\text{bg}} \dv{n_\text{bg}}{z} + q_\text{Parks} f'_q \left( \frac{2 \alpha}{\sqrt{T_\text{bg}}} \dv{T_\text{bg}}{z} - \frac{\alpha}{s} \dv{s}{z} \right)  \right]_{z=0} z \, ,
    \label{eq:q_pl_variation}
\end{align}
where $f'_q$ denotes $\partial f_q/\partial \alpha$.
While the heat flux asymmetry depends on the temperature and density gradients, this dependence is assumed to be negligible for the shielding length.
Thus $\partial \alpha/\partial T_\text{bg}$ is derived from the definition of $\lambda_T$ in \cref{eq:lambda_mfp} alone\footnote{Neglecting also the Coulomb logarithm dependence $\partial \ln\Lambda/\partial T_\text{bg}$.}.
Equivalently to the shielding length variation, we let $z = r_\text{p} \cos\theta$.
Consequently, the variation $\var{q_\text{pl}}$ corresponds to $q_1(\infty)\cos\theta$ in our asymmetric NGS model.
Comparison to the definition of the heat flux asymmetry parameter in \cref{eq:perturbation_normalized_bc} then leads to
\begin{equation}
    q_\text{rel} = \frac{q_1(\infty)}{q_\text{bc}} = \frac{1}{q_\text{pl}(s_0)} \eval{\dv{q_\text{pl}}{z}}_{z=0} r_\text{p} \, .
\end{equation}
Inserting the expressions of \cref{eq:plasmoid_shielding_isotropic,eq:q_pl_variation} and performing an equivalent derivation for the effective energy asymmetry finally gives
\begin{equation}
\boxed{%
\begin{aligned}
    q_\text{rel} &= r_\text{p} \left[ \left(\frac{3}{2} \frac{1}{T_\text{bg}} + \frac{f'_q}{f_q}  \frac{2 \alpha}{\sqrt{T_\text{bg}}} \right) \dv{T_\text{bg}}{z} + \frac{1}{n_\text{bg}} \dv{n_\text{bg}}{z} - \frac{f'_q}{f_q} \frac{\alpha}{s} \dv{s}{z} \right]_{z=0} \quad \quad \text{and} \\
    E_\text{rel} &= r_\text{p} \left[ \left( \frac{1}{T_\text{bg}} + \frac{f'_E}{f_E} \frac{2 \alpha}{\sqrt{T_\text{bg}}} \right) \dv{T_\text{bg}}{z} - \frac{f'_E}{f_E} \frac{\alpha}{s} \dv{s}{z} \right]_{z=0}
\end{aligned}
} \, ,
\label{eq:plasmoid_shielding_asymmetry}
\end{equation}
where the plasmoid shielding functions $f_q(\alpha)$ and $f_E(\alpha)$ are given by \cref{eq:plasmoid_shielding_q,eq:plasmoid_shielding_E}, and the shielding length variation $\eval{\dd s/\dd z}_{z=0}$ is described by \cref{eq:shielding_length_derivative}.
Note that in this model, $q_\text{rel}$ is always positive, but for a moderate temperature gradient, $E_\text{rel}$ is always negative.
Thus, $E_\text{rel}/q_\text{rel} \lesssim -1.17$ is possible, which would mean a pellet rocket acceleration towards the high-field side, according to \cref{fig:P1_at_r_p}.


% The degree of asymmetry in the neutral ablation cloud heating is obtained by taking the first order variation
% \begin{align}
%     \var{q_\text{pl}}(z) = \eval{\dv{q_\text{pl}}{z}}_{z=0} z

%     % q_1(\infty) \cos\theta = \eval{\pdv{q_\text{bc}}{s}}_{s=s_0} \var{s}(\theta) = \eval{\pdv{q_\text{bc}}{\alpha}}_{\alpha=\alpha_0} \underbrace{\eval{\pdv{\alpha}{s}}_{s=s_0}}_{=-\alpha_0/s_0} \eval{\pdv{s}{z}}_{z=0} r_p \cos\theta
%     q_1(\infty) \cos\theta &= \eval{\dv{q_\text{pl}}{z}}_{z=0} z
%     &= \left( \right)_{z=0} z
%     = \eval{\dv{q_\text{bc}}{z}}_{z=0} r_\text{p} \cos\theta \, ,
%     %= \eval{\pdv{q_\text{bc}}{\alpha}}_{\alpha=\alpha_0} \underbrace{\eval{\pdv{\alpha}{s}}_{s=s_0}}_{=-\alpha_0/s_0} \eval{\pdv{s}{z}}_{z=0} r_p \cos\theta
% \end{align}
% where $\var{z} = r_\text{p} \cos\theta$ reflects the same reasoning as explained for the shielding length variation in \cref{eq:shielding_length_variation}.
% Comparing this with the definition $q_\text{rel} = q_1(\infty)/q_\text(bc)$ .
% By handling the effective energy equivalently, the parameters to the asymmetric NGS model become
% \begin{equation}
% \boxed{%
% \begin{aligned}
%     q_\text{rel} &= \frac{-1}{f_q(\alpha_0)} \eval{\pdv{f_q}{\alpha}}_{\alpha=\alpha_0} \frac{\alpha_0}{s_0} \eval{\pdv{s}{z}}_{z=0} r_p \quad \text{and} \\
%     E_\text{rel} &= \frac{-1}{f_E(\alpha_0)} \eval{\pdv{f_E}{\alpha}}_{\alpha=\alpha_0} \frac{\alpha_0}{s_0} \eval{\pdv{s}{z}}_{z=0} r_p \, ,
% \end{aligned}
% }
% \label{eq:plasmoid_shielding_asymmetry}
% \end{equation}
% where the plasmoid shielding functions $f_q$ and $f_E$ are described by \cref{eq:plasmoid_shielding_q,eq:plasmoid_shielding_E}, and the shielding length variation $\eval{\partial s/\partial z}_{z=0}$ is described by \cref{eq:shielding_length_derivative}.
% Note that in this model, $E_\text{rel}/q_\text{rel}$ is always negative, since $\partial f_E/\partial \alpha$ is negative and $\partial f_q/\partial \alpha$ is positive.

In summary, the plasmoid shielding depends mainly on the pellet injection parameters $v_\text{p}$ and $r_\text{p}$, the background plasma parameters $T_\text{bg}$ and $n_\text{bg}$ (and their gradients), and the pellet position along the major radius of the magnetic confinement device $R_\text{m}$.
Since the plasmoid formation is not fully modelled here, values have to be given for the plasmoid temperature $T_\text{pl}$ and the ionization radius $r_\text{i}$.
Whereas, the plasmoid electron density can be estimated as
\begin{equation}
    n_\text{pl} = \frac{ \mathcal{G}}{2 \langle m_i \rangle c_\text{s}(T_\text{pl}) \pi r_\text{i}^2}
\end{equation}
by considering that ablated material is outflowing at the mass ablation rate $\mathcal{G}=mG$ along a tube of cross-sectional area $\pi r_\text{i}^2$.
Note that $m$ is the mass of the ablated molecules, i.e. before dissociation, and $\langle m_i \rangle$ is the mass of the ionized atoms.
However, the particle ablation rate $G$, as predicted by the isotropic NGS model in \cref{eq:prefactor_G}, depends itself on the heat flux $q_\text{bc}$ and effective electron energy $E_\text{bc}$.
Therefore, the plasmoid shielding, as estimated by \cref{eq:plasmoid_shielding_isotropic}, has to be calculated in a self-consistent iteration until the particle ablation rate $G$ is converged.
Then, the heat source asymmetry parameters in \cref{eq:plasmoid_shielding_asymmetry} can be used in \cref{eq:final_pellet_rocket_force} to predict the pellet rocket force.




%%%%%%%%%%%%%%%%%%%%%%%%%%%%%%%%%%%%%%%%%%%%%%%%%%%%%%%%%%%%%%%%%%%%%%%%%%%%%%%%%%%%%%%%%
\section{Model summary (putting it all together...)}
\label{sec:model_summary}