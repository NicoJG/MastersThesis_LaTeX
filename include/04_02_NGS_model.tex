\section{Neutral gas shielding (NGS)}
\label{sec:neutral_gas_shielding}

As noted in the introduction to \cref{chap:model}, the basis of our model is the isotropic NGS model, developed by \textcite{parks_effect_1978}.
Most of this section is dedicated to describing and reproducing this semi-analytical model.
The underlying physics processes related to ablation of hydrogen pellets were described already in \cref{sec:ablation_dynamics} and will not be repeated here.
The following text describes first the general ideas and approximations of the NGS model, which are used for both the isotropic and the asymmetric model. 
Then the mathematical details of how solutions are found in the isotropic case are presented in \cref{ssec:isotropic_model}.
Finally, \cref{ssec:asymmetric_perturbation_model} describes the perturbative extension in the form of our asymmetric NGS model.

Since the neutral ablation cloud can be considered a transonic ideal gas, the equation of state is
\begin{align}
    &\frac{\rho}{m} = \frac{p}{T}  &\text{(ideal gas law)} \, , \label{eq:ideal_gas_law}
%\end{align}
\intertext{%
with mass density $\rho$, pressure $p$, temperature\footnote{Temperatures are assumed to be in units of energy by including Boltzmann's constant.} $T$ and mass $m$ of one molecule (or atom) in the gas.
The full gas dynamics are obtained by considering the steady state conservation laws
}
%\begin{align}
&\vec{\nabla} \cdot (\rho \vec{v}) = 0 &\text{(mass conservation)} \, , \label{eq:full_mass_conservation} \\
&\rho (\vec{v} \cdot \vec{\nabla}) \vec{v} = - \vec{\nabla} p &\text{(momentum conservation)} \, , \label{eq:full_momentum_conservation} \\
&\vec{\nabla} \cdot \left[\left( \frac{\rho v^2}{2} + \frac{\gamma p}{\gamma - 1} \right) \vec{v}\right] = Q &\text{(energy conservation)} \, ,\label{eq:full_energy_conservation}
\end{align}
also known as Euler equations, with flow velocity $\vec{v}$, external heat source $Q$ and the adiabatic index\footnote{Also called heat capacity ratio.} of the gas $\gamma$.
In an ideal gas, the adiabatic index is related to the number of degrees of freedom $f$ of the gas particles through $\gamma = 1+2/f$, so it can be assumed to be $\gamma = 5/3$ for a monatomic gas, $\gamma = 7/5$ for a diatomic gas and $\gamma = 9/7$ for a linear triatomic gas.
In the case of frozen deuterium pellets, the temperature close to the pellet is so low that dissociation can be neglected, and the whole neutral ablation cloud is approximated as a \ce{D2} gas.

\begin{figure}
    \centering
    \includegraphics{figure/neutral_cloud_heating.pdf}
    \caption{Illustration of the asymmetric heating of the neutral gas ablation cloud by incoming electrons travelling along the magnetic field lines. Most of the heating occurs inside the sonic radius $r_\star$, close to the pellet radius $r_\text{p}$. The boundary between the neutral ablation cloud and the ionized ablation cloud is comparatively far away at $r_\text{i}$. The electrons are modelled to lose their energy radially inwards, while in reality they traverse the neutral ablation cloud on parallel straight lines from two sides.}
    \label{fig:neutral_cloud_heating}
\end{figure}

The external heat source $Q$ at each point in the gas is subject to the major approximations of the NGS model.
Heat conduction in the neutral cloud is neglected, since this occurs only on a much slower timescale than the heating from highly energetic electrons losing energy in the cloud through elastic scattering and inelastic processes \autocite{parks_effect_1978}.
While, in reality, the incoming electrons are constrained by the magnetic field to traverse the cloud on straight parallel lines, the NGS model approximates this path with an equivalent radial path, as illustrated in \cref{fig:neutral_cloud_heating}.
Therefore, the heat source
\begin{equation}
    Q = - \mu \vec{\nabla}\cdot \vec{q} = \mu \pdv{q}{x} \approx \mu \pdv{q}{r}
    \label{eq:heat_flux_approximation}
\end{equation}
is modelled as if the electrons lose their energy radially inwards instead of along the field lines in the direction $x$. 
This approximation is motivated by the fact that the energy flux reaching a spherical shell depends on the integrated density along the electron trajectory, which is geometrically similar for both the radial and the parallel paths \autocite{parks_model_1977-1}.
Considering the success of the isotropic NGS model, this approximation is retained in our model.
Only a fraction $\mu$ of the electron energy loss goes into heating the gas, while the rest goes mainly into Bremsstrahlung radiation and backscattered electrons.
\textcite{parks_effect_1978} state that the fraction $\mu$ is fairly well modelled to be between $60 \%$ and $70\%$ at all points in the cloud in the case of a hydrogenic gas.
Additionally, the incident electrons are approximated to all have the same energy once they reach the point $\vec{r}$ in the neutral ablation cloud.
These strong approximations of a mono-energetic beam of incoming electrons losing energy radially have been found to be sufficiently accurate for predicting pellet ablation rates in magnetic confinement fusion.
How those approximations degrade the accuracy of predictions of the pellet rocket effect has to be evaluated by comparison to experiments or numerical simulations.

Modelling the exact dynamics of how the electrons lose energy on their path through the gas is a non-trivial task.
Therefore, the NGS model uses empirical scaling laws for the scattering cross-sections of electrons in a hydrogen gas.
The electron dynamics are thus governed by the differential equations for the energy $E$ and heat flux $q$
\begin{align}
    &\dv{E}{r} = 2 \frac{\rho}{m} L(E) &\text{(electron energy loss)} \, , \label{eq:full_electron_energy_loss} \\
    &\dv{q}{r} = \frac{q}{\lambda_\text{mfp}(E)} = \frac{\rho}{m} q \Lambda(E) &\text{(effective heat flux)}  \, , \label{eq:full_effective_heat_flux}
\end{align}
where the mean free path $\lambda_\text{mfp}$ is modelled through the effective energy flux cross-section $\Lambda(E) = \hat{\sigma}_T(E) + 2 L(E)/E$, with the empirical energy loss function
\begin{gather}
L(E>\qty{20}{\eV}) = \frac{\qty{8.62e-15}{\eV.\cm^2}}{\left(\frac{E/\unit{\eV}}{100}\right)^{0.823} \!\!\!\! +\left(\frac{E/\unit{\eV}}{60}\right)^{-0.125}  \!\!\!\! +\left(\frac{E/\unit{\eV}}{48}\right)^{-1.94}} \, ,
\label{eq:energy_loss_function}
\end{gather}
derived by \textcite{miles_electron-impact_1972}. The corresponding effective backscattering cross-section
\begin{equation}
\hat{\sigma}_T (E) = \begin{dcases}
    \left(\frac{\num{8.8e-13}}{(E/\unit{\eV})^{1.71}}-\frac{\num{1.62e-12}}{(E/\unit{\eV})^{1.932}} \right)\, \unit{\cm^2} & \text{for } E>\qty{100}{eV} \\
    \left(\frac{\num{1.1e-14}}{(E/\unit{\eV})} \right) \unit{\cm^2} & \text{for } E<\qty{100}{\eV}
\end{dcases}
\end{equation}
was derived by \textcite{parks_model_1977-1} based on experimentally measured values by \textcite{maecker_ionen-_1955}.
These functions are visualized in \cref{fig:energy_attenuation}.

\begin{figure}
    \centering
    \includegraphics{figure/empirical_cross_sections.pdf}
    \caption{Empirical functions $\Lambda(E)$, $L(E)$ and $\hat{\sigma}_T(E)$ used in the NGS model by \textcite{parks_effect_1978} for the energy loss of electrons in a hydrogen gas.}
    \label{fig:energy_attenuation}
\end{figure}

The interdependence between the electron heat flux and the ablation cloud density at each point means that the full system of \cref{eq:ideal_gas_law,eq:full_mass_conservation,eq:full_momentum_conservation,eq:full_energy_conservation,eq:heat_flux_approximation,eq:full_effective_heat_flux,eq:full_electron_energy_loss} needs to be solved self-consistently.
For that, the correct number of boundary conditions have to be motivated that describe the physical system.
The low sublimation energy of hydrogen is the reason that the ablation cloud establishes itself to nearly fully shield the pellet.
Therefore, the heat flux as well as the gas temperature at the pellet surface can be neglected and taken to be exactly zero.
%The outer boundary of the neutral ablation cloud, where it meets the plasmoid ablation cloud, establishes itself as a shock front, where the flow velocity rapidly decreases to adjust to the pressure in the plasmoid.
Considering that the back pressure from the fusion plasma and the heat conduction play no significant role in the ablation dynamics, the boundary of the neutral ablation cloud can be seen as infinitely far away ($r_i \rightarrow \infty$) from the pellet with negligible pressure.
Given the heat flux and average energy of the electrons reaching the neutral ablation cloud, the boundary conditions can be summarized as
\begin{equation}
\begin{gathered}
    q(r_\text{p})=0, \quad T(r_\text{p})=0, \\
    p(r \rightarrow \infty) = 0, \quad
    q(r\rightarrow\infty) = q_\text{bc}, \quad E(r\rightarrow\infty) = E_\text{bc} \, .
    \label{eq:full_boundary_conditions}
\end{gathered}
\end{equation}
The heating parameters $q_\text{bc}(\theta,\varphi)$ and $E_\text{bc}(\theta,\varphi)$ depend on the plasmoid shielding of the background plasma electrons at temperature $T_\text{bg}$ and density $n_\text{bg}$, as modelled further down in \cref{sec:plasmoid_shielding}.
Note that no assumptions have to be made on the velocity at which ablated molecules leave the pellet surface, but it is found to be negligible in the numerical solutions under these boundary conditions.
However, for our asymmetric NGS model, we have to additionally assume the angular flow velocity at the pellet surface to be zero.

%%%%%%%%%%%%%%%%%%%%%%%%%%%%%%%%%%%%%%%%%%%%%%%%%%%%%%%%%%%%%%%%%%%%%%%%%%%
\subsection{Isotropic NGS model}
\label{ssec:isotropic_model}

Consider now the case of full spherical symmetry, where each quantity only depends on the radial coordinate $r$ and the velocity is purely radial ($\vec{v}=v \hat{r}$).
Using this symmetry to evaluate and integrate the mass conservation \cref{eq:full_mass_conservation} leads to
\begin{equation}
    4 \pi r^2 \frac{\rho}{m} v = \textit{const} = G \, .
    \label{eq:ngs_mass_conservation}
\end{equation}
This radially constant quantity represents the total outflow of particles through any spherical shell and is equal to the particle ablation rate\footnote{Note that \textcite{parks_effect_1978} defined $G$ to be the mass ablation rate $G_\text{Parks} = m \cdot G$.} $G$.
Similarly, using \cref{eq:ngs_mass_conservation} and the ideal gas law \cref{eq:ideal_gas_law}, the momentum conservation \cref{eq:full_momentum_conservation} and the energy conservation \cref{eq:full_energy_conservation} become
\begin{gather}
    \rho v \pdv{v}{r} = - \pdv{p}{r} \quad \quad \text{and} \label{eq:ngs_momentum_conservation} \\
    \frac{G}{4 \pi r^2} \pdv{r}(\frac{m}{2}v^2 + \frac{\gamma}{\gamma - 1} T) = \mu \pdv{q}{r} \label{eq:ngs_energy_conservation}
\end{gather}
respectively. 
Together with \cref{eq:full_electron_energy_loss,eq:full_effective_heat_flux} describing the incident electron dynamics and the boundary conditions in \cref{eq:full_boundary_conditions}, the full radial solutions can be determined.

A fully analytical expression for the solution is not tractable, so the equations were prepared for numerical integration.
It turns out to be useful to normalize all quantities to their values at the sonic radius $r_\star$, i.e. the radius at which the flow velocity $v$ transitions between subsonic to supersonic speeds, in the form
\begin{gather}
\begin{gathered}
    \widetilde{\rho}=\frac{\rho}{\rho_\star}, \quad \widetilde{p} = \frac{p}{p_\star}, \quad \widetilde{T} = \frac{T}{T_\star}, \quad \widetilde{v}=\frac{v}{v_\star}, \quad \widetilde{r}=\frac{r}{r_\star} \\
    \widetilde{q}=\frac{q}{q_\star}, \quad \widetilde{E}=\frac{E}{E_\star}, \quad \widetilde{\Lambda} = \frac{\Lambda}{\Lambda_\star}, \quad \widetilde{L} = \frac{L}{E_\star \Lambda_\star} , \quad \text{with} \quad \Lambda_\star=\Lambda(E_\star).
\end{gathered}
\label{eq:ngs_normalization}
\end{gather}
In this thesis, all quantities denoted with a $\star$ represent their physical values at the sonic radius.
Here, variables with a tilde  represent their normalized, dimensionless version.
However, the tildes will be omitted in the following to prevent visual clutter and all quantities can be considered normalized, if not stated otherwise.

Substituting the physical quantities by the normalized quantities in the ideal gas law \cref{eq:ideal_gas_law} and the mass conservation \cref{eq:ngs_mass_conservation} gives
\begin{align}
    \rho &= \frac{p}{T}\, \quad \text{and} \\
    r^2\frac{p}{T}v &= 1 \, .
\end{align}
The normalized set of differential equations is then obtained after some lengthy but straightforward algebra by additionally using the definition of the speed of sound in an ideal gas
\begin{equation}
    v_\star = \sqrt{\frac{\gamma T_\star}{m}} \, . \label{eq:speed_of_sound}
\end{equation}
%and evaluating \cref{eq:ideal_gas_law} and \cref{eq:ngs_mass_conservation} at the sonic radius.
The system of \cref{eq:ngs_momentum_conservation,eq:ngs_energy_conservation,eq:full_electron_energy_loss,eq:full_effective_heat_flux} can thus be rewritten in the normalized form
\begin{align}
    &\pdv{v^2}{r} = \frac{4v^2T}{(T-v^2)r}\left(\frac{q\Lambda r}{Tv} -1\right) \, , \label{eq:ngs_normalized_dv2}\\
    &\pdv{T}{r} = \frac{2\Lambda q}{v}-\frac{1}{2}(\gamma -1)\pdv{v^2}{r} \, , \label{eq:ngs_normalized_dT}\\
    &\pdv{E}{r} = 2\lambda _\star\frac{L}{r^2v}\, ,\label{eq:ngs_normalized_dE}\\
    &\pdv{q}{r} = \lambda _\star\frac{q\Lambda }{vr^2} \,.\label{eq:ngs_normalized_dq}
\end{align}
No approximations were made in this derivation and all normalization constants were combined into a single dimensionless quantity
\begin{equation}
    \lambda_\star = r_\star \Lambda_\star \frac{p_\star}{T_\star} \, .
    \label{eq:lambda_star_definition}
\end{equation}
Since, by design, all normalized quantities (except $L$) are 1 at the sonic radius $r=1$, the full radial dependence of $v(r)$, $T(r)$, $E(r)$ and $q(r)$ is determined by providing $\gamma$, $\lambda_\star$ and $E_\star$ (implicit in $\Lambda$ and $L$).
While $\gamma$ is a material dependent parameter, $\lambda_\star$ and $E_\star$ will be shown to be functions only of $E_\text{bc}$ for the chosen boundary conditions in \cref{eq:full_boundary_conditions}.

Considering $T(1)=v(1)=1$ introduces an apparent singularity in \cref{eq:ngs_normalized_dv2} in the form of $1/(T-v^2)$.
However, requiring that $\partial{v^2}/\partial{r}$ is finite at the sonic radius produces an additional constraint on the normalization constants, found during the derivation of \cref{eq:ngs_normalized_dv2,eq:ngs_normalized_dT} to be
\begin{equation}
    \frac{4 \pi r_\star^2}{G} \frac{(\gamma - 1)}{\gamma} \frac{\lambda_\star \mu q_\star}{T_\star} = 2 \, .\label{eq:ngs_singularity_constraint}
\end{equation}
Using L'Hôpital's rule enables evaluation of the apparent singularity at the sonic radius, yielding
\begin{equation}
    \chi_\star := \eval{\pdv{v^2}{r}}_{r = 1} = \frac{2}{\gamma+1} \left((3 - \gamma) + \sqrt{(3-\gamma)^2 - 2(\lambda_\star + \Psi_\star - 1)(\gamma + 1)} \right) \, ,
    \label{eq:chi_star}
\end{equation}
with an additional shorthand defined as $\Psi_\star := 2 \lambda_\star \left( L \pdv{\Lambda}{E} \right)_{E=1}$.
Now everything is known to start finding numerical solutions for the normalized quantities, which directly determine the physical quantities.

%%%%%%%%%%%%%%%%%%%%%%%%%%%%%%%%%%%%%%%%%%%%%%%%%%%%%%%%%%%%%%%%%%%%%%%%%%%%%%%%%%%%%%%%%%%
\subsection{Numerical solution of the isotropic NGS model}
\label{ssec:isotropic_numerics}

The goal of the procedure described below is to find numerical solutions for $v(r)$, $T(r)$, $E(r)$ and $q(r)$, which satisfy both the differential \cref{eq:ngs_normalized_dv2,eq:ngs_normalized_dT,eq:ngs_normalized_dE,eq:ngs_normalized_dq} and the boundary conditions in \cref{eq:full_boundary_conditions}. 
For this purpose, for some chosen $\gamma$ and $E_\star$, we adjust $\lambda_\star$ so that $T(r_p) = 0 = q(r_p)$ at some normalized radius $r_p<1$, which is then interpreted as the normalized pellet radius. 
Together with the values $q(\infty)$ and $E(\infty)$ all normalization constants are determined, and the full physical solution can be calculated. 
Remember that $E_\star$ denotes the incident electron energy at the sonic radius, which is not known a priori.
As a first guess, it suffices to set $E_\star \approx E_\text{bc}$, since most of the energy is absorbed only inside the sonic radius.
However, a scan over $E_\star$ allows setting it more precisely afterwards.

Many different methods exist to solve a set of first order differential equations numerically.
Knowing that 
\begin{equation*}
    v(r=1)=1,\quad T(r=1)=1,\quad E(r=1)=1,\quad q(r=1)=1,
\end{equation*}
any method would work which solves initial value problems.
Since the derivatives are determined if all quantities at one radial position are known, the full spatial dependence can be built up iteratively.
The method of numerical integration found to work well here is an \enquote{implicit multi-step variable-order (1 to 5) method based on a backward differentiation formula for the derivative approximation}\footnote{See \url{https://docs.scipy.org/doc/scipy/reference/generated/scipy.integrate.solve_ivp.html} .} based on the implementation by \textcite{shampine_matlab_1997}.
The solutions are calculated from $r=1$ in the directions of both increasing and decreasing $r$.
Optimization of $\lambda_\star$ is done through minimizing the error of the boundary conditions $T(r_p) = q(r_p) = 0$.
Here, \enquote{a modification of the Powell hybrid method as implemented in MINPACK}\footnote{See \url{https://docs.scipy.org/doc/scipy/reference/generated/scipy.optimize.root.html} .} is used \autocite{more_user_1980}.

\begin{figure}
    \centering
    \includegraphics{figure/ode0_solution_Estar_3.000e+04_gamma_7_5.pdf}
    \caption{Numerical example solution of the normalized isotropic ablation dynamics. Chosen are the parameters $\gamma = 7/5$ and $E_\star = \qty{30}{\kilo\eV} \approx E_\text{bc}$ to have a direct comparison to the solution shown by \textcite{parks_effect_1978}.}
    \label{fig:example_0th_order_solution}
\end{figure}

An example solution for $\gamma = 7/5$ and $E_\star = \qty{30}{\keV}$ is shown in \cref{fig:example_0th_order_solution}, where the subscript 0 denotes the isotropic NGS model quantities as opposed to the asymmetric NGS model quantities with a subscript 1, as introduced in the next section.
This same solution is shown by \textcite{parks_effect_1978} in their figs. 3 and 4, which serves as a validation of our implemented numerical procedure.
Visible is that $E(\infty) \approx 1$, which justifies $E_\star \approx E_\text{bc}$.

Another validation is performed by scanning over the parameters $\gamma$ and $E_\text{bc}$ and plotting the calculated results for $\lambda_\star$, $r_\text{p}$, $E(\infty)$ and $q(\infty)$, as shown in \cref{fig:ode0_scan}.
Again, we see a match to figs. 1 and 2 of the paper by \textcite{parks_effect_1978}.
Additionally, scaling laws for $\gamma = 5/3$ (dotted lines) are given in the paper, which agree well with our results.
It is important to note that in \cref{fig:ode0_scan} the energy dependence is shown on a logarithmic x-axis, while the variance on the y-axis is small and linear.
Thus, the dependence on $E_\text{bc}$ is very weak, and the quantities can be considered nearly constant.
The scaling laws representing our solution (dashed lines) are given in \cref{sec:appendix_scaling_laws}.
%(Do I want to show how well the boundary conditions are met? And that the limitation $E < \qty{20}{\eV}$ might matter a bit? Also, I might want to show $p_0(r_p)$)

\begin{figure}
    \centering
    \includegraphics{figure/recreated_Parks_E_inf.pdf}
    \caption{Weak dependence on $\gamma$ and $E_\text{bc}$ of the normalized isotropic NGS model output. The shown quantities $\lambda_\star$, $r_\text{p}$, $E_0(\infty)$ and $q_0(\infty)$ are needed to calculate the physical ablation dynamics.}
    \label{fig:ode0_scan}
\end{figure}

The solution of the normalized spherically symmetric quantities is all that is needed to model the normalized asymmetric perturbation, as described in the next section about the asymmetric NGS model.
However, to calculate any physical quantities, the sonic normalization constants, as defined in \cref{eq:ngs_normalization}, must be known.
Given the physical boundary conditions, i.e. model input parameters, $r_\text{p}$, $E_\text{bc}$ and $q_\text{bc}$, the results as presented in \cref{fig:ode0_scan} allow the direct calculation of
\begin{align}
    r_\star = \frac{r_p}{\widetilde{r}_p}, \quad 
    E_\star = \frac{E_\text{bc}}{\widetilde{E}(\infty)}, \quad \text{and} \quad
    q_\star = \frac{q_\text{bc}}{\widetilde{q}(\infty)}\,.
\end{align}
Apart from the sonic density $\rho_\star = m p_\star/T_\star$ (ideal gas law), the remaining unknown normalization constants are $p_\star$, $T_\star$ and $v_\star$.
These quantities, together with the particle ablation rate $G$, are determined by solving the system of \cref{eq:ngs_mass_conservation,eq:speed_of_sound,eq:ngs_singularity_constraint,eq:lambda_star_definition}, yielding
\begin{align}
    \Aboxed{p_\star &= \underbrace{ \frac{\lambda_\star}{\gamma} \left( \frac{\widetilde{r}_\text{p} (\gamma-1)^2}{4 \widetilde{q}^2(\infty)} \right)^\frac{1}{3} }_{f_p(E_\text{bc}, \gamma)} \cdot \left[ \frac{m (\mu q_\text{bc})^2}{\Lambda_\star r_\text{p}} \right]^\frac{1}{3}} \, , \label{eq:prefactor_p_star} \\
    T_\star &= \underbrace{ \frac{1}{\gamma} \left( \frac{\gamma-1}{2  \widetilde{r}_\text{p} \widetilde{q}(\infty)} \right)^\frac{2}{3} }_{f_T(E_\text{bc}, \gamma)} \cdot \left[ \sqrt{m} \Lambda_\star \mu q_\text{bc} r_\text{p} \right]^\frac{2}{3} \, , \label{eq:prefactor_T_star} \\
    v_\star &= \underbrace{ \left( \frac{\gamma - 1}{2 \widetilde{r}_\text{p} \widetilde{q}(\infty) } \right)^\frac{1}{3} }_{f_v(E_\text{bc}, \gamma)} \cdot \left[ \frac{\Lambda_\star \mu q_\text{bc} r_\text{p}}{m} \right]^\frac{1}{3} \, , \label{eq:prefactor_v_star} \\
    G &= \underbrace{ 4 \pi \lambda_\star \left( \frac{\gamma - 1}{2 \widetilde{r}_\text{p}^4 \widetilde{q}(\infty)} \right)^\frac{1}{3} }_{f_G(E_\text{bc}, \gamma)} \cdot \left[ \frac{\mu q_\text{bc} r_\text{p}^4}{\Lambda_\star^2 m} \right]^\frac{1}{3} \, . \label{eq:prefactor_G}
\end{align}
The physical quantities in the square brackets are the main contributions.
The prefactors $f_p$, $f_T$, $f_v$ and $f_G$ combine all dimensionless factors and are shown to only weakly depend on $E_\text{bc}$ in \cref{fig:prefactors_0th_order}.
Scaling laws for those prefactors are provided in \cref{sec:appendix_scaling_laws}.
Since the pellet rocket effect mainly depends on the pressure asymmetry at the pellet surface, the most important formula here is \cref{eq:prefactor_p_star}, where $f_p \approx 0.15$ and as stated earlier $\mu \approx 0.65$.
The importance of $p_\star$ will become clearer towards the end of the next section, in which our approach of modelling asymmetry in the ablation dynamics is detailed.
%(Should mention the mistakes that I have found in Parks paper in his formulas?)

\begin{figure}
    \centering
    \includegraphics{figure/prefactors_0th_order.pdf}
    \caption{Weak dependence on $\gamma$ and $E_\text{bc}$ of the dimensionless prefactors for $p_\star$, $T_\star$, $v_\star$ and $G$ of the isotropic NGS model in \cref{eq:prefactor_p_star,eq:prefactor_T_star,eq:prefactor_v_star,eq:prefactor_G}.}
    \label{fig:prefactors_0th_order}
\end{figure}

