\section{Asymmetric NGS model expressions derived with SymPy}
\label{sec:appendix_sympy_expressions}

\begingroup
\renewcommand*{\arraystretch}{2}
This section provides the full expressions found for the normalized asymmetric NGS model, as explained in \cref{ssec:asymmetric_perturbation_model}.
Algebraic operations were performed using the Python package SymPy, since the expressions are too large to handle manually.
Consider the system of \cref{eq:physical_perturbation_ideal_gas_law,eq:physical_perturbation_mass_conservation,eq:physical_perturbation_r_momentum_conservation,eq:physical_perturbation_energy_conservation,eq:physical_perturbation_theta_momentum_conservation,eq:physical_perturbation_effective_heat_flux,eq:physical_perturbation_electron_energy_loss} with $l=1$ and the changes due to normalization listed after \cref{eq:perturbation_normalized_bc}.
It describes the normalized perturbation $\vec{y}_1$ to the isotropic NGS model $\vec{y}_0$ and is now written in the form
\begin{equation}
    A \pdv{\vec{y}_1}{r} = B \vec{y}_1 \quad \text{with} \quad \vec{y}_1 = (p_1, T_1, v_{1,r}, v_{1,\theta}, q_1, E_1)^T \, .
\end{equation}
Since all the following expressions contain only the isotropic NGS model quantities, the subscript 0 is omitted.
We also denote $y' = \partial y/ \partial r$ and $\lambda = \lambda_\star$.
The corresponding matrices are
\begin{gather}
    A = \begin{pmatrix}- \dfrac{v}{T} & \dfrac{\rho v}{T} & - \rho & 0 & 0 & 0\\- \dfrac{1}{\gamma} & 0 & - \rho v & 0 & 0 & 0\\0 & 0 & 0 & - \rho v & 0 & 0\\- \dfrac{v}{\gamma - 1} - \dfrac{v^{3}}{2 T} & \dfrac{\rho v^{3}}{2 T} & - \rho v^{2} - k & 0 & \dfrac{2}{\lambda \left(\gamma - 1\right)} & 0\\0 & 0 & 0 & 0 & 1 & 0\\0 & 0 & 0 & 0 & 0 & 1\end{pmatrix} \\
    \intertext{and}
    B = \\\begin{pmatrix}\dfrac{\partial }{\partial r } \left( \dfrac{1}{T} \right) v + \dfrac{1}{T} \left(\vec{\nabla} \cdot \vec{v}\right) & - \dfrac{\partial }{\partial r } \left( \dfrac{\rho}{T} \right) v - \dfrac{\rho}{T} \left(\vec{\nabla} \cdot \vec{v}\right) & \dfrac{2}{r} \rho + \rho' & - \dfrac{2}{r} \rho & 0 & 0\\\dfrac{v}{T} v' & - \dfrac{\rho}{T} v v' & \rho v' & 0 & 0 & 0\\\dfrac{1}{\gamma r} & 0 & 0 & \dfrac{\rho}{r} v & 0 & 0\\B_{41} & - \dfrac{v}{2} \dfrac{\partial }{\partial r } \left( \dfrac{\rho v^{2}}{T} \right) - \dfrac{\left(\vec{\nabla} \cdot \vec{v}\right)}{2 T} \rho v^{2} & B_{43} & - \dfrac{2}{r} k & 0 & 0\\\dfrac{\Lambda}{T} \lambda q & - \dfrac{\Lambda}{T} \rho \lambda q & 0 & 0 & \Lambda \rho \lambda & \dfrac{\partial \Lambda}{\partial E } \rho \lambda q\\\dfrac{2}{T} L \lambda & - \dfrac{2}{T} L \rho \lambda & 0 & 0 & 0 & 2 \dfrac{\partial L}{\partial E } \rho \lambda\end{pmatrix}\nonumber ,\displaybreak[0] \\ \text{with} \nonumber \displaybreak[0] \\ 
B_{41} = \dfrac{v}{2} \dfrac{\partial }{\partial r } \left( \dfrac{v^{2}}{T} \right) + \left(\vec{\nabla} \cdot \vec{v}\right) \left(\dfrac{1}{\gamma - 1} + \dfrac{v^{2}}{2 T}\right),\displaybreak[0] \\ 
B_{43} = \dfrac{\partial }{\partial r } \left( \rho v \right) v + \left(\vec{\nabla} \cdot \vec{v}\right) \rho v + \dfrac{2}{r} k + k' \, .
\end{gather}
We have denoted
\begin{gather}
    k = \frac{1}{2} \rho v^2 + \frac{p}{\gamma-1} \, , \\
    \vec{\nabla} \cdot \vec{v} = \frac{1}{r^2} \cdot \frac{d}{dr}(r^2 v) \, .
\end{gather}
Symbolic computation allows us to find an analytic expression for $C = A^{-1}B$ so that
\begin{equation}
\boxed{%
    \pdv{\vec{y}_1}{r} = C \vec{y}_1} \, .
\end{equation}
The expression for matrix $C$ in terms of $3 \times 3$ blocks with $w=v^2$ is provided on the next page.

\begin{landscape}
\begin{gather}
    C = \begin{pmatrix} C_{11} & C_{12} \\ C_{21} & C_{22} \end{pmatrix} \displaybreak[0] \\
C_{11} = \frac{1}{T-v^2} \begin{pmatrix}\dfrac{2 \gamma w}{r} - \dfrac{2 \Lambda \gamma q \sqrt{w}}{T} & \dfrac{\gamma w'}{2 r^{2} \sqrt{w}} + \dfrac{2 \Lambda \gamma q}{T r^{2}} & \dfrac{\gamma \left(- T w' + w \left(T' + \dfrac{\gamma w'}{2} - \dfrac{w'}{2}\right)\right)}{r^{2} w}\\(D_{11})_{21} & (D_{11})_{22} & (D_{11})_{23}\\\dfrac{r \left(- \dfrac{T r w'}{2} - 2 T w + 2 \Lambda q r \sqrt{w} + \dfrac{r w w'}{2}\right)}{T} & \dfrac{- 2 \Lambda q - \dfrac{\sqrt{w} w'}{2}}{T} & \dfrac{T w'}{2 w} - T' - \dfrac{\gamma w'}{2} + w'\end{pmatrix} \displaybreak[0] \\
\displaybreak[0] \\ 
(D_{11})_{21} = - T' r^{2} \sqrt{w} + 2 \Lambda q r^{2} + 2 \gamma r w^{3/2} - 2 r w^{3/2} + \dfrac{T' r^{2} w^{3/2}}{T} - \dfrac{2 \Lambda \gamma q r^{2} w}{T}\displaybreak[0] \\ 
(D_{11})_{22} = T' - \dfrac{2 \Lambda q}{\sqrt{w}} + \dfrac{\gamma w'}{2} - \dfrac{w'}{2} - \dfrac{T' w}{T} + \dfrac{2 \Lambda \gamma q \sqrt{w}}{T}\displaybreak[0] \\ 
(D_{11})_{23} = \dfrac{- T T' - \dfrac{3 T \gamma w'}{2} + \dfrac{3 T w'}{2} + T' \gamma w + \dfrac{\gamma^{2} w w'}{2} - \dfrac{\gamma w w'}{2}}{\sqrt{w}}\displaybreak[0] \\
C_{12} = \frac{1}{T-v^2} \begin{pmatrix}- \dfrac{2 T \gamma}{r^{3}} & - \dfrac{2 \Lambda \gamma}{r^{2}} & - \dfrac{2 \dfrac{\partial \Lambda}{\partial E } \gamma q}{r^{2}}\\\dfrac{2 T \sqrt{w} \left(1 - \gamma\right)}{r} & \dfrac{2 \Lambda \left(T - \gamma w\right)}{\sqrt{w}} & \dfrac{2 \dfrac{\partial \Lambda}{\partial E } q \left(T - \gamma w\right)}{\sqrt{w}}\\\dfrac{2 T}{r} & 2 \Lambda & 2 \dfrac{\partial \Lambda}{\partial E } q\end{pmatrix} \displaybreak[0] \\
C_{21} = \begin{pmatrix}- \dfrac{r}{\gamma} & 0 & 0\\\dfrac{\Lambda \lambda q}{T} & - \dfrac{\Lambda \lambda q}{T r^{2} \sqrt{w}} & 0\\\dfrac{2 L \lambda}{T} & - \dfrac{2 L \lambda}{T r^{2} \sqrt{w}} & 0\end{pmatrix} \displaybreak[0] \\
C_{22} = \begin{pmatrix}- \dfrac{1}{r} & 0 & 0\\0 & \dfrac{\Lambda \lambda}{r^{2} \sqrt{w}} & \dfrac{\dfrac{\partial \Lambda}{\partial E } \lambda q}{r^{2} \sqrt{w}}\\0 & 0 & \dfrac{2 \dfrac{\partial L}{\partial E } \lambda}{r^{2} \sqrt{w}}\end{pmatrix} \, .
\end{gather}
\end{landscape}

As can be seen, the first three rows of $C$ share the apparent singularity $1/(T_0 - v_0^2)$ at $r=1$.
We now require that $\vec{y}'_1(r=1)$ is finite.
Defining the singularity-reduced matrix $D = (T_0 - v_0^2)C$, the first three rows of $D(r=1)$ are
\begin{gather}
    D_{1}(r=1) = \begin{pmatrix}0 & \gamma \left(\dfrac{\chi}{2} + 2\right) & \gamma \left(2 - \chi\right) & - 2 \gamma & - 2 \gamma & - \dfrac{\Psi \gamma}{L \lambda}\\0 & \dfrac{\chi \left(\gamma - 1\right)}{2} + 2 \gamma - 2 & - \chi \gamma + \chi + 2 \gamma - 2 & 2 - 2 \gamma & 2 - 2 \gamma & \dfrac{\Psi \left(1 - \gamma\right)}{L \lambda}\\0 & - \dfrac{\chi}{2} - 2 & \chi - 2 & 2 & 2 & \dfrac{\Psi}{L \lambda}\end{pmatrix} \, ,
\end{gather}
where $\chi$ and $\Psi$ are defined by \cref{eq:chi_star} and $L$ is the empirical electron energy loss function of \cref{eq:energy_loss_function} normalized through \cref{eq:ngs_normalization}.
This is a matrix of rank 1 and $D(r=1)\vec{y}_1$ yields
\begin{equation}
\boxed{%
    v_{1,\theta} = \left(1- \frac{\chi_\star}{2}\right) v_{1,r} + \left(1 + \frac{\chi_\star}{4}\right)T_1 - q_1 - \frac{\Psi_\star}{2 \lambda_\star L(E=1)} E_1} \, .
\end{equation}
This is used in the numerical procedure for solving for $\vec{y}_1(r)$ to reduce the number of unknowns at the sonic radius by one.

To be able to calculate the derivatives $\vec{y}'_1(r=1)$, we need to apply L'Hôpital's rule in the sense
\begin{gather*}
\lim_{r\rightarrow 1} \vec{y}'_1 = \lim_{r\rightarrow 1} \frac{1}{\frac{d}{dr}(T-w)} \frac{d}{dr} \left( D \vec{y}_1 \right) \\
\Leftrightarrow \lim_{r\rightarrow 1} \vec{y}'_1 = \lim_{r\rightarrow 1} \frac{1}{\frac{d}{dr}(T-w)} \left( D' \vec{y}_1 + D\vec{y}'_1 \right) \\
\Leftrightarrow \lim_{r\rightarrow 1} \left(\frac{d}{dr}(T-w)\mathbbone{}_{6x6} - D \right) \vec{y}'_1 = \lim_{r\rightarrow 1} D' \vec{y}_1 \\
\Leftrightarrow \lim_{r\rightarrow 1} \vec{y}'_1 = \lim_{r\rightarrow 1} \left(\frac{d}{dr}(T-w)\mathbbone{}_{6x6} - D \right)^{-1} D' \vec{y}_1.
\end{gather*}
So we need to find $D'(r=1)$ and $\left(\frac{d}{dr}(T-w)\mathbbone{}_{6x6} - D \right)^{-1}|_{r=1}$. 
We also define the matrix
\begin{gather}
C_\star = \lim_{r\rightarrow 1} \left(\frac{d}{dr}(T-w)\mathbbone{}_{6x6} - D \right)^{-1} D' \\
\text{so that} \quad \Rightarrow \quad \boxed{\vec{y}'_1(r=1) = C_\star \vec{y}_1(r=1)} \, .
\end{gather}
We now define $\Xi = 4 \lambda L^2|_{E=1} \frac{d^2 \Lambda}{dE^2}|_{E=1} $ and provide the resulting large expressions for the elements of $C_\star$ on the next pages.
\begin{landscape}
\begin{gather*}
C_{\star(4-6)(1-6)} = \begin{pmatrix}- \dfrac{1}{\gamma} & 0 & 0 & -1 & 0 & 0\\\lambda & - \lambda & 0 & 0 & \lambda & \dfrac{\Psi}{2 L}\\2 L \lambda & - 2 L \lambda & 0 & 0 & 0 & 2 \lambda \left. \dfrac{\partial L}{\partial \xi } \right|_{\substack{ \xi=1 }}\end{pmatrix}\displaybreak[0] \\C_{\star11} = \dfrac{16 \Psi \gamma - \chi^{2} \gamma^{2} - \chi^{2} \gamma + 16 \gamma \lambda - 8 \gamma - 8}{4 \left(\chi \gamma + \chi + 2 \gamma - 6\right)}
\displaybreak[0] \\C_{\star12} = \dfrac{\gamma \left(- 16 \Psi \chi \gamma - 16 \Psi \chi + 64 \Psi + \chi^{3} \gamma^{2} + 4 \chi^{3} \gamma + \chi^{3} + 8 \chi^{2} \gamma - 12 \chi^{2} - 16 \chi \gamma \lambda + 80 \chi \gamma - 16 \chi \lambda + 64 \gamma + 64 \lambda - 256\right)}{4 \left(\chi^{2} \gamma^{2} + 2 \chi^{2} \gamma + \chi^{2} + 2 \chi \gamma^{2} - 8 \chi \gamma - 10 \chi - 8 \gamma + 24\right)}
\displaybreak[0] \\C_{\star13} = \dfrac{\gamma \left(- 4 \Psi \chi \gamma - 4 \Psi \chi + 16 \Psi - \chi^{3} \gamma^{2} - 4 \chi^{3} \gamma - \chi^{3} + 4 \chi^{2} \gamma + 12 \chi^{2} - 4 \chi \gamma \lambda + 16 \chi \gamma - 4 \chi \lambda + 24 \chi + 32 \gamma + 16 \lambda - 128\right)}{2 \left(\chi^{2} \gamma^{2} + 2 \chi^{2} \gamma + \chi^{2} + 2 \chi \gamma^{2} - 8 \chi \gamma - 10 \chi - 8 \gamma + 24\right)}
\displaybreak[0] \\C_{\star14} = \dfrac{\gamma \left(- \chi^{2} \gamma^{2} - \chi^{2} \gamma + 2 \chi^{2} - 8 \chi \gamma - 8 \chi - 16 \gamma + 48\right)}{\chi^{2} \gamma^{2} + 2 \chi^{2} \gamma + \chi^{2} + 2 \chi \gamma^{2} - 8 \chi \gamma - 10 \chi - 8 \gamma + 24}
\displaybreak[0] \\C_{\star15} = \dfrac{2 \gamma \left(\Psi \chi \gamma + \Psi \chi - 4 \Psi - \chi^{2} \gamma + \chi \gamma \lambda - 8 \chi \gamma + \chi \lambda - 2 \chi - 8 \gamma - 4 \lambda + 32\right)}{\chi^{2} \gamma^{2} + 2 \chi^{2} \gamma + \chi^{2} + 2 \chi \gamma^{2} - 8 \chi \gamma - 10 \chi - 8 \gamma + 24}
\displaybreak[0] \\C_{\star16} = \dfrac{\gamma \left(- \Psi \chi^{2} \gamma + 2 \Psi \chi \gamma \lambda \left. \dfrac{\partial L}{\partial \xi } \right|_{\substack{ \xi=1 }} + 2 \Psi \chi \gamma \lambda - 8 \Psi \chi \gamma + 2 \Psi \chi \lambda \left. \dfrac{\partial L}{\partial \xi } \right|_{\substack{ \xi=1 }} + 2 \Psi \chi \lambda - 2 \Psi \chi - 8 \Psi \gamma - 8 \Psi \lambda \left. \dfrac{\partial L}{\partial \xi } \right|_{\substack{ \xi=1 }} - 8 \Psi \lambda + 32 \Psi + \Xi \chi \gamma \lambda + \Xi \chi \lambda - 4 \Xi \lambda\right)}{L \lambda \left(\chi^{2} \gamma^{2} + 2 \chi^{2} \gamma + \chi^{2} + 2 \chi \gamma^{2} - 8 \chi \gamma - 10 \chi - 8 \gamma + 24\right)}
\displaybreak[0] \\C_{\star21} = \dfrac{16 \Psi \gamma^{2} - 16 \Psi \gamma + \chi^{2} \gamma^{3} - \chi^{2} \gamma + 4 \chi \gamma^{3} - 16 \chi \gamma^{2} + 12 \chi \gamma + 16 \gamma^{2} \lambda - 8 \gamma^{2} - 16 \gamma \lambda + 8}{4 \gamma \left(\chi \gamma + \chi + 2 \gamma - 6\right)}
\displaybreak[0] \\C_{\star22} = \dfrac{- 16 \Psi \chi \gamma^{2} + 16 \Psi \chi + 64 \Psi \gamma - 64 \Psi - \chi^{3} \gamma^{3} - \chi^{3} \gamma^{2} - \chi^{3} \gamma + 3 \chi^{3} - 4 \chi^{2} \gamma^{3} + 8 \chi^{2} \gamma^{2} - 4 \chi^{2} - 16 \chi \gamma^{2} \lambda + 32 \chi \gamma^{2} - 16 \chi \gamma + 16 \chi \lambda - 16 \chi + 64 \gamma \lambda - 64 \gamma - 64 \lambda + 64}{4 \chi^{2} \gamma^{2} + 8 \chi^{2} \gamma + 4 \chi^{2} + 8 \chi \gamma^{2} - 32 \chi \gamma - 40 \chi - 32 \gamma + 96}
\displaybreak[0] \\C_{\star23} = \dfrac{\left(\chi \gamma + 3 \chi - 8\right) \left(- 4 \Psi \gamma + 4 \Psi - \chi^{2} \gamma^{2} + \chi^{2} \gamma + 2 \chi \gamma - 6 \chi - 4 \gamma \lambda + 4 \lambda + 8\right) + \left(\chi \gamma - \chi - 2 \gamma + 2\right) \left(8 \Psi + \chi^{2} \gamma + \chi^{2} - 8 \chi + 8 \lambda\right)}{2 \left(\chi^{2} \gamma^{2} + 2 \chi^{2} \gamma + \chi^{2} + 2 \chi \gamma^{2} - 8 \chi \gamma - 10 \chi - 8 \gamma + 24\right)}
\displaybreak[0] \\C_{\star24} = \dfrac{\chi \left(- \chi \gamma^{3} + 2 \chi \gamma^{2} + 3 \chi \gamma - 4 \chi + 4 \gamma^{2} - 16 \gamma + 12\right)}{\chi^{2} \gamma^{2} + 2 \chi^{2} \gamma + \chi^{2} + 2 \chi \gamma^{2} - 8 \chi \gamma - 10 \chi - 8 \gamma + 24}
\displaybreak[0] \\C_{\star25} = \dfrac{2 \left(- 2 \Psi \left(\chi \gamma - \chi - 2 \gamma + 2\right) + \lambda \left(\gamma - 1\right) \left(\chi \gamma + \chi - 4\right) + \left(\chi \gamma + 3 \chi - 8\right) \left(\Psi \gamma - \Psi + \chi \gamma - 2\right)\right)}{\chi^{2} \gamma^{2} + 2 \chi^{2} \gamma + \chi^{2} + 2 \chi \gamma^{2} - 8 \chi \gamma - 10 \chi - 8 \gamma + 24}
\displaybreak[0] \\C_{\star26} = \dfrac{\lambda \left(2 \Psi \left(\gamma - 1\right) \left(\chi \gamma + \chi - 4\right) \left. \dfrac{\partial L}{\partial \xi } \right|_{\substack{ \xi=1 }} + \Psi \left(\gamma - 1\right) \left(\chi \gamma + \chi - 4\right) - 2 \left(\Psi + \Xi\right) \left(\chi \gamma - \chi - 2 \gamma + 2\right)\right) - \dfrac{1}{2} \left(\Psi \left(\chi \left(\gamma - 1\right) - \chi \left(3 \gamma - 1\right) + 4\right) - 2 \lambda \left(\Psi + \Xi\right) \left(\gamma - 1\right)\right) \left(\chi \gamma + 3 \chi - 8\right)}{L \lambda \left(\chi^{2} \gamma^{2} + 2 \chi^{2} \gamma + \chi^{2} + 2 \chi \gamma^{2} - 8 \chi \gamma - 10 \chi - 8 \gamma + 24\right)}
\displaybreak[0] \\C_{\star31} = \dfrac{- 16 \Psi \gamma - \chi^{2} \gamma^{2} - \chi^{2} \gamma - 4 \chi \gamma^{2} + 12 \chi \gamma - 16 \gamma \lambda + 8 \gamma + 8}{4 \gamma \left(\chi \gamma + \chi + 2 \gamma - 6\right)}
\displaybreak[0] \\C_{\star32} = \dfrac{4 \Psi \chi \gamma + 4 \Psi \chi - 16 \Psi + \dfrac{\chi^{3} \gamma^{2}}{4} + \dfrac{\chi^{3}}{4} + \chi^{2} \gamma^{2} - 4 \chi^{2} \gamma + 4 \chi \gamma \lambda - 12 \chi \gamma + 4 \chi \lambda + 8 \chi - 16 \lambda + 16}{\chi^{2} \gamma^{2} + 2 \chi^{2} \gamma + \chi^{2} + 2 \chi \gamma^{2} - 8 \chi \gamma - 10 \chi - 8 \gamma + 24}
\displaybreak[0] \\C_{\star33} = \dfrac{2 \Psi \chi \gamma + 2 \Psi \chi - 8 \Psi + \chi^{3} \gamma - \chi^{2} \gamma^{2} - 2 \chi^{2} \gamma - 5 \chi^{2} + 2 \chi \gamma \lambda - 4 \chi \gamma + 2 \chi \lambda + 8 \chi - 8 \lambda + 16}{\chi^{2} \gamma^{2} + 2 \chi^{2} \gamma + \chi^{2} + 2 \chi \gamma^{2} - 8 \chi \gamma - 10 \chi - 8 \gamma + 24}
\displaybreak[0] \\C_{\star34} = \dfrac{\chi^{2} \left(\gamma^{2} + \gamma - 2\right)}{\chi^{2} \gamma^{2} + 2 \chi^{2} \gamma + \chi^{2} + 2 \chi \gamma^{2} - 8 \chi \gamma - 10 \chi - 8 \gamma + 24}
\displaybreak[0] \\C_{\star35} = \dfrac{2 \left(- 2 \Psi \left(\chi \gamma + 2 \gamma - 4\right) - \lambda \left(\chi \gamma + \chi - 4\right) + \left(\chi + 4\right) \left(\Psi \gamma - \Psi + \chi \gamma - 2\right)\right)}{\chi^{2} \gamma^{2} + 2 \chi^{2} \gamma + \chi^{2} + 2 \chi \gamma^{2} - 8 \chi \gamma - 10 \chi - 8 \gamma + 24}
\displaybreak[0] \\C_{\star36} = \dfrac{\lambda \left(- 2 \Psi \left(\chi \gamma + \chi - 4\right) \left. \dfrac{\partial L}{\partial \xi } \right|_{\substack{ \xi=1 }} - \Psi \left(\chi \gamma + \chi - 4\right) - 2 \left(\Psi + \Xi\right) \left(\chi \gamma + 2 \gamma - 4\right)\right) - \dfrac{1}{2} \left(\chi + 4\right) \left(\Psi \left(\chi \left(\gamma - 1\right) - \chi \left(3 \gamma - 1\right) + 4\right) - 2 \lambda \left(\Psi + \Xi\right) \left(\gamma - 1\right)\right)}{L \lambda \left(\chi^{2} \gamma^{2} + 2 \chi^{2} \gamma + \chi^{2} + 2 \chi \gamma^{2} - 8 \chi \gamma - 10 \chi - 8 \gamma + 24\right)}

.
\end{gather*}
\end{landscape}
\endgroup
